\documentclass{article}
\usepackage{amsmath}
\usepackage{amssymb}
\usepackage{mathtools}
\usepackage{graphicx}
\usepackage{enumitem}
\usepackage[margin=1in]{geometry}
\usepackage{titling}

\DeclareMathOperator{\Var}{Var}

\setlength{\droptitle}{-7em}   % This is your set screw

\begin{document}

\title{Math Statistics\\ Semiweekly HW 4}
\author{Ozaner Hansha}
\date{September 15, 2020}
\maketitle

\subsection*{Question 1}
\noindent\textbf{Problem:} The square root of the sample variance $\sigma_{\bar{X}}$ is generally called the standard error of the sample. If a sample size is changed from $n=30$ to $n=90$, how does this affect the standard error?
\bigskip

\noindent\textbf{Solution:} Recall that the standard error of the sample is given by:
$$\sigma_{\bar{X}}=\frac{\sigma}{\sqrt{n}}$$

From this definition, it is clear that the standard error is inversely proportional to the square root of the sample size. That is to say, as $n$ increases, $\sigma_{\bar{X}}$ decreases.
\end{document}
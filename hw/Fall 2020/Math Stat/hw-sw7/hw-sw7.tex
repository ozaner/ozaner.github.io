\documentclass{article}
\usepackage{amsmath}
\usepackage{amssymb}
\usepackage{mathtools}
\usepackage{graphicx}
\usepackage{enumitem}
\usepackage[margin=1in]{geometry}
\usepackage{titling}

\DeclareMathOperator{\Var}{Var}

\setlength{\droptitle}{-7em}   % This is your set screw

\begin{document}

\title{Math Statistics\\ Semiweekly HW 7}
\author{Ozaner Hansha}
\date{September 25, 2020}
\maketitle

\subsection*{Question 1}
Suppose we have a normally distributed population with unknown variance $\sigma^2$. Let $S^2$ be the sample variance of a sample with $n=5$.
\bigskip

\noindent\textbf{Part a:} Find $x_a$ such that:
$$P\left(\frac{4S^2}{\sigma^2}>x_a\right)=.05$$
\smallskip

\noindent\textbf{Solution:} Recall that $\frac{4S^2}{\sigma^2}\sim \chi^2_4$. And so, consulting our chi-square table, we have:
$$x_a\approx9.488$$
\smallskip

\noindent\textbf{Part b:} Find $x_b$ such that:
$$P\left(\frac{4S^2}{\sigma^2}<x_b\right)=.05$$
\smallskip

\noindent\textbf{Solution:} First note that:
\begin{align*}
    .05&=P\left(\frac{4S^2}{\sigma^2}<x_b\right)\\
    &=1-P\left(\frac{4S^2}{\sigma^2}\ge x_b\right)\tag{complement}\\
    .95&=P\left(\frac{4S^2}{\sigma^2}\ge x_b\right)
\end{align*}


Again, we have that $\frac{4S^2}{\sigma^2}\sim \chi^2_4$. And so, consulting our chi-square table, we have:
$$x_b=.711$$
\smallskip

\noindent\textbf{Part c:} If we observe our sample variance to be $s^2=12$, give a 90\% confidence interval of the population variance $\sigma^2$.
\bigskip

\noindent\textbf{Solution:} Note the following two events are mutually exclusive:
$$P\left(\frac{4S^2}{\sigma^2}<x_b\cap\frac{4S^2}{\sigma^2}>x_a\right)=0$$

And so we have the following:
\begin{align*}
    .05+.05&=P\left(\frac{4S^2}{\sigma^2}>x_a\right)+P\left(\frac{4S^2}{\sigma^2}<x_b\right)\\
    .1&=P\left(\frac{4S^2}{\sigma^2}>x_a\cup\frac{4S^2}{\sigma^2}<x_b\right)\tag{mutually exclusive}\\
    &=1-P\left(x_b<\frac{4S^2}{\sigma^2}<x_a\right)\tag{complement \& deMorgan}\\
    .9&=P\left(x_b<\frac{4S^2}{\sigma^2}<x_a\right)\\
    &=P\left(\frac{1}{x_b}>\frac{\sigma^2}{4S^2}>\frac{1}{x_a}\right)\tag{$\sigma^2,S^2,x_a,x_b$ are positive}\\
    &=P\left(\frac{4S^2}{x_b}>\sigma^2>\frac{4S^2}{x_a}\right)
\end{align*}

And so, our 90\% confidence interval of $\sigma^2$ is given by:
\begin{align*}
    \left[\frac{4s^2}{x_a},\frac{4s^2}{x_b}\right]&=\left[\frac{4*12}{x_a},\frac{4*12}{x_b}\right]\tag{$s^2=12$}\\
    &\approx\left[\frac{48}{9.488},\frac{48}{.711}\right]\tag{$x_a=9.488,x_b=.711$}\\
    &\approx[5.059,67.511]
\end{align*}

\end{document}
\documentclass{article}
\usepackage{amsmath}
\usepackage{amssymb}
\usepackage{mathtools}
\usepackage{graphicx}
\usepackage{enumitem}
\usepackage[margin=1in]{geometry}
\usepackage{titling}

\DeclareMathOperator{\Var}{Var}

\setlength{\droptitle}{-7em}   % This is your set screw

\begin{document}

\title{Math Statistics\\ Semiweekly HW 3}
\author{Ozaner Hansha}
\date{September 15, 2020}
\maketitle

\subsection*{Question 1}
\noindent\textbf{Problem:} Suppose that in the Calculus II classes at Utgard University, the mean score on the final exam is a 74\% with a standard deviation of 10\%. In a class of 100 students, assuming that they are randomly chosen, what is the probability that the mean score is at least 76\%?
\bigskip

\noindent\textbf{Solution:} First note that we expect test scores to have a normal distribution. As such, each of the i.i.d. 100 sampled scores $S_i$ should have the following distribution:
$$S_i\sim\mathcal N(.74,.01)$$

\textit{*note that since $\sigma=.1$ we have that $\sigma^2=.01$.}
\smallskip

Now recall that the sum of normal RVs is itself a normal RV whose mean is the sum of the summands' mean and whose variance is the sum of the summands' variance. In this case we have $\mu=100*.74$ and $\sigma^2=100*.01$ since all $S_i$ are distributed i.i.d.:
$$\sum_{i=1}^{100} S_i=S\sim\mathcal{N}\left(74,1\right)$$

And so our desired probability is given below:
\begin{align*}
  P\left(\frac{1}{100}S\ge .76\right)&=P\left(S\ge 76\right)\\
  &=1-P\left(S\le 76\right)\tag{complement}\\
  &=1-F_S(76)\tag{def. of cdf}\\
  &=1-\Phi\left(\frac{76-74}{1}\right)\tag{unit normal RV}\\
  &=1-\Phi(2)=\Phi(-2)\tag{complement of unit RV cdf}\\
  &\approx0.02275\tag{unit normal table lookup}
\end{align*}
\end{document}
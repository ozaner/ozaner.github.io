\documentclass{article}
\usepackage{amsmath}
\usepackage{amssymb}
\usepackage{mathtools}
\usepackage{graphicx}
\usepackage{enumitem}
\usepackage[margin=1in]{geometry}
\usepackage{titling}

\DeclareMathOperator{\Var}{Var}

\setlength{\droptitle}{-7em}   % This is your set screw

\begin{document}

\title{Math Statistics\\ Semiweekly HW 2}
\author{Ozaner Hansha}
\date{September 9, 2020}
\maketitle

\textit{*note that we characterize gamma distributions in terms of their shape $\alpha$ and their rate $\beta$.}

\subsection*{Question 1}
Suppose $X_1$ and $X_2$ are i.i.d, with support $(0,\infty)$ and the following pdf:
$$f(x)=3e^{-3x}$$
\smallskip

\noindent\textbf{Part a:} Are $X_1$ and $X_2$ gamma RVs?
\bigskip

\noindent\textbf{Solution:} Yes. To see this, note that all exponential RVs are gamma RVs:
\begin{align*}
  &\phantom{=\,\,}\frac{\beta^\alpha}{\Gamma(\alpha)}x^{\alpha-1}e^{-\beta x}\tag{pdf of gamma RV}\\
  &=\frac{\lambda^1}{\Gamma(1)}x^{1-1}e^{-\lambda x}\tag{let $\alpha=1$ and $\beta=\lambda$}\\
  &=\lambda e^{-\lambda x}\tag{pdf of exponential RV}
\end{align*}

Since $X_1$ and $X_2$ are clearly exponential RVs with paramater $\lambda=3$, they must also be gamma RVs with parameters $\alpha=1$ and $\beta=3$.
\bigskip
% \begin{align*}
%   &\phantom{=}\frac{x^{k-1}e^{-\frac{x}{\theta}}}{\Gamma(k)\theta^k}\tag{pdf of gamma RV}\\
%   &=\frac{x^{1-1}e^{-\frac{x}{\lambda^{-1}}}}{\Gamma(1)\lambda^{-1}}\tag{let $k=1$ and $\theta=\lambda^{-1}$}\\
%   &=\lambda e^{-\lambda x}\tag{pdf of exponential RV}
% \end{align*}

% Since $X_1$ and $X_2$ are clearly exponential RVs with paramater $\lambda=3$, they must also be gamma RVs with parameters $k=1$ and $\theta=\frac{1}{3}$.
% \bigskip

\noindent\textbf{Part b:} Find the pdf of $X_1+X_2$.
\bigskip

\noindent\textbf{Solution:} Since they are gamma RVs, we must have $X_1\sim\text{Gamma}(a_1,\beta)$ and $X_2\sim\text{Gamma}(a_2,\beta)$, for some $a_1,a_2$. Now consider the mgf of their sum $X_1+X_2$:
\begin{align*}
  \mathcal M_{X_1+X_2}(t)&=E[e^{t(X_1+X_2)}]\tag{def. of mgf}\\
  &=E[e^{tX_1}e^{tX_2}]\tag{power rule}\\
  &=E[e^{tX_1}]E[e^{tX_2}]\tag{independence of $X_1$ and $X_2$}\\
  &=\left(\frac{\beta}{\beta-t}\right)^{a_1}\left(\frac{\beta}{\beta-t}\right)^{a_2}\tag{mgf of a gamma RV}\\
  &=\left(\frac{\beta}{\beta-t}\right)^{a_1+a_2}\tag{power rule}
\end{align*}

As we can see, the mgf of $X_1+X_2$ is that of a gamma RV. In particular, we know that $\alpha_1+\alpha_2=2$ and that $\beta=3$. And so we have that $X_1+X_2\sim\text{Gamma}(2,3)$. The corresponding pdf for such a RV is given by:
\begin{align*}
  f_{X_1+X_2}(x)&=\frac{3^2}{\Gamma(2)}x^{2-1}e^{-3x}\tag{pdf of gamma RV}\\
  &\boxed{=9xe^{-3x}}
\end{align*}
\smallskip


\end{document}
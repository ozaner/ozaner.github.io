\documentclass{article}
\usepackage{amsmath}
\usepackage{amssymb}
\usepackage{mathtools}
\usepackage{graphicx}
\usepackage{enumitem}
\usepackage[margin=1in]{geometry}
\usepackage{titling}
\usepackage[dvipsnames]{xcolor}
\usepackage{graphicx}
\usepackage{tikz}
\usepackage{pgfplots}
\usetikzlibrary{arrows}
\usetikzlibrary{datavisualization.formats.functions}
\usepgfplotslibrary{fillbetween}
\usetikzlibrary{patterns}

\renewcommand{\vec}[1]{\mathbf{#1}}

\setlength{\droptitle}{-7em}   % This is your set screw

\begin{document}

\title{Foundations of QM\\ HW 2}
\author{Ozaner Hansha}
\date{October 12, 2020}
\maketitle

For the following questions consider the wavefunction $\psi_s$ given by:
$$\psi_s(x)=\frac{1}{4\sqrt{3}}x^2e^{-|x|/2},\quad x\in\mathbb R$$

And the wave function $\psi_a$ given by:
$$\psi_a(x)=\begin{cases}
    \psi_s(x),&x\ge 0\\
    -\psi_s(x),&x<0
\end{cases}$$

Also, let $X_n$ be a random variable that gives the position of a particle with wave function $\psi_n$ when measured.
\bigskip

\subsection*{Question 1}
\noindent\textbf{Problem:} Sketch $\psi_s(x)$: 
\bigskip

\noindent\textbf{Solution:} 
\begin{center}
    \begin{tikzpicture}
    \begin{axis}[
        xlabel={x},
        xmin=-10,xmax=10,
        ymin=-2,ymax=2,
        axis lines=center,
        legend style={legend cell align=right,legend plot pos=right}]
      
    \addplot[color=BrickRed,domain=-10:10,samples=100] {1/(4*sqrt(4))*x^2*e^(-abs(x)/2)};
    \addlegendentry{$\psi_s(x)$}
    \end{axis}
    \end{tikzpicture}
\end{center}
\newpage

\subsection*{Question 2}
\noindent\textbf{Problem:} Sketch $\psi_a(x)$: 
\bigskip

\noindent\textbf{Solution:} 
\begin{center}
    \begin{tikzpicture}
    \begin{axis}[
        xlabel={x},
        xmin=-10,xmax=10,
        ymin=-2,ymax=2,
        axis lines=center,
        legend style={legend cell align=right,legend plot pos=right}]
      
    \addplot[color=Purple,domain=0:10,samples=100] {1/(4*sqrt(4))*x^2*e^(-abs(x)/2)};
    \addlegendentry{$\psi_a(x)$}
    \addplot[color=Purple,domain=-10:0,samples=100] {-1/(4*sqrt(4))*x^2*e^(-abs(x)/2)};
    \end{axis}
    \end{tikzpicture}
\end{center}

\subsection*{Question 3}
\noindent\textbf{Problem:} What is the relationship between the probability distribution of $X_{s}$ and $X_{a}$?
\bigskip

\noindent\textbf{Solution:} These two random variables share the same distribution, that is:
$$X_{s}\sim X_{a}$$

To see this, let us denote the pdfs of $X_s$ and $X_a$ by $p_s$ and $p_a$ respectively. We now consider two cases:

\begin{itemize}
    \item Case 1, $x\ge0$:
    \begin{align*}
        p_a(x)&=|\psi_a(x)|^2\tag{pdf of a wavefunction}\\
        &=|\psi_s(x)|^2\tag{def. of $\psi_a(x)$ for $x\ge0$}\\
        &=p_s(x)\tag{pdf of a wavefunction}
    \end{align*}
    \item Case 2, $x<0$:
    \begin{align*}
        p_a(x)&=|\psi_a(x)|^2\tag{pdf of a wavefunction}\\
        &=|-\psi_s(x)|^2\tag{def. of $\psi_a(x)$ for $x<0$}\\
        &=|\psi_s(x)|^2\tag{modulus is invariant to sign}\\
        &=p_s(x)\tag{pdf of a wavefunction}
    \end{align*}
\end{itemize}

And so we have shown that $\forall x\in\mathbb R,\,\, p_a(x)=p_s(x)$. And since the pdfs of $X_a$ and $X_s$ are equivalent, they must share the same distribution.
\newpage

\subsection*{Question 4}
\noindent\textbf{Problem:} What is the probability that $X_s<0$?
\bigskip

\noindent\textbf{Solution:} First note that $\psi_s(x)$ is an even function:
\begin{align*}
    \psi_s(-x)&=\left|\frac{1}{4\sqrt{3}}(-x)^2e^{-|-x|/2}\right|^2\tag{def. of $\psi_s$}\\
    &=\left|\frac{1}{4\sqrt{3}}x^2e^{-|-x|/2}\right|^2\tag{$x^2=(-x)^2$}\\
    &=\left|\frac{1}{4\sqrt{3}}x^2e^{-|x|/2}\right|^2\tag{$|x|=|-x|$}\\
    &=\psi_s(x)\tag{def. of $\psi_s$}
\end{align*}

Now consider the following chain of equalities
\begin{align*}
    1&=\int_{-\infty}^\infty|\psi_s(x)|^2\,dx\tag{integral of a pdf over support is 1}\\
    &=\int_{-\infty}^0|\psi_s(x)|^2\,dx+\int_0^\infty|\psi_s(x)|^2\,dx\\
    &=\int_{-\infty}^0|\psi_s(x)|^2\,dx-\int_0^{-\infty}|\psi_s(-x)|^2\,dx\tag{expansion of interval of integration by $-1$}\\
    &=\int_{-\infty}^0|\psi_s(x)|^2\,dx+\int_{-\infty}^0|\psi_s(-x)|^2\,dx\tag{reverse interval of integration}\\
    &=\int_{-\infty}^0|\psi_s(x)|^2\,dx+\int_{-\infty}^0|\psi_s(x)|^2\,dx\tag{$\psi_s$ is an even function}\\
    &=2\int_{-\infty}^0|\psi_s(x)|^2\,dx\\
    \frac{1}{2}&=\int_{-\infty}^0|\psi_s(x)|^2\,dx\\
    &=P(X_s<0)
\end{align*}

And so, due to the symmetry of $\psi_s(x)$ about the y-axis, the desired probability is $\frac{1}{2}$.

\subsection*{Question 5}
\noindent\textbf{Problem:} For $\psi_{s+a}=\frac{\psi_s+\psi_a}{\sqrt{2}}$, what is the probability that $X_{s+a}<0$?
\bigskip

\noindent\textbf{Solution:} The desired probability is given by:
\begin{align*}
    P(X_{s+a}<0)&=\int_{-\infty}^0|\psi_{s+a}(x)|^2\,dx\\
    &=\int_{-\infty}^0\left|\frac{\psi_s+\psi_a}{\sqrt{2}}\right|^2\,dx\tag{def. of $\psi_{s+a}$}\\
    &=\int_{-\infty}^0\left|\frac{\psi_s-\psi_s}{\sqrt{2}}\right|^2\,dx\tag{$\forall x<0,\, \psi_a(x)=-\psi_s(x)$}\\
    &=\int_{-\infty}^0|0|^2\,dx\\
    &=0
\end{align*}

And so, due to the two wave functions destructively interfering with each other, the desired probability is 0.

\end{document}
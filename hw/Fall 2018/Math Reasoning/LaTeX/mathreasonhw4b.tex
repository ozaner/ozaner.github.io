\documentclass{article}
\usepackage{amsmath}
\usepackage{amssymb}

\begin{document}

\title{Intro to Math Reasoning HW 4b}
\author{Ozaner Hansha}
\date{October 3, 2018}
\maketitle

The functions $f$ and $g$ referenced in Problems 1-3 have the following domain and codomain: $f,g:\mathbb R\to \mathbb R$.

\section*{Problem 1}
\subsection*{Part a}
\textbf{Problem:} Are there two functions $f,g$ such that they both have limits as $x\to 0$?
\\\\
\textbf{Solution:} Yes, consider $f(x)=g(x)=x$. It clearly has a limit as $x\to 0$, namely 0.
\subsection*{Part b}
\textbf{Problem:} Is there a unique pair of functions $(f,g)$ such that they hold the same property in Part A?
\\\\
\textbf{Solution:} No, there is more than one pair of functions that satisfy this property. We gave one above, here is another one: $f(x)=x^2$ and $g(x)=5x+1$. They both have limits as $x\to 0$, with $\lim_{x\to 0}f(x)=0$ and $\lim_{x\to 0}g(x)=1$.
\subsection*{Part c}
\textbf{Problem:} Is there a pair of functions $(f,g)$ such that they do \textit{not} satisfy the above property?
\\\\
\textbf{Solution:} Yes, consider the following choice of $f$ and $g$:

$$f(x)=g(x)=\begin{cases}
    x+1, &\text{if } x\ge 0\\
    0, & \text{if }x<0
\end{cases}$$

This function has a right-sided limit of 1 as $x\to 0$ but a left-sided limit of 0. Thus the limit does not exist for either $f$ or $g$.

\section*{Problem 2}
\textbf{Problem:} Is the following statement true?: if $f$ has a limit as $x\to 0$ and $g$ is bounded, then the product $fg$ has a limit as $x\to 0$.
\\\\
\textbf{Solution:} This statement is false. Consider the following choices of functions:
\begin{align*}
  f(x)&=1\\
  g(x)&=\begin{cases}
    2, &\text{if } x\ge 0\\
    -2, & \text{if }x<0
  \end{cases}
\end{align*}

These functions satisfy the requirements for $f$ and $g$, namely:
\begin{align*}
  \lim_{x\to0}f(x)=1\tag{limit exists}\\
  \forall x\ |g(x)|\le 2\tag{$g$ is bounded}
\end{align*}

Also note that $\lim_{x\to0}g(x)$ does not exist (the left hand and right hand limits are $-2$ and $2$ respectively and thus do not line up). The reason this is important is because when we multiply the functions we get $f(x)g(x)=g(x)$. This is because we set $f(x)=1$. As a result $fg$ doesn't have a limit as $x\to0$ just like $g$. And so the statement we set out to disprove is indeed false.

\section*{Problem 3}
\textbf{Problem:} Prove the following statement: if $\lim_{x\to 0}f(x)=0$ and $g$ is bounded, then the product $fg$ has a limit as $x\to 0$.
\\\\
\textbf{Solution:} This is true and we can see this by writing down the definition of limit as $x\to 0$ for $f$:

\begin{align*}
\lim_{x\to 0}f(x)=0&\equiv(\forall\epsilon>0)(\exists\delta>0)(\forall x\in\mathbb R)\ 0<|x-0|<\delta\rightarrow|f(x)-0|<\epsilon\\
&\equiv(\forall\epsilon>0)(\exists\delta>0)(\forall x\in\mathbb R)\ 0<|x|<\delta\rightarrow|f(x)|<\epsilon
\end{align*}

and for the product of the functions $fg$:
\begin{align*}
\lim_{x\to 0}f(x)g(x)=0&\equiv(\forall\epsilon>0)(\exists\delta>0)(\forall x\in\mathbb R)\ 0<|x-0|<\delta\rightarrow|f(x)g(x)-0|<\epsilon\\
&\equiv(\forall\epsilon>0)(\exists\delta>0)(\forall x\in\mathbb R)\ 0<|x|<\delta\rightarrow|f(x)g(x)|<\epsilon
\end{align*}

Putting these together, this means we must prove the following:

\begin{gather*}
  (\forall\epsilon>0)(\exists\delta>0)(\forall x\in\mathbb R)\ 0<|x|<\delta\rightarrow|f(x)|<\epsilon\\
  \implies\\
  (\forall\epsilon>0)(\exists\delta>0)(\forall x\in\mathbb R)\ 0<|x|<\delta\rightarrow|f(x)g(x)|<\epsilon
\end{gather*}

First let's call the bound on $g$ the constant $M$. As in $|g(x)|\le M$. Now let us note that $|f(x)g(x)|\le|f(x)||g(x)|$ via the triangle inequality. Finally let express the $\epsilon$ in the first statement as $\frac{\epsilon}{M}$. Making the antecedent:

$$(\forall\frac{\epsilon}{M}>0)(\exists\delta>0)(\forall x\in\mathbb R)\ 0<|x|<\delta\rightarrow|f(x)|<\frac{\epsilon}{M}$$

And so for a given choice of $\delta$ we find that $|f(x)|<\frac{\epsilon}{M}$ which means that $|f(x)g(x)|\le|f(x)||g(x)|< M\cdot\frac{\epsilon}{M}=\epsilon$


Thus, $|f(x)g(x)|<\epsilon$ making the consequent of the statement true. This chains back up the statement with each implication's consequent being true until we have proved the whole statement.

\section*{Problem 4}
\textbf{Problem:} Prove the following:
$$(A \wedge B)\vee C\equiv(A\vee C)\wedge(B\vee C)$$
\textbf{Solution:} Here is a truth table:
\begin{center}
\begin{tabular}{ccccccc}
$A$ & $B$ & $C$ & $A\vee C$ & $B\vee C$ & $(A\vee C)\wedge(B\vee C)$ & $(A \wedge B)\vee C$\\
\midrule
\hline
F&F&F&F&F&F&F\\
F&F&T&T&T&T&T\\
F&T&F&F&T&F&F\\
F&T&T&T&T&T&T\\
T&F&F&T&F&F&F\\
T&F&T&T&T&T&T\\
T&T&F&T&T&T&T\\
T&T&T&T&T&T&T\\
\end{tabular}
\end{center}

\section*{Problem 5}
\textbf{Problem:} Prove the following:
$$\neg(A\rightarrow B)\equiv(A \wedge\neg B)$$
\textbf{Solution:} Here is a truth table:
\begin{center}
\begin{tabular}{cccccc}
$A$ & $B$ & $\neg B$ & $A\rightarrow B$ & $\neg(A\rightarrow B)$ & $A \wedge\neg B$\\
\midrule
\hline
F&F&T&T&F&F\\
F&T&F&T&F&F\\
T&F&T&F&T&T\\
T&T&F&T&F&F\\
\end{tabular}
\end{center}

\end{document}

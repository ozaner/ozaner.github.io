\documentclass{article}
\usepackage{amsmath}
\usepackage{amssymb}

\begin{document}

\title{Intro to Math Reasoning HW 9a}
\author{Ozaner Hansha}
\date{November 14, 2018}
\maketitle

\section*{Problem 1}
\textbf{Problem:} Prove that the distributive property holds for finitary addition.
\\\\
\textbf{Solution:} First we'll define distributivity for an $n$-ary sum via the following predicate for any $y$ and summands $x_i$:
$$P(n)\equiv y\left(\sum^n_{i=1}x_i\right)=\sum^n_{i=1}x_iy$$

Now note that $P(1)$ is true:
\begin{align*}
  y\left(\sum^n_{i=1}x_i\right)&=\sum^n_{i=1}x_iy\tag*{$P(1)$}\\
  y(x_1)&=yx_1\tag{def. of $1$-ary sum}
\end{align*}

Now we will prove that $P(n)\rightarrow P(n+1)$:
\begin{align*}
  y\sum^n_{i=1}x_i&=\sum^n_{i=1}yx_i\tag{$P(n)$ given}\\
  \left(y\sum^n_{i=1}x_i\right)+yx_{n+1}&=\left(\sum^n_{i=1}yx_i\right)+yx_{n+1}\\
  y\left(\left(\sum^n_{i=1}x_i\right)+x_{n+1}\right)&=\left(\sum^n_{i=1}yx_i\right)+yx_{n+1}\tag{distrib. prop. of binary $+$}\\
  y\left(\sum^{n+1}_{i=1}x_i\right)&=\left(\sum^{n+1}_{i=1}yx_i\right)\tag{def. of $(n+1)$-ary sum}
\end{align*}

And so we have shown that $P(n+1)$ follows from $P(n)$. The PMI tells us that:
\begin{align*}
&P(n)\implies P(n+1)\\
&P(1)\\
\therefore\ &{\forall n\in \mathbb{Z}^+,\ P(n)}
\end{align*}

And so $n$-ary addition is distributive for finite positive $n$.

\section*{Problem 2}
\textbf{Problem:} Prove that the associative property holds for finitary multiplication.
\\\\
\textbf{Solution:} First we'll define distributivity for an $n$-ary sum via the following predicate given $x_i$:
$$P(n,k)\equiv(1<k<n)\wedge\left(\prod^n_{i=1}x_i\right)=\left(\prod^k_{i=1}x_i\right)\left(\prod^n_{i=k+1}x_i\right)$$

Now note that $(\forall k,\ 1<k<n)\ P(3,k)$ is true, we just need to check $k=2$:
\begin{align*}
  \left(\prod^3_{i=1}x_i\right)&=\left(\prod^2_{i=1}x_i\right)\left(\prod^{3}_{i=2+1}x_i\right)\tag*{$P(3,2)$}\\
  \left(\prod^2_{i=1}x_i\right)x_3&=\left(\prod^2_{i=1}x_i\right)\left(\prod^3_{i=3}x_i\right)\tag{def. of finitary multiplication}\\
  \left(\prod^2_{i=1}x_i\right)x_3&=\left(\prod^2_{i=1}x_i\right) x_3\tag{def. of $1$-ary multiplication}
\end{align*}

\textit{Also note that $(\forall k,\ 1<k<n)\ P(1,k)\wedge P(2,k)$ is vacuously true.}\\

Now we will prove that $(\forall k,\ 1<k<n)\ P(n,k)\rightarrow P(n+1,k)$:
\begin{align*}
  \left(\prod^n_{i=1}x_i\right)&=\left(\prod^k_{i=1}x_i\right)\left(\prod^n_{i=k+1}x_i\right)\tag{given}\\
  \left(\prod^n_{i=1}x_i\right)x_{n+1}&=\left(\prod^k_{i=1}x_i\right)\left(\prod^n_{i=k+1}x_i\right)x_{n+1}\\
  \left(\prod^{n+1}_{i=1}x_i\right)&=\left(\prod^k_{i=1}x_i\right)\left(\prod^{n+1}_{i=k+1}x_i\right)\tag{def. of finitary multiplication}
\end{align*}

But this only proves $k$ bounded by the previous $n$ not $n+1$. To resolve this we now have to prove that $(P(n,k)\wedge(1<k+1<n))\rightarrow P(n,k+1)$:

\begin{align*}
  \left(\prod^n_{i=1}x_i\right)&=\left(\prod^k_{i=1}x_i\right)\left(\prod^n_{i=k+1}x_i\right)\tag{given}\\
  &=\left(\prod^k_{i=1}x_i\right)\left(\prod^n_{i=k+2}x_i\right)x_{k+1}\tag{def. of finitary multiplication}\\
  &=\left(\prod^k_{i=1}x_i\right)x_{k+1}\left(\prod^n_{i=k+2}x_i\right)\tag{associativity of $3$-ary mult.}\\
  &=\left(\prod^{k+1}_{i=1}x_i\right)\left(\prod^n_{i=k+2}x_i\right)\tag{def. of finitary multiplication}
\end{align*}

\textit{Notice that line 2 required that $1<k+1<n$ in order to use that definition of finitary multiplication.}\\

And so by induction we have:
\begin{align*}
&(\forall k,\ 1<k<n)\ P(3,k)\\
&(\forall k,\ 1<k<n)\ P(n,k)\rightarrow P(n+1,k)\\
&(\forall n\in \mathbb{Z}^+)(\forall k,\ 1<k<3)\ P(n,k)\tag{special case for 1 \& 2}\\
\therefore\ &(\forall n\in \mathbb{Z}^+)\ P(n,2)\tag{only $k=2$ satisfies this}
\end{align*}

Using induction once more we find:
\begin{align*}
&(\forall n\in \mathbb{Z}^+)\ P(n,2)\\
&(P(n,k)\wedge(1<k+1<n))\rightarrow P(n,k+1)\\
\therefore\ &{(\forall n,k\in \mathbb{Z}^+)\ (\forall k,\ 1<k<n)\ P(n,k)}
\end{align*}

\section*{Problem 3}
\textbf{Problem:} Prove the following for all $n\in\mathbb Z^+$:
$$\sum_{i=1}^ni^3=\left(\sum_{i=1}^ni\right)^2$$
\textbf{Solution:} First we define the following predicate:
$$P(n)\equiv \sum_{i=1}^ni^3=\left(\sum_{i=1}^ni\right)^2$$

Now note that $P(1)$ is true:
\begin{align*}
  \sum_{i=1}^1i^3&=\left(\sum_{i=1}^1i\right)^2\tag*{$P(1)$}\\
  1^3&=(1)^2\tag{def. of $1$-ary sum}
\end{align*}

Now we will prove that $P(n)\rightarrow P(n+1)$:
\begin{align*}
  \sum_{i=1}^ni^3&=\left(\sum_{i=1}^ni\right)^2\tag{$P(n)$ given}\\
  \left(\sum_{i=1}^ni^3\right)+(n+1)^3&=\left(\sum_{i=1}^ni\right)^2+(n+1)^3\\
  \left(\sum_{i=1}^{n+1}i^3\right)&=\left(\sum_{i=1}^ni\right)^2+(n+1)^3\tag{def. of finitary sum}\\
  &=\left(\frac{n(n+1)}{2}\right)^2+(n+1)^3\tag{$n$th triangular number}\\
  &=\frac{n^4+6n^3+13n^2+12n+4}{4}\tag{algebra}\\
  &=\left(\frac{(n+1)(n+2)}{2}\right)^2\tag{Lemma 1}\\
  \left(\sum_{i=1}^{n+1}i^3\right)&=\left(\sum_{i=1}^{n+1}i\right)^2\tag{$(n+1)$th triangular number}\\
\end{align*}

Notice that we justified the 5th step as `Lemma 1'. This is because factoring is hard and so we will simply show that this is true in reverse, by multiplying out the square of the $(n+1)th$ triangular number:
\begin{align*}
  \left(\frac{(n+1)(n+2)}{2}\right)^2&=\left(\frac{n^2+3n+2}{2}\right)^2\\
  &=\frac{(n^2+3n+2)^2}{4}\\
  &=\frac{n^4+6n^3+13n^2+12n+4}{4}
\end{align*}

And so we have shown that $P(n+1)$ follows from $P(n)$. The PMI tells us that:
\begin{align*}
&P(n)\implies P(n+1)\\
&P(1)\\
\therefore\ &{\forall n\in \mathbb{Z}^+,\ P(n)}
\end{align*}

And so the equality holds for all positive $n$.

\section*{Problem 4}
\textbf{Problem:} Given some set of $k$ real (or complex) constants $a_i$, we define an LHCC recurrence relation $R$ of order $k$ as:
$$x(n)=\sum_{i=1}^ka_ix(n-i)=a_1x(n-1)+\cdots+a_kx(n-k)$$

Show that if two sequences satisfy $R$, then a linear combination of the two also satisfy it. Then use induction to show that a linear combination of $m$ solutions to $R$ also satisfies $R$.
\\\\
\textbf{Solution}
\\\\
\textbf{$2$ Sequence Case}
\begin{align*}
  y_1(n)&=a_1y_1(n-1)+\cdots+a_ky_1(n-k)\tag{given}\\
  &=\sum_{i=1}^ka_iy_1(i)\\
  y_2(n)&=a_1y_2(n-1)+\cdots+a_ky_2(n-k)\tag{given}\\
  &=\sum_{i=1}^ka_iy_2(n)
\end{align*}

Now we can simply multiply the equations by any real (or complex) constants $b_1$ and $b_2$ then sum the resulting equations to show the $2$ sequence case does indeed satisfy $R$:
\begin{align*}
  b_1y_1(n)&=b_1\sum_{i=1}^ka_iy_1(n-i)\\
  b_2y_2(n)&=b_2\sum_{i=1}^ka_iy_2(n-i)\\
  b_1y_1(n)+b_2y_2(n)&=b_1\sum_{i=1}^ka_iy_1(n-i)+b_2\sum_{i=1}^{k}a_iy_2(n-i)\\
  &=\sum_{i=1}^ka_i(b_1y_1(n-i))+\sum_{i=1}^ka_i(b_2y_2(n-i))\tag{finitary distrib. prop.}\\
  &=\sum_{i=1}^ka_i(b_1y_1(n-i)+b_2y_2(n-i))\tag{finitary distrib. prop.}\\
  &=a_1(b_1y_1(n-1)+b_2y_2(n-1))+\cdots+a_k(b_1y_1(n-k)+b_2y_2(n-k))
\end{align*}
\\\\\\
\textbf{$m$ Sequence Case}

Let $y_j(i)$ be any list of $m$ solutions to $R$ and let $b_j$ be any list of $m$ real (or complex) constants. We'll use this to define a predicate:
\begin{align*}
P(m)\equiv\sum_{j=1}^mb_jy_j(n)&=\sum_{i=1}^k\sum_{j=1}^ma_ib_jy_j(n-i)\\
&=\sum_{j=1}^m\sum_{i=1}^ka_ib_jy_j(n-i)
\end{align*}

We proved $P(2)$ above for the arbitrary solutions and constants $y_1, y_2, b_1,b_2$. Now we just need to prove that $P(m)\rightarrow P(m+1)$ given some new solution $y_{m+1}(i)$ and constant $b_{m+1}$:
\begin{align*}
  \sum_{j=1}^mb_jy_j(n)&=\sum_{j=1}^m\sum_{i=1}^ka_ib_jy_j(n-i)\tag{given}\\
  \sum_{j=1}^mb_jy_j(n)+b_{m+1}y_{m+1}(n)&=\left(\sum_{j=1}^m\sum_{i=1}^ka_ib_jy_j(n-i)\right)+b_{m+1}y_{m+1}(n)\\
  \sum_{j=1}^{m+1}b_jy_j(n)&=\left(\sum_{j=1}^m\sum_{i=1}^ka_ib_jy_j(n-i)\right)+b_{m+1}y_{m+1}(n)\tag{def. finitary addition}\\
  &=\left(\sum_{j=1}^m\sum_{i=1}^kb_jy_j(n-i)\right)+\sum_{i=1}^ka_ib_{m+1}y_{m+1}(n-i)\tag{def. of solution}\\
  &=\sum_{j=1}^{m+1}\sum_{i=1}^kb_jy_j(n-i)\tag{def. finitary addition}
\end{align*}

And so we have shown that $P(m+1)$ follows from $P(m)$. The PMI tells us that:
\begin{align*}
  &P(2)\\
  &P(m)\implies P(m+1)\\
  \therefore\ &{\forall m\le2,\ P(m)}
\end{align*}

\section*{Problem 5}
\textbf{Problem:} Derive the explicit formula for the Fibonacci sequence:
$$f_1=f_2=1;\ f_n=f_{n-1}+f_{n-2}$$

Then evaluate it at $n=3,4,5$.
\\\\
\textbf{Solution:} First we solve for the roots of the characteristic polynomial:
\begin{align*}
  x^2-x-1=0\\
  r=\frac{1\pm\sqrt{5}}{2}
\end{align*}

Denoting the roots $r_1$ and $r_2$, we know that there exists constants $b_1$ and $b_2$ such that:
$$f_n=b_1r_1^n+b_2r_2^n$$

And since $f_1=f_2=1$ we can solve for the constants:
\begin{gather*}
  f_1=b_1r_1+b_2r_2=1\\
  f_2=b_1r_1^2+b_2r_2^2=1\\
  b_2=\frac{1-b_1r_1}{r_2}\\
  b_1r_1^2+r_2(1-b_1r_1)=1
\end{gather*}

Plugging in the roots and solving for $b_1$ we find:
$$b_1=\frac{1}{\sqrt 5}$$

Plugging this back into the first equation, we can now solve for $b_2$
$$b_2=-\frac{1}{\sqrt 5}$$

This leaves us with the following explicit formula for the Fibonacci sequence:
$$f_n=\frac{1}{\sqrt 5}\left(\left(\frac{1+\sqrt 5}{2}\right)^n-\left(\frac{1+\sqrt 5}{2}\right)^n\right)$$

\end{document}

\documentclass{article}
\usepackage{amsmath}
\usepackage{amssymb}

\begin{document}

\title{Honors Calculus III HW \#2}
\author{Ozaner Hansha}
\date{September 24, 2018}
\maketitle

\section{Problem 1}
\textbf{Problem:} Find a right-handed orthonormal basis $\{\mathbf{u}_1,\mathbf{u}_2,\mathbf{u}_3\}$ such that:
\begin{gather}
\left(\exists k\in\mathbb{R}^+\right)\mathbf{u}_1=k(4,4,7)\\
\mathbf{u}_2\cdot(1,0,2)=0
\end{gather}
\textbf{Solution:} To find $\mathbf{u}_1$ we must normalize the $(4,4,7)$. This leaves us with:
\begin{gather*}
\|(4,4,7)\|=\sqrt{4^2+4^2+7^2}=9\\
\mathbf{u}_1=\frac{1}{9}(4,4,7)
\end{gather*}

To find $\mathbf{u}_2$, recall that it must be orthogonal to $(1,0,2)$ (by condition 2) and $\mathbf{u}_1$ (because it's an orthonormal basis). This means that if we take the normalized cross product of these two vectors, we will have a vector that matches both criteria:
\begin{align*}
\mathbf{u}_2&=\frac{(1,0,2)\times\frac{1}{9}(4,4,7)}{\|(1,0,2)\times\frac{1}{9}(4,4,7)\|}\\
&=\frac{\frac{1}{9}(-8,1,4)}{\left\|\frac{1}{9}(-8,1,4)\right\|}=\frac{\frac{1}{9}(-8,1,4)}{\left|\frac{1}{9}\right|\left\|(-8,1,4)\right\|}\\
&=\frac{\frac{1}{9}(-8,1,4)}{\left|\frac{1}{9}\right|\cdot 9}\\
&=\frac{1}{9}(-8,1,4)
\end{align*}

Now we simply take the cross product of $\mathbf{u}_1$ and $\mathbf{u}_2$, both of which we know to be orthonormal, to find a third orthonormal vector:

\begin{align*}
\mathbf{u}_3&=\mathbf{u}_1\times\mathbf{u}_2\\
&=\frac{1}{9}(4,4,7)\times\frac{1}{9}(-8,1,4)\\
&=\frac{1}{9^2}(9,-72,36)\\
&=\frac{1}{9}(1,-8,4)
\end{align*}

\textit{Note that the order of the cross product was chosen in accordance to the right-handedness condition.}

$$\left\{\frac{1}{9}(4,4,7),\frac{1}{9}(-8,1,4),\frac{1}{9}(1,-8,4)\right\}$$

However, note that when we found the cross product of $(1,0,2)\times\frac{1}{9}(4,4,7)$ we could have switched their order to arrive at a $\mathbf{u}_2$ with a flipped sign that still satisfied the required conditions. And to retain righthandedness, $\mathbf{u}_3$ would also be flipped (in accordance to the anticommutivity of the cross product). This gives us another equally valid basis:

$$\left\{\frac{1}{9}(4,4,7),\frac{-1}{9}(-8,1,4),\frac{-1}{9}(1,-8,4)\right\}$$

And so, there are two valid bases that satisfy the given conditions.

\section{Problem 2}
\textbf{Problem:} Given two lines $\ell_1$ and $\ell_2$ parametrized below:
\begin{gather*}
\mathbf{x}:\mathbb{R}\to\ell_1\\
\mathbf{x}(t)=(1,2,2)+t(0,3,3)\\
\mathbf{y}:\mathbb{R}\to\ell_2\\
\mathbf{y}(s)=s(2,1,2)
\end{gather*}

What is the distance between these lines? Also, what values of $t$ and $s$ minimize $\|\mathbf{x}(t)-\mathbf{y}(s)\|$?
\\\\
\textbf{Solution:} Notice that $\ell_1$ is in the direction of $(0,3,3)$ and that $\ell_2$ is in the direction of $(2,1,2)$. The shortest path from a line to another line is perpendicular to both of them. As such, we will take the cross product of these vectors (and normalize it to get a vector only describing the direction from $\ell_1$ to $\ell_2$):
\begin{align*}
  \mathbf u&=\frac{(2,1,2)\times(0,3,3)}{\|(2,1,2)\times(0,3,3)\|}\\
  &=\frac{(-3,-6,6)}{\|(-3,-6,6)\|}=\frac{(-3,-6,6)}{9}\\
  &=\frac{1}{3}(-1,-2,2)
\end{align*}

Now we simply take any point on $\ell_1$ and any point on $\ell_2$, produce the vector that joins them $\mathbf w$, and project that vector onto $\mathbf u$. The magnitude of this vector is the distance we are looking for:

\begin{gather*}
  \mathbf x(0)=(1,2,2)\\
  \mathbf y(0)=(0,0,0)\\
  \mathbf w=\mathbf x(0)-\mathbf y(0)=(1,2,2)
\end{gather*}

While we could project $\mathbf w$ onto $\mathbf u$ and take the magnitude of the result, notice that because $\mathbf u$ is already a unit vector it suffices to simply take the dot product of the two vectors ($\mathbf w$'s component in the $\mathbf u$ direction) and take its absolute value:
\begin{align*}
  |\mathbf w\cdot\mathbf u|=\left|(1,2,2)\cdot\frac{1}{3}(-1,-2,2)\right|=\frac{1}{3}
\end{align*}

So we are done (with the first part) and the distance is $\frac{1}{3}$. To find what $t$ and $s$ actually minimize this we must minimize the following:

$$\|\mathbf x(t)-\mathbf y(s)\|=\sqrt{(1-2s)^2+(2+3t-s)^2+(2+3t-2s)^2}$$

Now we just have to take the partial derivative of the above function with respect to both $t$ and $s$ and set them equal to zero. Also, note that minimizing the norm is the same as minimizing the norm squared, so we will do just that to make the calculus a little easier:

\begin{gather*}
  \frac{}{\partial t}\partial \|\mathbf x(t)-\mathbf y(s)\|^2=36t=18s+24=0\\
  \frac{}{\partial s}\partial \|\mathbf x(t)-\mathbf y(s)\|^2=18s-18t-16=0\\
\end{gather*}

We can now solve for $t$ and $s$ via the following system of equations:

\begin{gather*}
  36t=18s+24=0\\
  18s-18t-16=0\\
  t=\frac{-4}{9} \wedge s=\frac{4}{9}
\end{gather*}

\section{Problem 3}
\textbf{Problem:} Let the line $\ell_1$ pass through $(1,2,2)$ and $(1,5,5)$, and let the line $\ell_2$ be given by $\mathbf x_0+(2,1,2)$. The set of all points $\mathbf x_0$ where these lines meet (i.e $\ell_1\cap\ell_2\not=\emptyset$) forms a plane. Give this plane in the form $ax+by+cz=d$.
\\\\
\textbf{Solution:} Call a point that $\ell_1$ and $\ell_2$ intersect in $\mathbf p$. We can now define the plane the lines sit on as the set of all $\mathbf x$ such that $\mathbf a\cdot(\mathbf p-\mathbf p_0)=0$ where $\mathbf a\not=\mathbf 0$. Because $\ell_2$ is in this plane, so too is $\mathbf x_0$.

Now we'll parameterize the plane using that arbitrary point on the plane $\mathbf p$ (and the fact that $(1,5,5)-(1,2,2)=(0,3,3)$ for $\ell_1$):

\begin{gather*}
  \mathbf r(t)=\mathbf p+t(0,3,3)\\
  \mathbf q(s)=\mathbf p+s(2,1,2)
\end{gather*}

Plugging in $\mathbf a(t)$ and $\mathbf b(s)$ in for $\mathbf x$ into $\mathbf a\cdot(\mathbf x-\mathbf p)=0$ (which they must satisfy given that they parameterize the plane) we find the following:

\begin{gather*}
  \mathbf a\cdot (0,3,3)=0\\
  \mathbf a\cdot (2,1,2)=0
\end{gather*}

This means $\mathbf a$ is some scalar multiple of $(0,3,3)\times(2,1,2)=(3,6,-6)$. This is equivalent to saying it is a scalar multiple of $(-1,-2,2)$ (i.e multiply by $\frac{-1}{3}$) and so by taking another point on the plane, say $(1,2,2)$ (which is on line $\ell_1$) we can compute the standard form of a plane $ax+by+cz=d$ with $(a,b,c)=(-1,-2,2)$ and $d=(-1,-2,2)\cdot(1,2,2)=-1$:

$$-x-2y+2z=-1$$

\section{Problem 4}
The householder reflection $h_\mathbf u$ given by $\mathbf u$ is defined as $h_\mathbf u(\mathbf x)=\mathbf x-2(\mathbf x\cdot\mathbf u)u$. Also note that $\{\mathbf e_1,\mathbf e_2,\mathbf e_3\}$ represents the canonical basis of $\mathbb R^3$
\subsection{Part a}
\textbf{Problem:} Find a $\mathbf u$ such that $h_\mathbf u(\mathbf e_1)=\frac{1}{9}(4,4,7)$
\\\\
\textbf{Solution:} We have to find a $\mathbf u$ such that the following is true:

\begin{align*}
  \mathbf u&=\frac{\mathbf e_1-\frac{1}{9}(4,4,7)}{\left\|\mathbf e_1-\frac{1}{9}(4,4,7)\right\|}\\
  &=\frac{\frac{1}{9}(5,-4,-7)}{\left\|\frac{1}{9}(5,-4,-7)\right\|}\\
  &=\frac{\frac{1}{9}(5,-4,-7)}{\left\|\frac{1}{9}(5,-4,-7)\right\|}\\
  &=\frac{\frac{1}{9}(5,-4,-7)}{\frac{10}{\sqrt 3}}\\
  &=\frac{1}{3\sqrt{10}}(5,-4,-7)
\end{align*}

\subsection{Part b}
\textbf{Problem:} Using the $\mathbf u$ found in part a, compute $h_\mathbf u(\mathbf e_2)$ and $h_\mathbf u(\mathbf e_3)$. Also show that $\{h_\mathbf u(\mathbf e_1),h_\mathbf u(\mathbf e_2),h_\mathbf u(\mathbf e_3)\}$ is a left-handed orthonormal basis of $\mathbb R^3$.
\\\\
\textbf{Solution:} By doing the computations with $\mathbf u=\frac{1}{3\sqrt{10}}(5,-4,-7)$ we find:

\begin{gather*}
  h_\mathbf u(\mathbf e_2)=\frac{1}{45}(20,29,-28)\\
  h_\mathbf u(\mathbf e_3)=\frac{1}{45}(35,-28,-4)
\end{gather*}

Of course $h_\mathbf u(\mathbf e_1)$, $h_\mathbf u(\mathbf e_2)$, and $h_\mathbf u(\mathbf e_3)$ are orthonormal because both length and orthogonality are preserved by the householder transformation. To check if they form a lefthanded basis the following must be true:

$$h_\mathbf u(\mathbf e_1)\times h_\mathbf u(\mathbf e_2)=-h_\mathbf u(\mathbf e_3)$$

And indeed when we do the calculations we find:

$$\frac{1}{9}(4,4,7)\times\frac{1}{45}(20,29,-28)=\frac{-1}{45}(35,-28,-4)$$

\section{Problem 5}
Let $\mathbf v_1$, $\mathbf v_2$, and $\mathbf v_3$ be any vectors in $\mathbb R^3$ such that $\mathbf v_1\cdot(\mathbf v_2\times\mathbf v_3)\not=0$

\subsection{Part a}
\textbf{Problem:} Prove that $|\mathbf v_1\cdot(\mathbf v_2\times\mathbf v_3)|=|\mathbf v_2\cdot(\mathbf v_3\times\mathbf v_1)|=|\mathbf v_3\cdot(\mathbf v_2\times\mathbf v_1)|$.
\\\\
\textbf{Solution:} Recall triple product identity, which states that any cyclic permutation of the vectors $a$, $b$, and $c$ in the form $a\cdot(b\times c)$ is equivalent to each other. So, simply call $\mathbf v_1$, $\mathbf v_2$, and $\mathbf v_3$ $a$, $b$, and $c$ respectively and then take the absolute value of the quantity inside (just a weaker statement than the triple product identity) and we're done.

\subsection{Part b}
\textbf{Problem:} Call the scalar triple product referenced in part a $D$. Define the 3 vectors:
\begin{gather*}
  \mathbf w_1=\frac{1}{D}\mathbf v_2\times\mathbf v_3\\
  \mathbf w_2=\frac{1}{D}\mathbf v_3\times\mathbf v_1\\
  \mathbf w_3=\frac{1}{D}\mathbf v_1\times\mathbf v_2
\end{gather*}

Show that for all $1\le i$ and $j\le 3$ the following is true: $\mathbf v_i\cdot\mathbf w_j=\delta_{ij}$
\\\\
\textbf{Solution:} Note that the triple product of any two vectors is $0$ if any two of them are equal. This is a result of being able to cyclically permutate the vectors until the two equivalent vectors are together in the cross product part of the triple product. And so:

\begin{gather*}
  \mathbf v_1\cdot\mathbf w_1=\left(\frac{1}{D}\right)\mathbf v_1\cdot\mathbf v_2\times\mathbf v_3=1\\
  \mathbf v_2\cdot\mathbf w_1=\left(\frac{1}{D}\right)\mathbf v_2\cdot\mathbf v_2\times\mathbf v_3=0\\
  \mathbf v_3\cdot\mathbf w_1=\left(\frac{1}{D}\right)\mathbf v_3\cdot\mathbf v_2\times\mathbf v_3=0\\
\end{gather*}

The first one being true because the triple product is $D$ and the second and third being true because of the property noted above. The same argument follows for $\mathbf w_2$ and $\mathbf w_3$.

\section{Problem 6}
Let $v_1,v_2,v_3,w_1,w_2$ and $w,3$ be the same ones from problem 5.
\subsection{Part a}
\textbf{Problem:} Show that $\operatorname{span}(\{v_1,v_2,v_3\})=\mathbb R^3$
\\\\
\textbf{Solution:} We know that $\operatorname{span}(\{v_1,v_2,v_3\})$ forms a subspace of $\mathbb R^3$ because it is closed under scalar multiplication and vector addition. Also note that any subspace of $\mathbb R^3$ must either be a line or plane through the origin, or all of $\mathbb R^3$ itself. So we just have to show that there exists no plane that can contain these 3 vectors.

The plane through the origin containing $\mathbf v_1$ and $\mathbf v_2$ is defined by the equation $x\cdot(\mathbf v_1\times\mathbf v_2)=0$. However, recall from problem 5 that $v_3$ does not satisfy that equation. And so, the three vectors are not contained in a plane (ruling out a line as well) thus they must span the entirety of $\mathbb R^3$.

\subsection{Part b}
\textbf{Problem:} Show that all vectors in $\mathbb R^3$ can be expressed as a unique linear combination of the vectors $t_1\mathbf v_1+t_2\mathbf v_2+t_3\mathbf v_3$ and that $t_j=\mathbf w_j\cdot\mathbf x$.
\\\\
\textbf{Solution:} Because, as we've shown in part a, these \textbf{three} vectors span all of real \textbf{three}-space, we can conclude that there exists a unique triplet of scalars that satisfy the following for all $\mathbf x\in\mathbb R^3$:

$$t_1\mathbf v_1+t_2\mathbf v_2+t_3\mathbf v_3=\mathbf x$$

So we've proved the first statement. Now taking the dot product of both sides with $\mathbf w_1$ we get:

$$\mathbf w_1\cdot\mathbf x=t_1(\mathbf w_1\cdot\mathbf v_1)+t_2(\mathbf w_1\cdot\mathbf v_2)+t_3(\mathbf w_1\cdot\mathbf v_3)=t_1$$

Remember from problem 5 that because the indices don't match (i.e. orthogonal), the dot product equals 0 for the last two terms. The same argument follows for $\mathbf w_2$ and $\mathbf w_3$.

\section{Problem 7}
Define the following three vectors as so:
\begin{gather*}
  \mathbf v_1=(1,0,1)\\
  \mathbf v_2=(1,1,1)\\
  \mathbf v_3=(1,2,3)
\end{gather*}

\subsection{Part a}
\textbf{Problem:} Find three vectors $\mathbf w_1$, $\mathbf w_2$, and $\mathbf w_3$ such that for all $1\le i$ and $j\le 3$ the following is true: $\mathbf v_i\cdot\mathbf w_j=\delta_{ij}$
\\\\
\textbf{Solution:} First we compute the following three vectors:
\begin{gather*}
  \mathbf v_2\times\mathbf v_3=(1,-2,1)\\
  \mathbf v_3\times\mathbf v_1=(2,2,-2)\\
  \mathbf v_1\times\mathbf v_2=(-1,0,1)
\end{gather*}

Now, using the formula from problem 6 part a, we divide them by $D=\mathbf v_1\cdot(\mathbf v_2\times\mathbf v_3)=2$, and arrive at:
\begin{gather*}
  \mathbf w_1=\mathbf v_2\times\frac{1}{2}\mathbf v_3=(1,-2,1)\\
  \mathbf w_2=\mathbf v_3\times\mathbf v_1=(1,1,-1)\\
  \mathbf w_3=\mathbf v_1\times\frac{1}{2}\mathbf v_2=(-1,0,1)
\end{gather*}

\subsection{Part b}
\textbf{Problem:} Find three numbers $t_1$, $t_2$, and $t_3$ such that:

$$t_1(1,0,1)+t_2(1,1,1)+t_3(1,2,3)=(12,-7,19)$$
\textbf{Solution:} Now, using the formula from problem 6 part b, we just solve for the constants:
\begin{gather*}
  t_1=\mathbf w_1\cdot(12,-7,19)=\frac{45}{2}\\
  t_2=\mathbf w_2\cdot(12,-7,19)=-14\\
  t_3=\mathbf w_3\cdot(12,-7,19)=\frac{7}{2}
\end{gather*}

\section{Problem 8}
\textbf{Problem:} Show that for any 3 vectors $\mathbf a$, $\mathbf b$, and $\mathbf c$ in $\mathbb R^3$ that:

$$(\mathbf b\times\mathbf c)\cdot[(\mathbf c\times\mathbf a)\times(\mathbf a\times\mathbf b)]=|\mathbf a\cdot(\mathbf b\times\mathbf c)|^2$$
\textbf{Solution:} Recall Lagrange's identity, that is for any $\mathbf x,\mathbf y,\mathbf z\in\mathbb R^3$:

$$\mathbf x\times(\mathbf y\times\mathbf z)=(\mathbf x\cdot\mathbf z)\mathbf y-(\mathbf x\cdot\mathbf y)\mathbf z$$

Now we define $\mathbf x=(\mathbf c\times\mathbf a)$, $\mathbf y=\mathbf a$, and $\mathbf z=\mathbf b$. Plugging these into the identity we find:

$$(\mathbf c\times\mathbf a)\times(\mathbf a\times\mathbf b)=((\mathbf c\times\mathbf a)\cdot\mathbf b)\mathbf a-((\mathbf c\times\mathbf a)\cdot\mathbf a)\mathbf b$$

Because of the fact that any if any 2 vectors in the triple product are equal the product is 0, we can cancel out the second term on the left-hand side to arrive at:

$$(\mathbf c\times\mathbf a)\times(\mathbf a\times\mathbf b)=((\mathbf c\times\mathbf a)\cdot\mathbf b)\mathbf a$$

Now we just take the dot product of both sides with $(\mathbf b\times\mathbf c)$:

\begin{align*}
  (\mathbf b\times\mathbf c)\cdot[(\mathbf c\times\mathbf a)\times(\mathbf a\times\mathbf b)]&=(\mathbf b\times\mathbf c)\cdot[((\mathbf c\times\mathbf a)\cdot\mathbf b)\mathbf a]\\
  &=((\mathbf c\times\mathbf a)\cdot\mathbf b)((\mathbf b\times\mathbf c)\cdot\mathbf a)\tag{distribute}\\
  &=((\mathbf a\times\mathbf b)\cdot\mathbf c)((\mathbf a\times\mathbf b)\cdot\mathbf c)\tag{cyclic permutate}\\
  &=|\mathbf a\cdot(\mathbf b\times\mathbf c)|^2
\end{align*}

And we are done.

\end{document}

\documentclass{article}
\usepackage{amsmath}
\usepackage{amssymb}

\begin{document}

\title{Honors Calculus III HW \#7}
\author{Ozaner Hansha}
\date{November 19, 2018}
\maketitle

\section*{Exercise 1}
Consider the following matrix:
$$A=\begin{bmatrix}
    3 & 5 \\
    2 & 3 \\
\end{bmatrix}$$
\subsection*{Part a}
\textbf{Problem:} Find $A^{-1}$.
\\\\
\textbf{Solution:} $\operatorname{det}(A)=3*3-5*2=-1$ and so:
$$A^{-1}=\begin{bmatrix}
    -3 & 5 \\
    2 & -3 \\
\end{bmatrix}$$

\subsection*{Part b}
\textbf{Problem:} Solve the following equations:
$$A\mathbf x=(3,2)\ \ \ \  A\mathbf x=(2,2)\ \ \ \  A\mathbf x=(-1,7)$$
\textbf{Solution:} Using the inverse we computed above the solutions are simply:
\begin{align*}
  \begin{bmatrix}
      -3 & 5 \\
      2 & -3 \\
  \end{bmatrix}(3,2)^\top&=(1,0)\\
  \begin{bmatrix}
      -3 & 5 \\
      2 & -3 \\
  \end{bmatrix}(2,2)^\top&=(4,-2)\\
  \begin{bmatrix}
      -3 & 5 \\
      2 & -3 \\
  \end{bmatrix}(-1,7)^\top&=(38,-23)\\
\end{align*}

\section*{Exercise 2}
Consider the following matrix:
$$A=\begin{bmatrix}
    3 & 5 & 1 \\
    2 & 3 & 1 \\
    1 & 1 & 2 \\
\end{bmatrix}$$
\subsection*{Part a}
\textbf{Problem:} Find $A^{-1}$.
\\\\
\textbf{Solution:} Consider the columns of $A=(\mathbf v_1,\mathbf v_2,\mathbf v_3)$, by computing the following cross products:
\begin{align*}
  \mathbf v_2\times\mathbf v_3=(5,-9,2)
  \mathbf v_1\times\mathbf v_3=(3,-5,1)\\
  \mathbf v_1\times\mathbf v_2=(-1,2,-1)\\
\end{align*}

We can arrange them row-wise into a matrix, then multiply it by the inverse of $\operatorname{det}(A)=-1$:
$$A^{-1}=\begin{bmatrix}
    -5 & 9 & -2 \\
    -3 & 5 & -1 \\
    1 & -2 & 1 \\
\end{bmatrix}$$

\subsection*{Part b}
\textbf{Problem:} Solve the following equations:
$$A\mathbf x=(3,2,1)\ \ \ \  A\mathbf x=(2,2,1)\ \ \ \  A\mathbf x=(-1,7,1)$$
\textbf{Solution:} Now we just matrix multiply by the inverse to solve for $\mathbf x$:
\begin{align*}
  \begin{bmatrix}
      -5 & 9 & -2 \\
      -3 & 5 & -1 \\
      1 & -2 & 1 \\
  \end{bmatrix}(3,2,1)^\top&=(1,0,0)\\
  \begin{bmatrix}
      -5 & 9 & -2 \\
      -3 & 5 & -1 \\
      1 & -2 & 1 \\
  \end{bmatrix}(2,2,1)^\top&=(6,-3,-1)\\
  \begin{bmatrix}
      -5 & 9 & -2 \\
      -3 & 5 & -1 \\
      1 & -2 & 1 \\
  \end{bmatrix}(-1,7,1)^\top&=(66,-37,-14)\\
\end{align*}

\section*{Exercise 3}
\textbf{Problem:} Prove the following for any pair of $2\times2$ matrices $A$ and $B$:
$$\operatorname{det}(AB)=\operatorname{det}(A)\operatorname{det}(B)\ \ \ \ \ \  \operatorname{det}(A^{-1})=\operatorname{det}(A)^{-1}$$
\textbf{Solution:} Notice that:
\begin{align*}
  \operatorname{det}(A)\operatorname{det}(B)&=\operatorname{det}\left(\begin{bmatrix}
      a_1 & b_1 \\
      c_1 & d_1 \\
  \end{bmatrix}\right)
  \left(\begin{bmatrix}
      a_2 & b_2 \\
      c_2 & d_2 \\
  \end{bmatrix}\right)=
  (a_1d_1-b_1c_1)(a_2d_2-b_2c_2)\\
  &=a_1a_2d_1d_2-a_1b_2c_2d_1-a_2b_1c_1d_2+b_1b_2c_1c_2
\end{align*}

Now notice that:
\begin{align*}
  \operatorname{det}(AB)&=\operatorname{det}\left(\begin{bmatrix}
      a_1 & b_1 \\
      c_1 & d_1 \\
  \end{bmatrix}
  \begin{bmatrix}
      a_2 & b_2 \\
      c_2 & d_2 \\
  \end{bmatrix}\right)=\operatorname{det}\left(
  \begin{bmatrix}
      a_1a_2+b_1c_2 & a_1b_2+b_1d_2 \\
      c_1a_2+d_1c_2 & c_1b_2+d_1d_2 \\
  \end{bmatrix}\right)\\
  &=(a_1a_2+b_1c_2)(c_1b_2+d_1d_2)-(a_1b_2+b_1d_2)(c_1a_2+d_1c_2)\\
  &=a_1a_2d_1d_2-a_1b_2c_2d_1-a_2b_1c_1d_2+b_1b_2c_1c_2
\end{align*}

And so for any pair of $2\times2$ matrices, the first identity holds. The second one can be shown more simply. First notice that:
$$(\operatorname{det}(A))^{-1}=\operatorname{det}\left(
\begin{bmatrix}
    a & b \\
    c & d \\
\end{bmatrix}
\right)^{-1}=(ad-bc)^{-1}$$

Now notice that:
\begin{align*}(\operatorname{det}(A^{-1})&=\operatorname{det}
\left((ad-bc)^{-1}\begin{bmatrix}
    d & -b \\
    -c & a \\
\end{bmatrix}\right)\\
&=\left((ad-bc)^{-1}\right)^2\operatorname{det}\left(
\begin{bmatrix}
    d & -b \\
    -c & a \\
\end{bmatrix}\right)\tag{$\operatorname{det}(kA)=k^n\operatorname{det}(A)$}\\
&=(ad-bc)^{-2}(da-(-b)(-c))\\
&=(ad-bc)^{-1}
\end{align*}

And so the second equality holds as well.

\section*{Exercise 4}
\subsection*{Part a}
\textbf{Problem:} Find the Jacobian of the following function, evaluate it at $(-1,1)$ then compute its inverse:
$$\mathbf f(x,y)=((x^3-x^2)y,xy+x-y)$$
\textbf{Solution:} First we'll compute $J\mathbf f$:
$$J\mathbf f(x,y)=
\begin{bmatrix}
    3x^2y-2xy & x^3-x^2 \\
    y+1 & x-1 \\
\end{bmatrix}$$

Evaluating at $(-1,1)$ we get:
$$J\mathbf f(-1,1)=
\begin{bmatrix}
    5 & -2 \\
    2 & -2 \\
\end{bmatrix}$$

The inverse is found just like before:
$$(J\mathbf f(-1,1))^{-1}=
\frac{1}{5*(-2)-(-2)*2}\begin{bmatrix}
    -2 & 2 \\
    -2 & 5 \\
\end{bmatrix}=
\frac{1}{6}\begin{bmatrix}
    2 & -2 \\
    2 & -5 \\
\end{bmatrix}$$

\subsection*{Part b}
\textbf{Problem:} Find the Jacobian of the following function, evaluate it at $(-2,-3)$ then compute its inverse:
$$\mathbf g(x,y)=\left(u^2-v^2+uv,\frac{u^3}{3}-v^2\right)$$
\textbf{Solution:} First we'll compute $J\mathbf g$:
$$J\mathbf g(u,v)=
\begin{bmatrix}
    2u+v & u-2v \\
    u^2 & -2v \\
\end{bmatrix}$$

Evaluating at $(-2,-3)$ we get:
$$J\mathbf f(-2,-3)=
\begin{bmatrix}
    -7 & 7 \\
    4 & 6 \\
\end{bmatrix}$$

The inverse is found just like before:
$$(J\mathbf f(-1,1))^{-1}=
\frac{1}{(-7)*6-7*4}\begin{bmatrix}
    6 & -7 \\
    -4 & -7 \\
\end{bmatrix}=
\frac{1}{70}\begin{bmatrix}
    -6 & 7 \\
    4 & 7 \\
\end{bmatrix}$$

\section*{Exercise 5}
Consider the functions $\mathbf f$ and $\mathbf g$ defined above. Notice that their codomain and domain match up meaning we can define the following composition:
$$\mathbf h(x,y)=\mathbf g(\mathbf f(x,y))$$
\subsection*{Part a}
\textbf{Problem:} Give an explicit formula for $\mathbf h(x,y)$ then find the Jacobian of $\mathbf h$ at $(-1,1)$.
\\\\
\textbf{Solution:} Finding the explicit formula is a matter of substitution:
\begin{align*}
  h_1(x,y)&=y^2(x^6-2x^5+2x^4-2x^3+2x-1)+y(x^4-x^3-2x^2)-x^2\\
  h_2(x,y)&=\frac{y^3}{3}(x^9-3x^8+3x^7-x^6)-y^2(x^2-2x+1)+2y(x^2-x)-x^2
\end{align*}

Computing the Jacobian and plugging in $(-1,1)$ we find:
$$J\mathbf h(-1,1)=\begin{bmatrix}
    -27 & 6 \\
    32 & -20 \\
\end{bmatrix}$$

\subsection*{Part b}
\textbf{Problem:} Verify that the chain rule holds for these functions at the point $\mathbf x_0=(1,2)$. That is to say:
$$J(\mathbf g\circ\mathbf f)(\mathbf x_0)=[J(\mathbf g(\mathbf f(\mathbf x_0))][J(\mathbf f(\mathbf x_0)]$$
\textbf{Solution:} Now notice that $\mathbf f(-1,1)=(-2,-3)$ meaning:
$$[J(\mathbf g(\mathbf f(\mathbf x_0))][J(\mathbf f(\mathbf x_0)]=[J(\mathbf g(\mathbf f(-2,3))][J(\mathbf f(-1,1)]$$

Multiplying these two matrices we find:
$$[J(\mathbf g(\mathbf f(-2,3))][J(\mathbf f(-1,1)]=\begin{bmatrix}
    -8 & 1 \\
    9 & 4 \\
\end{bmatrix}
\begin{bmatrix}
    -5 & -2 \\
    2 & -2 \\
\end{bmatrix}=
\begin{bmatrix}
    -27 & 6 \\
    32 & -20 \\
\end{bmatrix}$$

And so, in accordance with our result from Part a, we have verified the chain rule.

\section*{Exercise 6}
\subsection*{Part a}
\textbf{Problem:} Show that the following function $\mathbf m:\mathbb R^{n+1}\to\mathbb R^n$ is differentiable and calculate the Jacobian.
$$\mathbf m(x_1,\cdots,x_n,x_{n+1})=x_{n+1}(x_1,\cdots,x_n)$$
\textbf{Solution:} Notice that the partial derivatives of $\mathbf m$ take the form:
$$\frac{\partial m_j}{\partial x_i}=\delta_{ij}$$

Notice this leaves out the last case, which we can deal with for the whole vector:
$$\frac{\partial \mathbf m}{\partial x_{n+1}}=(x_1,\cdots,x_n)$$

And since all these partials are defined and continuous, the function is certainly differentiable. The Jacobian then is simply the organization of the above partial derivatives into an $(n+1)\times n$ matrix:
$$J\mathbf m=
\begin{bmatrix}
    1 & 0 & 0 & \cdots & 0 & x_1 \\
    0 & 1 & 0 & \cdots & 0 & x_2 \\
    0 & 0 & 1 & \cdots & 0 & x_3 \\
    \vdots & \vdots & \vdots & \ddots & \vdots & \vdots\\
    0 & 0 & 0 & \cdots & 1 & x_n \\
\end{bmatrix}$$

Or column-wise $J\mathbf m=(\mathbf e_1,\cdots,\mathbf e_n,\mathbf x)$.

\subsection*{Part b}
\textbf{Problem:} Consider two differentiable functions $\mathbf f:\mathbb R^n\to\mathbb R^m$ and $\mathbf g:\mathbb R^l\to\mathbb R^k$. Now consider the function:
$$\mathbf h(\mathbf x,\mathbf y)=(\mathbf f(\mathbf x),\mathbf g(\mathbf y))$$
where $\mathbf h:\mathbb R^n\times\mathbb R^l\to\mathbb R^m\times\mathbb R^k$. Assuming it exists, calculate the Jacobian. Then verify that this linear map satisfies the definition of the derivative for $\mathbf h$.
\\\\
\textbf{Solution:}

\subsection*{Part c}
\textbf{Problem:} whatever
\\\\
\textbf{Solution:}

\end{document}

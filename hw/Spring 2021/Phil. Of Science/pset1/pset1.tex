\documentclass{article}
\usepackage[dvipsnames]{xcolor}
\usepackage{amsmath}
\usepackage{graphicx}
\usepackage{enumitem}
\usepackage{centernot}
\usepackage{setspace}
\usepackage[margin=0.95in]{geometry}
\usepackage{titling}

\setlength{\droptitle}{-7em}   % This is your set screw

\begin{document}

\title{Philosophy of Science\\ Problem Set \#1}
\author{Ozaner Hansha}
\date{February 28, 2021}
\maketitle

\subsection*{Question 1}
\noindent\textbf{Part a:} What are the two basic ingredients of evolution?
\bigskip

\noindent\textbf{Response:} The two primary factors that lead to evolution in a population are:
\begin{enumerate}
    \item Substantial variation in that population
    \item A `struggle' for `survival' between the members of the population. Which is to say that, based on the individuals' characteristics, only some members of the population will survive into the future (either in themselves or their children/derivatives).
\end{enumerate}
\bigskip

\noindent\textbf{Part b:} Describe four populations: one with both ingredients, one with neither, one with one ingredient but not the other, and one with the latter but not the former.
\bigskip

\noindent\textbf{Response:} Below we give 4 populations:
\begin{enumerate}
    \item \textbf{Has factors 1 \& 2}: The population of cell phones, with it's large variety of different features, specifications, and prices. Also recall that there is competition between these different variations of cell phones for market share amongst consumers.
    \item \textbf{Has factor 1 only}: Consider a container filled with 500 marbles, of different colors and sizes, that is just sitting in a warehouse doing nothing forever.
    \item \textbf{Has factor 2 only}: Consider a marble track with 500 identical lanes for 500 identical marbles that takes approximately 5 seconds to finish. The marbles that finish after 5 seconds will fall into a fire pit and be destroyed. The winning marbles are duplicated and added back into the population. As the tracks and marbles are identical, random variation between the tracks at any one time (e.g. wind) is essentially the only determining factor of winning. In other words, any one marble has an equal probability of winning as any other.
    \item \textbf{Has neither factor}: Consider a large population of essentially identical sand particles in deep space.
\end{enumerate}
\bigskip

\noindent\textbf{Part c:} For the first population, offer an evolutionary explanation of why one trait is more prevalent in earlier generations than it is in later generations.
\bigskip

\noindent\textbf{Response:} Consider the current population of cell phones. Almost all of them have touch screens. However, looking back at this population in say the late 1990s and early 2000s, we find that this feature of cell phones was not nearly as prominent. Evolutionarily speaking, this is because in the early 2000s phones with touch screens (in large part due to their touch screens) became more popular than non-touch screen phones. As such, the former were sold more so than the latter. Companies that did not change their phones to have this feature could no longer compete against those who did, and thus stopped producing such phones or shifted into producing ones that \textit{did} have touch screens. This led to only companies producing touch screen phones surviving into the future, leaving the population of cell phones to consist almost entirely of touch screen phones.
\bigskip

\noindent\textbf{Part d:} For the first population, offer an evolutionary explanation of why one trait is less prevalent in earlier generations than it is in later generations.
\bigskip

\noindent\textbf{Response:} Consider the current population of cell phones. Almost none of them have IR blasters. These were common in the late 1990s and early 2000s yet today they are no longer prevalence in the cell phone population. Evolutionarily speaking it is because, as more consumer found the inclusion of an IR blaster to be a non-factor in their purchasing decision (presumably due to the rise of other local communication protocols like bluetooth and NFC), companies that continued to include them were simply wasting resources in including features in their phones that would not impact is survivability (i.e. its chance of being purchased). Thus the companies that produced phones with IR blasters either went out of business or shifted to not including them, leaving us with an IR blaster-less population of cell phones.
\bigskip

\noindent\textbf{Part e:} For one of the other populations, say what explains the prevalence or lack thereof of a certain trait at later times.  In addition, explain why that trait’s prevalence does not admit of an evolutionary explanation.
\bigskip

\noindent\textbf{Response:} Consider population 3, that of the marbles on tracks. We might wonder why, over time, the population does not include individuals more likely to survive (i.e. make it in under 5 seconds). This is because all the marbles are identical (i.e. the population has no substantial variation). As a result, the process of selection (choosing the winning marbles) does not favor any particular trait, indeed there are none to favor. Without initial variation or even some way of producing variation (e.g. mutation), the marbles will stay identical in each round.

Indeed since this population has only one of the two factors necessary for evolution, its traits over time cannot be explained via evolution.

\subsection*{Question 4}
\noindent\textbf{Part a:} Explain how evolutionary debunking arguments against moral realism are supposed to work.
\bigskip

\noindent\textbf{Response:} An evolutionary debunking argument against moral realism ought to show that the fact that we have evolution implies that the common belief that there are moral truths is false or at least unlikely. We detail it in part c.
\bigskip

\noindent\textbf{Part b:} Using the optometrist example, explain why an evolutionary argument against moral realism that satisfies all of Vavova’s criteria would be more of a worry for moral realists than a skeptical argument against moral realism that satisfies none of those criteria.
\bigskip

\noindent\textbf{Response:} Let us remind ourselves of Vavova's criteria. To satisfy it, our evolutionary debunking argument must be empirical,targeted, and epistemological.

Let us consider the optometrist example. A person sees a blue wall. The skeptic casts doubt on this by levying the possibility that the person is actually dreaming and that what he sees is not real. The optometrist casts doubt on this by instead noting that the person's test-results show that they are blue-yellow colorblind.

While both arguments satisfy the last criterion, namely that they make an epistemological claim, i.e. you are not actually looking at a blue wall, only the latter satisfies all the criteria.

First, the skeptic's argument is not empirical as you couldn't, even in principle, test if you were dreaming/in a simulation unless it was designed to be such. Indeed, we could come up with unfalsifiable arguments to any philosophical problem but they are hardly interesting nor shed any light on the situation (e.g. that's not a duck its an alien that just acts, looks, and is otherwise empirically unmistakable as a duck).

Secondly, the skeptic's argument isn't targeted. This is to say that you can levy the dreaming argument against anything and everything. Indeed the person telling you the argument could be part of the dream, you might be dreaming that the logic to follow the dreaming argument is valid when it really is not, etc. This is even more devastating than the argument being unfalsifiable because now accepting that you are dreaming is contradictory to the argument itself.

In contrast, the optometrist's argument is targeted as it only makes a statement about your perception of color and not all of reality. It is alo empirical as it relies on an empirical test (the color blindess test) rather than being some unfalsifiable claim.

This example implies that an evolutionary argument ought to satisfy those 3 criteria, lest they be as weak as the skeptic's argument.

\bigskip

\noindent\textbf{Part c:} State an evolutionary debunking argument against moral realism that satisfies all of Vavova’s criteria and raise two objections to that argument.
\bigskip

\noindent\textbf{Response:} We first note that ourselves and our beliefs, including our moral beliefs, are products of evolutionary forces. We then note that evolution favors individuals that are more fit, and not necessarily those that know the truth. Taking this to heart and assuming we are moral realists, this would imply that the moral beliefs we hold to be true are in all likelihood false. After all, it is not experiment nor empirical evidence that has produced these beliefs but our psychology, culture, and society all of which the workings are influenced by evolution.

As a result, there is a proverbial `gap' between moral realism and evolution, and we have good reason to believe that our moral beliefs are actually mistaken. And since we have no real reason to believe our moral beliefs are correct, it would be fair to say we lack moral knowledge.

One objection that might be touted at this argument is at ``we have no real reason to believe our moral beliefs are correct.'' Indeed, if one could provide independent confirmation of our moral beliefs then we would have good reason to believe our moral beliefs.

Another similar objection is to deny that there is a gap between moral realism and the fit traits evolution imparts on us. One way would be to show that knowing moral truths are correlated with fitness.
\bigskip

\noindent\textbf{Part d:} Say whether you think the objections succeed and why.
\bigskip

\noindent\textbf{Response:} Neither of these objections seem particularly good to me as they both rely on there being an actual moral objective that we can somehow test and verify to provide either an independent confirmation of our moral beliefs, or show that they are correlated with fitness. If we could provide a verifiable mechanism of moral beliefs (whatever that even means) then we would have no point in discussing this and morality would be a grounded and empirical science. Clearly, however, that is currently not the case and it remains wholly unclear how that could come to be.

While it's not as clear cut as this, the objections seem circular as they contest that there are indeed moral facts by making the stronger claim that there are (somehow) \textit{verifiable} moral facts.

\subsection*{Question 5}
\noindent\textbf{Part a:} Explain how Plantinga's evolutionary argument against naturalism is supposed to work.
\bigskip

\noindent\textbf{Response:} Platinga's argument goes like this:
\begin{enumerate}
    \item[P1] If both naturalism and evolution are true, then the probability that our cognitive faculties are reliable is low.
    \item[P2] If this probability is low, then anybody who belives in both naturalism and evolution has reason to doubt that their own cognitive faculties are reliable.
    \item[P3] If you have reason to doubt your cognitive faculties are reliable then you have reason to doubt all of your beliefs (including that of naturalism and evolution). 
    \item[C ] So, it would appear that it is irrational to believe in both naturalism and evolution as it is unstable.
\end{enumerate}

While this argument seems reasonable, it hinges on P1 and it's not immediately clear why that would be. However, as in the lecture, splitting up what beliefs could possibly be into 4 cases shows us that this seems the case.
\bigskip

\noindent\textbf{Part b:} Provide two possible responses to the argument.
\bigskip

\noindent\textbf{Response:} The rationale of P1 discussed the different possible ways evolution could have imparted beliefs on us. However, since we know we are humans with a specific evolutionary history, we should consider that history rather than possible ones. In this history it seems to us as though we have basic desires like needing food, shelter, wanting to reproduce etc. Having true beliefs seems to have helped humans to attaining these goals.

To take seriously Platinga's argument then, we must consider if there is an alternate story of human evolution in which we have `weird' desires and false beliefs about them that all happen to line up to how we are today.

A second objection is that of systematic error. While it may be true that humans often introduce systematic error into their beliefs, e.g. attributing consciousness to misunderstood phenomena, often times they are still mostly right. Certainly believing that the harvest wasn't bountiful because the rain god wasn't happy isn't exactly correct, it still is the case that it was the lack of rain that affected the harvest. And so even in having systemically false beliefs, humans seem to be able to come across mostly true statements, and can then further refine them, despite evolution under naturalism.
\bigskip

\noindent\textbf{Part c:} Say how the existence of God is supposed to help with the problem Plantinga raises.
\bigskip

\noindent\textbf{Response:} Platinga holds that believing in the conjunction of God and evolution deals with the rationale used to justify P1. In this view God would ensure that our evolution would happen in such a way as for us to have reliable cognitive faculties. 

Unlike naturalism, in which nature would not care one way or the other if our cognitive faculties are reliable, just if they are fit.
\bigskip

\noindent\textbf{Part d:} Give an example of each of the following for a hypothetical human ancestor:
\begin{itemize}
    \item A belief-desire pair where the belief is false, but the pair as a whole makes the organism more likely to survive.
    \item A belief-desire pair where the belief is true, but the pair as a whole makes the organism less likely to survive.
\end{itemize}
\bigskip

\noindent\textbf{Response:} If a human believed that eating certain (unbeknownst to them) poisonous plants is not looked favorably in the eyes of God, and indeed desired to be looked favorably at by God, then they would not eat the plants. This makes the human more likely to survive despite their belief being false.

Similarly, if a human believed that the plant was poisonous and also desired to not die, then they would not eat the plant either. Here their belief is true \textit{and} the human more likely to survive.
\bigskip

\noindent\textbf{Part e:} Say how the answer to 5d is relevant to Platinga’s argument.
\bigskip

\noindent\textbf{Response:} Part d shows that the fitness of beliefs a human ancestor may have can be independent of the truth. This gives credence to P1 of his argument as it could be the case that our beliefs are not true at all and are just fit, if there was no god to guide the development of those beliefs (or rather the cognitive faculties that produced them).

\end{document}
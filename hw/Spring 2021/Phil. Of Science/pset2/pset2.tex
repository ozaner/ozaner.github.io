\documentclass{article}
\usepackage[dvipsnames]{xcolor}
\usepackage{amsmath}
\usepackage{graphicx}
\usepackage{enumitem}
\usepackage{centernot}
\usepackage{setspace}
\usepackage[margin=0.95in]{geometry}
\usepackage{titling}

\setlength{\droptitle}{-7em}   % This is your set screw

\begin{document}

\title{Philosophy of Science\\ Problem Set \#2}
\author{Ozaner Hansha}
\date{April 6, 2021}
\maketitle

\subsection*{Question 1}
\noindent\textbf{Part a:} What does it mean to say that the universe is ``fine-tuned for life''?
\bigskip

\noindent\textbf{Response:} When people say that the universe is ``fine-tuned'' for life, they are remarking that the existence of life seems to be a coincidence of the particular values of the fundamental physical constants, and that even a slight perturbation of these laws would make life impossible.
\bigskip

\noindent\textbf{Part b:} It has been noted that the universe is fine-tuned for rocks just as much as it is fine-tuned for life. Explain what this means and give a reason for believing it.
\bigskip

\noindent\textbf{Response:} Just as the fundamental physical constants can be viewed as conspiring to give rise to a universe fit for life, they too can be viewed as conspiring to give rise to the existence of rocks. After all, just as with life, just a slight perturbation of the physical constants would result in a wildly different universe with a different chemisty or, more likely, no chemistry. This would prevent structures like rocks, or essentially anything else, from appearing.
\bigskip

\noindent\textbf{Part c:} Consider the Rock-God Hypothesis: that the universe was created by a God who was especially interested in rocks. Give an argument for this hypothesis over the hypothesis that the universe was created by chance. How does the Rock-God Hypothesis compare to the hypothesis that the universe was created by a God who was especially interested in life?
\bigskip

\noindent\textbf{Response:} The arguments is nearly identical to the familiar fine-tuning hypothesis.
\begin{enumerate}
    \item If the fundamental constants were just slightly different, there would be no rocks.
    \item  Without a Rock-God, the probability that the constants would turn out to be what they are (that is favoring rocks) on their own is very low.
    \item \textit{With} a Rock-God, the probability that the constants would turn out to be what they are is very high.
    \item Since the universe \textit{is} fine-tuned for rocks, the probability that there is a Rock-God is very high.
\end{enumerate}

It becomes clear then, that our reasoning for believing the fine-tuning of life works just as well for anything else whose existence is contingent on the constants of the universe (read: pretty much everything).
\bigskip

\noindent\textbf{Part d:} Thomas is an avid roulette player, having placed a lot of bets throughout his life.  His favorite bet is to bet on 0 to win, which has a 1/38 probability of winning (in America).  Suppose he tells you that he once won betting on 0 three times in a row.  Is this surprising?  Now suppose you walk in on him playing roulette, and see him win betting on 0 three times in a row.  Is this surprising? If the two outcomes differ in how surprising they are, say why.
\bigskip

\noindent\textbf{Response:} The probability of Thomas winning his bet on 0 three times in a row on any given occasion is given by:
$$\left(\frac{1}{38}\right)^3=\frac{1}{54872}\approx0.00182\%$$

This is a very low chance for this to happen for any specific instance of him betting such as, for example, an instance in which I walk in on him making a bet. On the other hand, that he has achieved this over the course of his \textit{entire} betting history is less surprising. Further, the more bets he has made over his life the less surprising this fact is. This is, of course, because the probability of any event $E$ occuring at least once grows as more trials are conducted. More tries, more chances.
\bigskip

\noindent\textbf{Part e:} What does the case(s) of Thomas tell us about the multiverse response to the fine-tuning argument?
\bigskip

\noindent\textbf{Response:} The multiverse response to fine-tuning uses this same reasoning to explain why our universe isn't fine-tuned. Given enough universes, one is bound to have the right fundamental constants for life. And it is tautological that we exist in such a universe and not another.
\bigskip

\subsection*{Question 2}
Suppose you are doing a logic puzzle in an Escape Room type event. If you successfully solve the puzzle, you will learn a secret about the room. If you incorrectly solve the puzzle, you will receive a false message about the room. After receiving the message, you can report your solution. If the reported solution is correct, you win an awesome prize. If the reported solution is incorrect, you get a “prize” that is the negative equivalent of awesome. If you don’t report a solution, you get nothing. You have an impressive track record for this type of event, having arrived at the correct solution 99\% of the time. After a lot of difficult logical inferences, you appear to solve the puzzle and receive the message: this room contains an undetectable gas, inhalation of which degrades logical thinking skills and leads people to incorrectly solve puzzles 1\% more often than they normally would. You detect no such gas.
\bigskip

\noindent\textbf{Part a:} Describe a sense in which a precise level of confidence in the hypothesis that you solved the puzzle is cognitively unstable.
\bigskip

\noindent\textbf{Response:} If you did indeed solve the puzzle correctly, then the message at the end is true and your odds of solving the puzzle weren't 99\% but 98\%. If you didn't solve the puzzle correctly, then your odds of solving the puzzle were indeed 99\%, but you still managed to fail with odds of 1\%.

You can't know which level of confidence to trust since you don't know if you solved it correctly.
\bigskip

\noindent\textbf{Part b:} How does the cognitive instability of that hypothesis bear on whether you should reject it? Explain a general moral of your answer for objections that criticize hypotheses for being cognitively unstable.
\bigskip

\noindent\textbf{Response:} If a hypothesis implies that the very reasoning used to prove that hypothesis is suspect or invalid, then it is cognitively unstable. If we do not reject such hypotheses, we lie open to skepticism about ourselves. Indeed there is an implicit starting point of our chain of reasoning about the world via induction. We assume that we are humans, observing said world with workable (not necessarily perfect) cognitive facilities. If we throw this assumption out, we are left with nothing but skepticism and no way out. We could be brains in a jar, made to think we are real to prove some sick meta-philsopher's point about cognitive instability. Or we could be anything else. There is nothing to be gleaned by reason and observation if we cannot trust it.

However for the hypothesis given above it, unless I interpreted it wrong, seems more like an unknown than the sort of cognitive instability that arises from, say, Boltzmann brains. In this case, it would seem we should assume that we solved it correctly as a measly 1\% decrease in solve rate still puts the case of correctly solved puzzle much higher than incorrectly solved.
\bigskip

\noindent\textbf{Part c:} Come up with an example not from the lecture/reading in which it is rational to reject a cognitively unstable hypothesis. Give a reason that one of the following is either more like that hypothesis or more like the hypothesis that you solved the logic problem: most observers are Boltzmann brains, the evolutionary argument against moral realism works, evolution and naturalism are both true.
\bigskip

\noindent\textbf{Response:} Believing that you are a computer simulation is a cognitively unstable hypothesis. If this was true then your very belifs about being a simulation may be suspect to tampering by some pc manager. It is more like the boltzmann brain hypothesis than the escape room as it makes us question all our experiences rather than some controlled and accounted for subset.
\bigskip

\noindent\textbf{Part d:} Come up with an example of a cosmology that was not in the lecture/reading-that is, a way the universe could be-that is susceptible to the Boltzmann brain problem. Then come up with an example that was not in the lecture/reading of a cosmology that is not susceptible to the Boltzmann Brain problem.  (The examples do not have to satisfy any other constraints—for instance, they do not have to be plausible from a physics-standpoint.)
\bigskip

\noindent\textbf{Response:} A cosmology that is susceptible to the problem is one in which the entropy was always extremley high. Real observers are still possible in such a universe, they would just be very long lived excitations. As such, they would be far less likely than Boltzmann brains, just as in the original cosmology.

A cosmology in which the universe is not susceptible to this problem is one in which entropy reaches some maximum in finite time (small enough that the chance of Boltzmann brains is negligible), then proceeds to reverse back to low entropy (for whatever physical reason). If this happens in a cycle of increasing then decreasing, then real observers are far more likely than fake ones.
\bigskip

\noindent\textbf{Part e:} In general, what does it take for a hypothesis to be susceptible to the Boltzmann brain problem?
\bigskip

\noindent\textbf{Response:} A hypothesis must imply that our observation and reasoning are inaccurate to be susceptible to the Boltzmann brain problem. This is to say, their truth must call into question the acceptance of their truth.
\bigskip

\subsection*{Question 3}
\noindent\textbf{Part a:} How should you respond to this news-should you conclude that you’re probably a simulated brain rather than an ordinary observer? Give three reasons why someone might think the answer is ``no''.
\bigskip

\noindent\textbf{Response:} Three reasons why you might reject that you are such a brain are:
\begin{enumerate}
    \item That very same broadcast notes that what these simulated brain states believe in is mostly false. So, if you were such a brain state, then your believing that you were one, or that you saw this broadcast, or that Tesla really has such a warehouse is unjustified.
    \item On the off chance you were a real observer, choosing to believe you were one is more beneficial to than not. And if you were a simulated brain, then believing you are not brings no harm as you'll cease to be soon enough.
    \item If you believe in a non-deceiving God or a Past Hypothesis equivalent to not being a Tesla brain, then you have reason to believe you are not one.
\end{enumerate}
\bigskip

\noindent\textbf{Part b:} Raise an objection to each of these answers.
\bigskip

\noindent\textbf{Response:} Some objections to them:
\begin{enumerate}
    \item While this may be true, the fact remains that the probability that you are a simulated brain that happened to receive the true information that Tesla has a warehouse of simulated brains is still higher than being a real observer of that fact.
    \item How beneficial a belief is has no bearing on its validity.
    \item Simply asserting the problem away, via God or some other hypothesis, seems like a cheap way to ignore the problem.
\end{enumerate}
\bigskip

\noindent\textbf{Part c:} Say how this situation is importantly different from one in which a skeptic simply challenges you to show that you’re not a brain in a vat.
\bigskip

\noindent\textbf{Response:} In this situation, you already have observational evidence of a huge collection of brains in vats. Unlike with the skeptic who you might argue that such a collection is implausible.
\bigskip

\noindent\textbf{Part d:} Is there an (empirical) experiment you could do in this scenario to determine whether you’re a simulated brain? If so, describe it. If not, explain why this does not immediately entail that there’s no way to adjudicate between the hypothesis that you’re a simulated brain and the hypothesis that you’re an ordinary observer.
\bigskip

\noindent\textbf{Response:} There is no empirical test to determine this, just as there is no empirical test to determine if I am not currently a Boltzmann brain or if there is a God. While this does mean our hypothesis (that we are simulated) is unfalsifiable, we can still decide between believing it or not on philosphical grounds. For example, one might reject this hypothesis because it is cognitively unstable.
\bigskip

\noindent\textbf{Part e:} How is this situation like one in which a scientific theory predicts that most observers are Boltzmann brains? Identify one possible difference between the two situations that might justify a different response to them.
\bigskip

\noindent\textbf{Response:} This is similar to the Boltzmann brains example because both hypotheses lead to distrusting our own reasoning, and thus the reasoning we used to justify the hypotheses. They are different, however, in that the Boltzmann brain idea can be brushed off by simply discarding cognitively unstable cosmologies, while this situation has demonstrated to us the reality of simulated brains. Ee cannot simply posit that these brains don't exist, while simultaneously trusting our own senses and reasoning (which is what allowed those Tesla engineers to make those brains).
\bigskip


\end{document}
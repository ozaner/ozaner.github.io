\documentclass{article}
\usepackage{amsmath,mathtools}
\usepackage{amssymb}
\usepackage[dvipsnames]{xcolor}
\usepackage{graphicx}
\usepackage{tikz}
\usepackage{pgfplots}
\usetikzlibrary{arrows}
\usetikzlibrary{datavisualization.formats.functions}
\usepgfplotslibrary{fillbetween}
\usetikzlibrary{patterns}
\usepackage{xargs}
\usepackage{enumitem}
\usepackage{systeme}
\usepackage{centernot}
\usepackage{physics}
\usepackage{xfrac}
\usepackage{titling}
\usepackage[margin=1in]{geometry}
\usepackage[skins,theorems]{tcolorbox}
\tcbset{highlight math style={enhanced,
  colframe=blue,colback=white,arc=0pt,boxrule=1pt}}

% calculus commands
\renewcommand{\eval}[3]{\left[#1\right]_{#2}^{#3}}

% linear algebra commands
\renewcommand\vec{\mathbf}
\newenvironment{sysmatrix}[1]
{\left[\begin{array}{@{}#1@{}}}
{\end{array}\right]}
\newcommand{\ro}[1]{%
\xrightarrow{\mathmakebox[\rowidth]{#1}}%
}
\newlength{\rowidth}% row operation width
\AtBeginDocument{\setlength{\rowidth}{3em}}

%set theory commands
\newcommand{\pset}[1]{\mathcal P(#1)}
\newcommand{\card}[1]{\operatorname{card}(#1)}
\newcommand{\R}{\mathbb R}

%optimization commands
\DeclareMathOperator*{\argmax}{arg\,max}
\DeclareMathOperator*{\argmin}{arg\,min}

\setlength{\droptitle}{-7em}   % This is your set screw

\begin{document}

\title{Linear Optimization\\HW \#1}
\author{Ozaner Hansha}
\date{February 1, 2021}
\maketitle

\subsection*{Problem a}
\noindent\textbf{Problem 4:} Solve for $\vec x$:
\begin{equation*}
    \begin{bmatrix}
        1&3&3\\2&3&5\\2&1&3
    \end{bmatrix}\vec x=\begin{bmatrix}
        4\\0\\1
    \end{bmatrix}
\end{equation*}

\noindent\textbf{Solution:} First we formulate our problem in terms of an augmented matrix, then we perform Guassian elimination: 
\begin{align*}
    \begin{sysmatrix}{rrr|r}
        1&3&3&4\\2&3&5&0\\2&1&3&1
    \end{sysmatrix}
    &\ro{r_2-2r_1}
    \begin{sysmatrix}{rrr|r}
        1&3&3&4\\0&-3&-1&-8\\2&1&2&1
    \end{sysmatrix}
    &&\ro{r_3-2r_1}
    \begin{sysmatrix}{rrr|r}
        1&3&3&4\\0&-3&-1&-8\\0&-5&-3&-7
    \end{sysmatrix}\\
    &\ro{(\sfrac{-1}{3})r_2}
    \begin{sysmatrix}{rrr|r}
        1&3&3&4\\0&1&\sfrac{1}{3}&\sfrac{8}{3}\\0&-5&-3&-7
    \end{sysmatrix}
    &&\ro{r_1-3r_2}
    \begin{sysmatrix}{rrr|r}
        1&0&2&-4\\0&1&\sfrac{1}{3}&\sfrac{8}{3}\\0&-5&-3&-7
    \end{sysmatrix}\\
    &\ro{r_3+5r_2}
    \begin{sysmatrix}{rrr|r}
        1&0&2&-4\\0&1&\sfrac{1}{3}&\sfrac{8}{3}\\0&0&\sfrac{-4}{3}&\sfrac{19}{3}
    \end{sysmatrix}
    &&\ro{(\sfrac{-3}{4})r_3}
    \begin{sysmatrix}{rrr|r}
        1&0&2&-4\\0&1&\sfrac{1}{3}&\sfrac{8}{3}\\0&0&1&\sfrac{-19}{4}
    \end{sysmatrix}\\
    &\ro{r_1-2r_3}
    \begin{sysmatrix}{rrr|r}
        1&0&0&\sfrac{11}{2}\\0&1&\sfrac{1}{3}&\sfrac{8}{3}\\0&0&1&\sfrac{-19}{4}
    \end{sysmatrix}
    &&\ro{r_2-(\sfrac{1}{3})r_3}
    \begin{sysmatrix}{rrr|r}
        1&0&0&\sfrac{11}{2}\\0&1&0&\sfrac{17}{4}\\0&0&1&\sfrac{-19}{4}
    \end{sysmatrix}\\
    &\implies\hspace{14pt}\systeme{
        x_1 = \sfrac{11}{2},
        x_2 = \sfrac{17}{4},
        x_3 = \sfrac{-19}{4}}
\end{align*}

And so our solution is $\vec x=\begin{bmatrix}
    \sfrac{11}{2} \\
    \sfrac{17}{4} \\
    \sfrac{-19}{4}
\end{bmatrix}$.
\bigskip

\noindent\textbf{Problem 16:} Find the solutions to:
\begin{equation*}
    \begin{bmatrix}
        1&2\\2&4
    \end{bmatrix}\vec x=\begin{bmatrix}
        1\\2
    \end{bmatrix}
\end{equation*}

Give the the solutions that maximize and minimize (if they exist) the cost function $C(x_1,x_2)=x_1^2+x_2^2$.
\bigskip

\noindent\textbf{Solution:} First note that, by performing the row operation $r_2-2r_1$, the system above reduces to:
\begin{equation*}
    \begin{bmatrix}
        1&2\\0&0
    \end{bmatrix}\vec x=\begin{bmatrix}
        1\\0
    \end{bmatrix}\implies x_1+2x_2=1
\end{equation*}

And so the solution set to the problem is given by:
\begin{equation*}
    \{x_1,x_2\in\mathbb R\mid x_1+2x_2=1\}
\end{equation*}

Now let us restrict the cost function to the solution set found above:
\begin{align*}
    C(x_1,x_2)&=x_1^2+x_2^2\\
    &=(1-2x_2)^2+x_2^2\tag{$x_1+2x_2=1\implies x_1=1-2x_2$}\\
    &=(1-4x_2+4x_2^2)+x_2^2\\
    &=\underbrace{5x_2^2-4x_2+1}_{a=5,b=-4,c=1}
\end{align*}

As the cost function is a quadratic polynomial, it has a single extremum at its vertex. That extremum is a global minimum if $a>0$ (i.e. the parabola opens upwards) and a global maximum if $a<0$ (i.e. it opens downwards). In our case it is a minimum:
\begin{align*}
    \argmin_{x_2}\,5x_2^2-4x_2+1&=\frac{-b}{2a}\tag{vertex of quadratic w/ $a>0$}\\
    &=\frac{4}{10}
\end{align*}

And using the equation we solved for earlier we know that the corresponding value of $x_1$ is given by:
\begin{equation*}
    x_1=1-2x_2=1-2\cdot\frac{4}{10}=\frac{1}{5}
\end{equation*}

And so the minimizing solution to the system of equations under cost function $C$ is given by $(\sfrac{1}{5},\sfrac{4}{10})$. On the other hand there is no maximum, as the cost function is a quadratic with $a>0$ (i.e. the parabola opens upwards). As such the cost can grow arbitrarily high as we increase $x_2$.
\bigskip

\noindent\textbf{Problem 22:} Give a differentiable function $f:\R\to\R$ that doesn't have an extremum at one of its critical points.
\bigskip

\noindent\textbf{Solution:} The function $f(x)=x^3$ has a single critical point at $x=0$:
\begin{align*}
    0&=\dv{x}x^3\\
    &=3x^2\\
    0&=x
\end{align*}

To see that this critical point is neither a min or max but instead an inflection point, consider the second derivative of $x^3$:
\begin{equation*}
    \dv[2]{x}x^3=\dv{x}3x^2=6x
\end{equation*}

You'll note that the second derivative is negative until it reaches the critical point $x=0$ and then turns positive after. In other words, $x^3$ is concave downwards to the left of the critical point and concave upwards to the right, and is thus an inflection point and not a min/max.
\bigskip
\newpage

\noindent\textbf{Problem 39:} Consider the following linear constraints:
\begin{align*}
    30x_1+5x_2\ge 60\\
    15x_1+10x_2\ge 70\\
    x_1,x_2\ge 0
\end{align*}

Graphed, these constraints give us the following:
\begin{center}
    \begin{tikzpicture}
    \begin{axis}[
        xlabel=$x_1$,
        ylabel=$x_2$,
        xmin=-2,xmax=10,
        ymin=-2,ymax=20,
        axis lines=center,
        legend pos=outer north east,
        legend style={legend cell align=right,legend plot pos=right}] 

    %green stuff
    \addplot[ForestGreen,thick,forget plot] coordinates{(0,-2) (0,20)};
    \addplot[ForestGreen,thick] coordinates{(-2,0) (10,0)};
    \addlegendentry{$x_1,x_2\ge0$}
    \plot[name path=P1, thick,samples=100,domain=0:10,
    forget plot,draw=none] {20};
    \plot[name path=P2,thick,samples=100,domain=0:10,
        forget plot,draw=none] {0};
    \addplot[fill=ForestGreen,
        opacity=.3,
        forget plot,
        % pattern=horizontal lines,
        pattern color=ForestGreen]
        fill between [of=P1 and P2, soft clip={domain=0:10}];

    %red stuff
    \addplot[color=Red,domain=-2:10,samples=100] {12-6*x};
    \addlegendentry{$30x_1+5x_2\ge10$}
    \plot[name path=G1, thick,samples=100,domain=-2:10,
        forget plot,draw=none] {12-6*x};
    \plot[name path=G2,thick,samples=100,domain=-2:10,
        forget plot,draw=none] {20};
    \addplot[fill=Red,
        % opacity=.8,
        forget plot,
        pattern=horizontal lines,
        pattern color=Red]
        fill between [of=G1 and G2, soft clip={domain=-2:10}];

    % Blue stuff
    \addplot[color=blue,domain=-2:10,samples=100] {14/3-x};
    \addlegendentry{$15x_1+10x_2\ge70$}
    \plot[name path=B1, thick,samples=100,domain=-2:10,
        forget plot,draw=none] {14/3-x};
    \plot[name path=B2,thick,samples=100,domain=-2:10,
        forget plot,draw=none] {20};
    \addplot[fill=blue,
        % opacity=.8,s
        forget plot,
        pattern=vertical lines,
        pattern color=blue]
        fill between [of=B1 and B2, soft clip={domain=-2:10}];

    %points
    \node[circle,fill,inner sep=1.5pt] at (axis cs:0,12) {};
    \node[circle,fill,inner sep=1.5pt] at (axis cs:1.5,3.2) {};
    \node[circle,fill,inner sep=1.5pt] at (axis cs:4.6,0) {};

    \end{axis}
    \end{tikzpicture}
\end{center}

Where the red lines, blue lines, and solid green background all overlap is the space of feasible solutions (i.e. solutions that satisfy all constraints). The 3 black dots are the vertices of the feasible space, the solution to any linear optimization problem under these constraints will be one of them.

Find a linear cost function $C(x_1,x_2)$ such that, under these constraints, the minimizing point is the right most vertex.
\bigskip

\noindent\textbf{Solution:} Choosing $C(x_1,x_2)=x_2$ as our cost function, the following is a graph of its level sets for $C=4,3,2,1,0,-1$:

\begin{center}
    \begin{tikzpicture}
    \begin{axis}[
        xlabel=$x_1$,
        ylabel=$x_2$,
        xmin=-2,xmax=10,
        ymin=-2,ymax=20,
        axis lines=center,
        legend pos=outer north east,
        legend style={legend cell align=right,legend plot pos=right}]

    %green stuff
    \addplot[ForestGreen,thick,forget plot] coordinates{(0,-2) (0,20)};
    \addplot[ForestGreen,thick] coordinates{(-2,0) (10,0)};
    \addlegendentry{$x_1,x_2\ge0$}
    \plot[name path=P1, thick,samples=100,domain=0:10,
    forget plot,draw=none] {20};
    \plot[name path=P2,thick,samples=100,domain=0:10,
        forget plot,draw=none] {0};
    \addplot[fill=purple,
        opacity=.3,
        forget plot,]
        fill between [of=P1 and P2, soft clip={domain=4.6:10}];

    %red stuff
    \addplot[color=Red,domain=-2:10,samples=100] {12-6*x};
    \addlegendentry{$30x_1+5x_2\ge10$}
    \plot[name path=G1, thick,samples=100,domain=-2:10,
        forget plot,draw=none] {12-6*x};
    \plot[name path=G2,thick,samples=100,domain=-2:10,
        forget plot,draw=none] {20};
    \addplot[fill=purple,
        opacity=.3,
        forget plot,]
        fill between [of=G1 and G2, soft clip={domain=0:1.5}];

    % Blue stuff
    \addplot[color=blue,domain=-2:10,samples=100] {14/3-x};
    \addlegendentry{$15x_1+10x_2\ge70$}
    \plot[name path=B1, thick,samples=100,domain=-2:10,
        forget plot,draw=none] {14/3-x};
    \plot[name path=B2,thick,samples=100,domain=-2:10,
        forget plot,draw=none] {20};
    \addplot[fill=purple,
        opacity=.3,
        forget plot,]
        fill between [of=B1 and B2, soft clip={domain=1.5:4.6}];

    %points
    \node[circle,fill,inner sep=1.5pt] at (axis cs:0,12) {};
    \node[circle,fill,inner sep=1.5pt] at (axis cs:1.5,3.2) {};
    \node[circle,fill,inner sep=1.5pt] at (axis cs:4.6,0) {};

    %level sets
    \addplot[dashed,domain=-2:10,samples=50,forget plot] {4};
    \node[label={0:{\scriptsize $C=4$}}] at (axis cs:5,4.5) {};
    \addplot[dashed,domain=-2:10,samples=50,forget plot] {3};
    \node[label={0:{\scriptsize $3$}}] at (axis cs:6,3.5) {};
    \addplot[dashed,domain=-2:10,samples=50,forget plot] {2};
    \node[label={0:{\scriptsize $2$}}] at (axis cs:6,2.5) {};
    \addplot[dashed,domain=-2:10,samples=50,forget plot] {1};
    \node[label={0:{\scriptsize $1$}}] at (axis cs:6,1.5) {};
    \addplot[dashed,domain=-2:10,samples=50,forget plot] {0};
    \node[label={0:{\scriptsize $0$}}] at (axis cs:6,0.5) {};
    \addplot[dashed,domain=-2:10,samples=50] {-1};
    \node[label={0:{\scriptsize $-1$}}] at (axis cs:6,-0.5) {};
    \addlegendentry{$C=4,3,2,1,0,-1$}

    \end{axis}
    \end{tikzpicture}
\end{center}

As we can see, of the feasible solutions (shaded in pink), the cost function reaches a minimum at $C=0$. This level set is presicely where the right most vertex is located (along with an infinite set of other optimal solutions on the line $x_1=0$). In other words, the linear cost function $C(x_1,x_2)=x_2$ has the rightmost vertex as a minimal solution. 

\subsection*{Problem b}
\noindent\textbf{Problem:} Given the following linear constraints, graph the set of feasible solutions:
\begin{align*}
    x+y\le5\\
    2x+y\le8\\
    x,y\ge0
\end{align*}

\noindent\textbf{Solution:} Graphing the constraints we have:
\begin{center}
    \begin{tikzpicture}
    \begin{axis}[
        xlabel=$x$,
        ylabel=$y$,
        xmin=-2,xmax=8,
        ymin=-2,ymax=10,
        axis lines=center,
        legend pos=outer north east,
        legend style={legend cell align=right,legend plot pos=right}] 

    %green stuff
    \addplot[ForestGreen,thick,forget plot] coordinates{(0,-2) (0,10)};
    \addplot[ForestGreen,thick] coordinates{(-2,0) (8,0)};
    \addlegendentry{$x,y\ge0$}
    \plot[name path=P1, thick,samples=100,domain=0:8,
    forget plot,draw=none] {10};
    \plot[name path=P2,thick,samples=100,domain=0:8,
        forget plot,draw=none] {0};
    \addplot[fill=ForestGreen,
        opacity=.3,
        forget plot,
        % pattern=horizontal lines,
        pattern color=ForestGreen]
        fill between [of=P1 and P2, soft clip={domain=0:10}];

    %red stuff
    \addplot[color=Red,domain=-2:8,samples=100] {5-x};
    \addlegendentry{$x+y\le5$}
    \plot[name path=G1, thick,samples=100,domain=-2:8,
        forget plot,draw=none] {5-x};
    \plot[name path=G2,thick,samples=100,domain=-2:8,
        forget plot,draw=none] {-2};
    \addplot[fill=Red,
        % opacity=.8,
        forget plot,
        pattern=horizontal lines,
        pattern color=Red]
        fill between [of=G1 and G2, soft clip={domain=-2:10}];

    % Blue stuff
    \addplot[color=blue,domain=-2:8,samples=100] {8-2*x};
    \addlegendentry{$2x+y\le8$}
    \plot[name path=B1, thick,samples=100,domain=-2:8,
        forget plot,draw=none] {8-2*x};
    \plot[name path=B2,thick,samples=100,domain=-2:8,
        forget plot,draw=none] {-2};
    \addplot[fill=blue,
        % opacity=.8,s
        forget plot,
        pattern=vertical lines,
        pattern color=blue]
        fill between [of=B1 and B2, soft clip={domain=-2:10}];

    %points
    \node[circle,fill,inner sep=1.5pt] at (axis cs:0,5) {};
    \node[circle,fill,inner sep=1.5pt] at (axis cs:3,2) {};
    \node[circle,fill,inner sep=1.5pt] at (axis cs:4,0) {};
    \node[circle,fill,inner sep=1.5pt] at (axis cs:0,0) {};

    \end{axis}
    \end{tikzpicture}
\end{center}

Where the space of feasible solutions is the are where the red lines, blue lines, and solid green backgrounds overlap. The vertices are given by the 4 black dots.

\subsection*{Problem c}
\noindent\textbf{Problem:} Consider the following linear optimization problem:
\begin{center}
    $\begin{aligned}
        &{\text{Maximize}}
        &&3x+4y \\
        &{\text{subject to}}
        &&x+3y\le6\\
        &
        &&4x+3y\le12\\
        &{\text{and}}
        &&x,y\ge0
    \end{aligned}$
\end{center}

Graph the set of feasible solutions, plot the level curves of the cost function for $k=6,8,10,12$, find the maximum value and the point(s) at which this maximum is achieved.
\bigskip

\noindent\textbf{Solution:} Below is a graph of the constraints and resulting vertices and feasible region:
\begin{center}
    \begin{tikzpicture}
    \begin{axis}[
        xlabel=$x$,
        ylabel=$y$,
        xmin=-2,xmax=7,
        ymin=-2,ymax=6,
        axis lines=center,
        legend pos=outer north east,
        legend style={legend cell align=right,legend plot pos=right}] 

    %green stuff
    \addplot[ForestGreen,thick,forget plot] coordinates{(0,-2) (0,7)};
    \addplot[ForestGreen,thick] coordinates{(-2,0) (8,0)};
    \addlegendentry{$x,y\ge0$}
    \plot[name path=P1, thick,samples=100,domain=0:7,
    forget plot,draw=none] {6};
    \plot[name path=P2,thick,samples=100,domain=0:7,
        forget plot,draw=none] {0};
    \addplot[fill=ForestGreen,
        opacity=.3,
        forget plot,
        % pattern=horizontal lines,
        pattern color=ForestGreen]
        fill between [of=P1 and P2, soft clip={domain=0:7}];

    %red stuff
    \addplot[color=Red,domain=-2:8,samples=100] {2-x/3};
    \addlegendentry{$x+3y\le6$}
    \plot[name path=G1, thick,samples=100,domain=-2:8,
        forget plot,draw=none] {2-x/3};
    \plot[name path=G2,thick,samples=100,domain=-2:8,
        forget plot,draw=none] {-2};
    \addplot[fill=Red,
        % opacity=.8,
        forget plot,
        pattern=horizontal lines,
        pattern color=Red]
        fill between [of=G1 and G2, soft clip={domain=-2:7}];

    % Blue stuff
    \addplot[color=blue,domain=-2:8,samples=100] {4-4*x/3};
    \addlegendentry{$4x+3y\le12$}
    \plot[name path=B1, thick,samples=100,domain=-2:8,
        forget plot,draw=none] {4-4*x/3};
    \plot[name path=B2,thick,samples=100,domain=-2:8,
        forget plot,draw=none] {-2};
    \addplot[fill=blue,
        % opacity=.8,s
        forget plot,
        pattern=vertical lines,
        pattern color=blue]
        fill between [of=B1 and B2, soft clip={domain=-2:7}];

    %points
    \node[circle,fill,inner sep=1.5pt] at (axis cs:0,2) {};
    \node[circle,fill,inner sep=1.5pt] at (axis cs:2,1.3) {};
    \node[circle,fill,inner sep=1.5pt] at (axis cs:3,0) {};
    \node[circle,fill,inner sep=1.5pt] at (axis cs:0,0) {};

    \end{axis}
    \end{tikzpicture}
\end{center}

Below we shade only the feasible region (in pink) as well as graph the desired level sets of the cost function:
\begin{center}
    \begin{tikzpicture}
    \begin{axis}[
        xlabel=$x$,
        ylabel=$y$,
        xmin=-2,xmax=7,
        ymin=-2,ymax=6,
        axis lines=center,
        legend pos=outer north east,
        legend style={legend cell align=right,legend plot pos=right}] 

    %green stuff
    \addplot[ForestGreen,thick,forget plot] coordinates{(0,-2) (0,7)};
    \addplot[ForestGreen,thick] coordinates{(-2,0) (8,0)};
    \addlegendentry{$x,y\ge0$}
    \plot[name path=P1, thick,samples=100,domain=0:7,
    forget plot,draw=none] {6};
    \plot[name path=P2,thick,samples=100,domain=0:7,
        forget plot,draw=none] {0};

    %red stuff
    \addplot[color=Red,domain=-2:8,samples=100] {2-x/3};
    \addlegendentry{$x+3y\le6$}
    \plot[name path=G1, thick,samples=100,domain=-2:8,
        forget plot,draw=none] {2-x/3};
    \plot[name path=G2,thick,samples=100,domain=-2:8,
        forget plot,draw=none] {0};
    \addplot[fill=purple,
        opacity=.3,
        forget plot]
        fill between [of=G1 and G2, soft clip={domain=0:2}];

    % Blue stuff
    \addplot[color=blue,domain=-2:8,samples=100] {4-4*x/3};
    \addlegendentry{$4x+3y\le12$}
    \plot[name path=B1, thick,samples=100,domain=-2:8,
        forget plot,draw=none] {4-4*x/3};
    \plot[name path=B2,thick,samples=100,domain=-2:8,
        forget plot,draw=none] {0};
    \addplot[fill=purple,
        opacity=.3,
        forget plot,]
        fill between [of=B1 and B2, soft clip={domain=2:3}];

    %points
    \node[circle,fill,inner sep=1.5pt] at (axis cs:0,2) {};
    \node[circle,fill,inner sep=1.5pt] at (axis cs:2,1.3) {};
    \node[circle,fill,inner sep=1.5pt] at (axis cs:3,0) {};
    \node[circle,fill,inner sep=1.5pt] at (axis cs:0,0) {};

%     %level sets
    \addplot[dashed,domain=-2:10,samples=50,forget plot] {3/2-3*x/4};
    \node[label={0:{\scriptsize $k=6$}}] at (axis cs:3.6,-1.8) {};
    \addplot[dashed,domain=-2:10,samples=50,forget plot] {2-3*x/4};
    \node[label={0:{\scriptsize $8$}}] at (axis cs:4.7,-1.6) {};
    \addplot[dashed,domain=-2:10,samples=50,forget plot] {5/2-3*x/4};
    \node[label={0:{\scriptsize $10$}}] at (axis cs:5,-1.4) {};
    \addplot[dashed,domain=-2:10,samples=50] {3-3*x/4};
    \node[label={0:{\scriptsize $12$}}] at (axis cs:5.4,-1.2) {};
    \addlegendentry{$k=6,8,10,12$}

    \end{axis}
    \end{tikzpicture}
\end{center}

As we can see from the level sets, the cost function increases in the $(+x, +y)$ direction, while it decreases in the $(-x,-y)$ direction.

Clearly then, since the feasible solution the farthest in the $(+x, +y)$ direction is the top left vertex, it is the maximal solution. This point is the intersection of the blue and red line. To solve for $x$ we have:
\begin{align*}
    &4x+3y=12\tag{blue line}\\
    -&(x+3y=6)\tag{red line}\\
    &\noindent\rule{4cm}{0.4pt}\\
    &3x=6\\
    \implies&x=2
\end{align*}

With $x$ in hand, we can solve for $y$ quite simply by plugging it into either line:
\begin{align*}
    6&=x+3y\tag{red line}\\
    &=2+3y\tag{plug in $x=2$}\\
    \frac{4}{3}&=y
\end{align*}

And so the minimal point is at $(2,\sfrac{4}{3})$. We can find the maximal value by simply plugging this point into the cost function:
$$C(2,\sfrac{4}{3})=3\cdot2+4\cdot\frac{4}{3}=\frac{34}{3}$$
\newpage

\subsection*{Problem d}
\noindent\textbf{Problem:} Solve the following linear optimization problem:
\begin{center}
    $\begin{aligned}
        &{\text{Maximize}}
        &&3x+y \\
        &{\text{subject to}}
        &&-3x+y\ge6\\
        &
        &&3x+5y\le15\\
        &{\text{and}}
        &&x,y\ge0
    \end{aligned}$
\end{center}

\noindent\textbf{Solution:}  Below is a graph of the constraints and their resulting vertices and feasible region:
\begin{center}
    \begin{tikzpicture}
    \begin{axis}[
        xlabel=$x$,
        ylabel=$y$,
        xmin=-5,xmax=7,
        ymin=-2,ymax=9,
        axis lines=center,
        legend pos=outer north east,
        legend style={legend cell align=right,legend plot pos=right}] 

    %green stuff
    \addplot[ForestGreen,thick,forget plot] coordinates{(0,-2) (0,9)};
    \addplot[ForestGreen,thick] coordinates{(-2,0) (9,0)};
    \addlegendentry{$x,y\ge0$}
    \plot[name path=P1, thick,samples=100,domain=0:9,
    forget plot,draw=none] {9};
    \plot[name path=P2,thick,samples=100,domain=0:9,
        forget plot,draw=none] {0};
    \addplot[fill=ForestGreen,
        opacity=.3,
        forget plot,
        % pattern=horizontal lines,
        pattern color=ForestGreen]
        fill between [of=P1 and P2, soft clip={domain=0:9}];

    %red stuff
    \addplot[color=Red,domain=-5:9,samples=100] {6+3*x};
    \addlegendentry{$-3x+y\ge6$}
    \plot[name path=G1, thick,samples=100,domain=-5:9,
        forget plot,draw=none] {6+3*x};
    \plot[name path=G2,thick,samples=100,domain=-5:9,
        forget plot,draw=none] {9};
    \addplot[fill=Red,
        % opacity=.9,
        forget plot,
        pattern=horizontal lines,
        pattern color=Red]
        fill between [of=G1 and G2, soft clip={domain=-5:9}];

    % Blue stuff
    \addplot[color=blue,domain=-5:9,samples=100] {3-3*x/5};
    \addlegendentry{$4x+3y\le12$}
    \plot[name path=B1, thick,samples=100,domain=-5:9,
        forget plot,draw=none] {3-3*x/5};
    \plot[name path=B2,thick,samples=100,domain=-5:9,
        forget plot,draw=none] {-2};
    \addplot[fill=blue,
        % opacity=.9,s
        forget plot,
        pattern=vertical lines,
        pattern color=blue]
        fill between [of=B1 and B2, soft clip={domain=-5:9}];
    \end{axis}
    \end{tikzpicture}
\end{center}

You'll notice that there is no region in which all three constraints overlap, which is to say, there is no feasible region. As such, this linear optimization problem has no solutions.

\end{document}
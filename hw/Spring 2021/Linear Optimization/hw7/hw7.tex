\documentclass{article}
\usepackage{amsmath,mathtools}
\usepackage{amssymb}
\usepackage[dvipsnames]{xcolor}
\usepackage{graphicx}
\usepackage{tikz}
\usepackage{float}
\usepackage{subcaption}
\usepackage{pgfplots}
\usetikzlibrary{arrows}
\usetikzlibrary{datavisualization.formats.functions}
\usepgfplotslibrary{fillbetween}
\usetikzlibrary{patterns}
\usepackage{xargs}
\usepackage{enumitem}
\usepackage{systeme}
\usepackage{centernot}
\usepackage{physics}
\usepackage{xfrac}
\usepackage{titling}
\usepackage[margin=1in]{geometry}
\usepackage[skins,theorems]{tcolorbox}
\tcbset{highlight math style={enhanced,
  colframe=blue,colback=white,arc=0pt,boxrule=1pt}}

% calculus commands
\renewcommand{\eval}[3]{\left[#1\right]_{#2}^{#3}}

% linear algebra commands
\newcommand{\icol}[1]{% inline column vector
  \begin{bsmallmatrix}#1\end{bsmallmatrix}%
}
\renewcommand\vec{\mathbf}
\newenvironment{sysmatrix}[1]
{\left[\begin{array}{@{}#1@{}}}
{\end{array}\right]}
\newcommand{\ro}[1]{%
\xrightarrow{\mathmakebox[\rowidth]{#1}}%
}
\newlength{\rowidth}% row operation width
\AtBeginDocument{\setlength{\rowidth}{3em}}

%set theory commands
\newcommand{\pset}[1]{\mathcal P(#1)}
\newcommand{\card}[1]{\operatorname{card}(#1)}
\newcommand{\R}{\mathbb R}

%optimization commands
\DeclareMathOperator*{\argmax}{arg\,max}
\DeclareMathOperator*{\argmin}{arg\,min}

%misc commands
\newcommand*\circled[1]{\tikz[baseline=(char.base)]{
             \node[shape=circle,draw,inner sep=2pt] (char) {#1};}}

\setlength{\droptitle}{-7em}   % This is your set screw

\begin{document}

\title{Linear Optimization\\HW \#7}
\author{Ozaner Hansha}
\date{March 22, 2021}
\maketitle

\subsection*{Chapter 7}
\noindent\textbf{Problem a:} Use the two-phase simplex method to solve the following:
$$\begin{aligned}
    &{\text{Maximize}}
    &&x_1+x_2-x_3-x_4\\
    &{\text{subject to}}
    &&x_1+2x_2+x_3\le7\\
    &
    &&2x_1-x_2-x_3-3x_4\le-1\\
    &{\text{and}}
    &&\vec x\ge 0
\end{aligned}$$
\bigskip

\noindent\textbf{Solution:} First let us transform the problem to standard form:
\begin{align*}
    \left\{\begin{aligned}
        &{\text{Maximize}}
        &&x_1+x_2-x_3-x_4\\
        &{\text{subject to}}
        &&x_1+2x_2+x_3\le7\\
        &
        &&2x_1-x_2-x_3-3x_4\le-1\\
        &{\text{and}}
        &&\vec x\ge 0
    \end{aligned}\right.
    &\implies
    \underbrace{\left\{\begin{aligned}
        &{\text{Maximize}}
        &&x_1+x_2-x_3-x_4\\
        &{\text{subject to}}
        &&x_1+2x_2+x_3+s_1=7\\
        &
        &&2x_1-x_2-x_3-3x_4+s_2=-1\\
        &{\text{and}}
        &&\vec x,\vec s\ge 0
    \end{aligned}\right.}_{\text{introduce slack variables } \vec s}\\
    &\implies
    \underbrace{\left\{\begin{aligned}
        &{\text{Maximize}}
        &&\begin{bmatrix}
            1&1&-1&-1&0&0
        \end{bmatrix}^\top\vec z\\
        &{\text{subject to}}
        &&\begin{bmatrix}
            1&2&1&0&1&0\\
            2&-1&-1&-3&0&1
        \end{bmatrix}\vec z=\begin{bmatrix}
            7\\-1
        \end{bmatrix}\\
        &{\text{and}}
        &&\vec z\ge 0
    \end{aligned}\right.}_{\text{let }\vec z=\icol{\vec x\\\vec s}}
\end{align*}

Note though that the solution given by $\icol{\vec 0\\\vec b}$ is not feasible. As such, we must proceed with the two-phase method and find another BFS. To do this, we add an artificial variable $t$ to the constraints corresponding to the negative entries of $\vec b$, giving us an auxillary problem:
\begin{align*}
    \underbrace{\left\{\begin{aligned}
        &{\text{Maximize}}
        &&x_1+x_2-x_3-x_4\\
        &{\text{subject to}}
        &&x_1+2x_2+x_3+s_1=7\\
        &
        &&2x_1-x_2-x_3-3x_4+s_2-t=-1\\
        &{\text{and}}
        &&\vec x,\vec s, t\ge 0
    \end{aligned}\right.}_{\text{introduce artificial variable }t}
    \implies
    \underbrace{\left\{\begin{aligned}
        &{\text{Maximize}}
        &&\begin{bmatrix}
            1&1&-1&-1&0&0&0
        \end{bmatrix}^\top\vec z\\
        &{\text{subject to}}
        &&\begin{bmatrix}
            1&2&1&0&1&0&0\\
            2&-1&-1&-3&0&1&-1
        \end{bmatrix}\vec z=\begin{bmatrix}
            7\\-1
        \end{bmatrix}\\
        &{\text{and}}
        &&\vec z\ge 0
    \end{aligned}\right.}_{\text{let }\vec z=\icol{\vec x\\\vec s\\t}}
\end{align*}

A BFS to the auxillary problem is given by $\begin{bmatrix}
    0&0&0&0&7&0&1
\end{bmatrix}^\top$. We can now use the tableau method to find another BFS to this problem that sets $t=0$. To do that we want to maximize $-t$:
\begin{align*}
    &2x_1-x_2-x_3-3x_4+s_2-t=-1\\
    \implies&-t=-2x_1+x_2+x_3+3x_4-s_2-1
\end{align*}

Leaving us with this problem:
$$\begin{aligned}
    &{\text{Maximize}}
    &&\begin{bmatrix}
        -2&1&1&3&0&-1&0
    \end{bmatrix}^\top\vec z\\
    &{\text{subject to}}
    &&\begin{bmatrix}
        1&2&1&0&1&0&0\\
        2&-1&-1&-3&0&1&-1
    \end{bmatrix}\vec z=\begin{bmatrix}
        7\\-1
    \end{bmatrix}\\
    &{\text{and}}
    &&\vec z\ge 0
\end{aligned}$$

We can use the tableau method to solve this:
\begin{align*}
\begin{array}{ccccccc|c}
    x_1&x_2&x_3&x_4&s_1&s_2&t\\\hline
    1&2&1&0&1&0&0&7\\
    -2&1&1&\circled3&0&-1&1&1\\\hline
    2&-1&-1&-3&0&1&0&-1\\
    1&1&-1&-1&0&0&0&0
\end{array}
&\ro{\substack{r_3+r_2\\r_3/3}}
\begin{array}{ccccccc|c}
    x_1&x_2&x_3&x_4&s_1&s_2&t\\\hline
    1&2&1&0&1&0&0&7\\
    -\sfrac{2}{3}&\sfrac{1}{3}&\sfrac{1}{3}&1&0&-\sfrac{1}{3}&\sfrac{1}{3}&\sfrac{1}{3}\\\hline
    0&0&0&0&0&0&1&0\\
    -\sfrac{1}{3}&-\sfrac{4}{3}&\sfrac{2}{3}&0&0&\sfrac{1}{3}&0&-\sfrac{1}{3}
\end{array}
\end{align*}

At this point we have a positive BFS to the auxillary problem given by $\begin{bmatrix}
    0&0&0&\sfrac{1}{3}&7&0
\end{bmatrix}^\top$ And since its last entry (corresponding to the artificial variable $t$) is 0, we also have a BFS for the original canonical problem: $\begin{bmatrix}
    0&0&0&\sfrac{1}{3}&7
\end{bmatrix}^\top$

We can now finally perform phase 2 of the simplex method by adjusting our last tableau and solving:
\begin{align*}
\begin{array}{cccccc|c}
    x_1&x_2&x_3&x_4&s_1&s_2\\\hline
    1&2&1&0&1&0&7\\
    -\sfrac{2}{3}&\circled{\sfrac{1}{3}}&\sfrac{1}{3}&1&0&-\sfrac{1}{3}&\sfrac{1}{3}\\\hline
    -\sfrac{1}{3}&-\sfrac{4}{3}&\sfrac{2}{3}&0&0&\sfrac{1}{3}&-\sfrac{1}{3}
\end{array}
&\ro{\substack{r_3+4r_2\\3r_2\\r_1-2r_2}}
\begin{array}{cccccc|c}
    x_1&x_2&x_3&x_4&s_1&s_2\\\hline
    \circled5&0&-1&-6&1&2&5\\
    -2&1&1&3&0&-1&1\\\hline
    -3&0&2&4&0&-1&1
\end{array}
&\ro{\substack{r_3+4r_2\\3r_2\\r_1-2r_2}}
\begin{array}{cccccc|c}
    x_1&x_2&x_3&x_4&s_1&s_2\\\hline
    1&0&-\sfrac{1}{5}&-\sfrac{6}{5}&\sfrac{1}{5}&\sfrac{2}{5}&1\\
    0&1&\sfrac{3}{5}&\sfrac{3}{5}&\sfrac{2}{5}&-\sfrac{1}{5}&3\\\hline
    0&0&\sfrac{7}{5}&\sfrac{2}{5}&\sfrac{3}{5}&\sfrac{1}{5}&4
\end{array}
\end{align*}

With the bottom row nonnegative, we have completed the simplex algorithm. The maximum is 4, and the $\vec z$ that achieves it is given by:
$$\vec z=\begin{bmatrix}
    x_1\\x_2\\x_3\\x_4\\s_1\\s_2
\end{bmatrix}=\begin{bmatrix}
    1\\3\\0\\0\\0\\0
\end{bmatrix}$$

\subsection*{Chapter 8}
\noindent\textbf{Problem 1:} Write a constraint that prevents two movies from being on the same screen at the same time.
\bigskip

\noindent\textbf{Solution:} Consider the following:
\begin{align*}
    x_{t,m,s}=1\implies\sum_{\tau=t}^{t+r_m-1}\sum_{\mu=1}^M x_{\tau,\mu,s}=1&\iff x_{t,m,s}-2<0\implies \sum_{\tau=t}^{t+r_m-1}\sum_{\mu=1}^M x_{\tau,\mu,s}\le 1\\
    &\iff x_{t,m,s}-2<0\implies 1-\sum_{\tau=t}^{t+r_m-1}\sum_{\mu=1}^M x_{\tau,\mu,s}\ge 0\\
    &\iff\begin{cases}
        z\in\{0,1\}\\
        Nz\ge x_{t,m,s}-2\\
        x_{t,m,s}-2+(1-z)N\ge 0\\
        1-\sum_{\tau=t}^{t+r_m-1}\sum_{\mu=1}^M x_{\tau,\mu,s}\ge -zN
    \end{cases}
\end{align*}

We let our bound $N=MT$ as neither variable can exceed it.
\bigskip

\noindent\textbf{Problem 2:} Write a constraint that prevents any movie from running after the theater closes.
\bigskip

\noindent\textbf{Solution:} Consider the following:
\begin{align*}
    x_{t,m,s}=1\implies t+r_m\le T&\iff x_{t,m,s}-2<0\implies t+r_m\le T\\
    &\iff x_{t,m,s}-2<0\implies T-t-r_m\ge 0\\
    &\iff\begin{cases}
        z\in\{0,1\}\\
        Nz\ge x_{t,m,s}-2\\
        x_{t,m,s}-2+(1-z)N\ge 0\\
        T-t-r_m\ge -zN
    \end{cases}
\end{align*}

We let our bound $N=1$ as neither variable can exceed it.
\bigskip

\noindent\textbf{Problem 3:} Write a constraint that in any 8 consecutive time blocks a movie must start somewhere in the theater.
\bigskip

\noindent\textbf{Solution:} The constraint is given by:
\begin{align*}
    t\le T-8\implies\sum_{\tau=t}^{t+8}x_{t,m,s}\ge 1&\iff t-T-8\le0\implies\sum_{\tau=t}^{t+8}x_{t,m,s}\ge 1\\
    &\iff t-T-9<0\implies\sum_{\tau=t}^{t+8}x_{t,m,s}-1\ge 0\\
    &\iff\begin{cases}
        z\in\{0,1\}\\
        Nz\ge t-T-9\\
        t-T-9+(1-z)N\ge 0\\
        \sum_{\tau=t}^{t+8}x_{t,m,s}-1\ge -zN
    \end{cases}
\end{align*}

We let our bound $N=7$ as both variables cannot exceed it.
\bigskip

\noindent\textbf{Problem 4:} The constraint from problem 3 turns out to cause enormous trouble; prove if you do not have constraints such as this that a feasible movie schedule exists. Must a feasible schedule exist with this constraint?
\bigskip

\noindent\textbf{Solution:} Consider the following counterexample to the constraint from problem 3. If the movie theater has only one screen and if all the running times of the movies $r_m$ exceed 8 blocks, then clearly the screen will not be able to play another movie in time after the first. In other words, a feasible schedule does \textit{not} necessarily exist with the constraint from problem 3.

If we remove this constraint, then a feasible solution \textit{does} exist, namely playing no movies. Indeed, playing no movies vacuously satisfies the ``no movies playing simultaneously'' condition as well as the ``no movies playing after closing'' condition from problems 1 \& 2.

\end{document}
\documentclass{article}
\usepackage{amsmath,mathtools}
\usepackage{amssymb}
\usepackage[dvipsnames]{xcolor}
\usepackage{graphicx}
\usepackage{xargs}
\usepackage{enumitem}
\usepackage{systeme}
\usepackage{centernot}
\usepackage{physics}
\usepackage{titling}
\usepackage[margin=1in]{geometry}
\usepackage[skins,theorems]{tcolorbox}
\tcbset{highlight math style={enhanced,
  colframe=blue,colback=white,arc=0pt,boxrule=1pt}}

\renewcommand{\eval}[3]{\left[#1\right]_{#2}^{#3}}
\newcommandx{\der}[2][1= , 2=t]{\frac{d#1}{d#2}}
\renewcommand\vec{\mathbf}

\newenvironment{sysmatrix}[1]
{\left[\begin{array}{@{}#1@{}}}
{\end{array}\right]}
\newcommand{\ro}[1]{%
\xrightarrow{\mathmakebox[\rowidth]{#1}}%
}
\newlength{\rowidth}% row operation width
\AtBeginDocument{\setlength{\rowidth}{3em}}

%set theory commands
\newcommand{\posint}{\mathbb J}
\newcommand{\pset}[1]{\mathcal P(#1)}
\newcommand{\card}[1]{\operatorname{card}(#1)}
% \newcommand{\posint}{\mathbb Z^+}


\setlength{\droptitle}{-7em}   % This is your set screw

\begin{document}

\title{Intro to Real Analysis\\HW \#1}
\author{Ozaner Hansha}
\date{January 30, 2021}
\maketitle

Let $\posint$ be the set of positive integers.

\subsection*{Problem 1}
\noindent\textbf{Problem:} Prove the following:
$$\forall n\in\posint,\,4\cdot3^n>(n+2)^2$$

\noindent\textbf{Solution:} Consider the predicate $P(n)\equiv4\cdot3^n>(n+2)^2$. We can prove $P(1)$ quite simply:
\begin{align*}
    P(1)&\equiv4\cdot3>(1+2)^2\\
    &\equiv12>9\\
    &\equiv\top
\end{align*}

Now let us consider the following lemma $\forall n\in\posint$:
\begin{align*}
    0&<2n^2+6n+3\tag{$n>0\implies 2n^2+6n+3>0$}\\
    n^2+6n+9&<3n^2+12n+12\\
    (n+3)^2&<3(n+2)^2\tag{Lemma 1}\\
\end{align*}

Now we will prove that $P(n)\implies P(n+1)$:
\begin{align*}
    4\cdot3^{n+1}&=3(4\cdot3^n)\tag{additive prop. of exponent}\\
    &>3(n+2)^2\tag{assume the antecedent, $P(n)$}\\
    &>(n+3)^2\tag{Lemma 1}\\
    4\cdot3^{n+1}&>(n+1+2)^2\tag{$P(n+1)$}
\end{align*}

And so we have shown $P(1)$ and $P(n)\implies P(n+1)$. By the principle of mathematical induction, this implies that $P(n)$ holds for all integers greater than or equal to 1, in other words:
$$\forall n\in\posint,\,4\cdot3^n>(n+2)^2$$

\subsection*{Problem 2}
\noindent\textbf{Problem:} For an integer $k$, let $k\posint$ be the set of positive integers which are multiples of $k$. Prove that $\card{2\posint}=\card{3\posint}$.
\bigskip

\noindent\textbf{Solution:} Recall that for two sets to have the same cardinality there must exist a bijection between them. Consider the function $f:2\posint\to3\posint$ defined by $n\to\frac{3n}{2}$. We will first prove that $f$ is injective. Consider arbitrary $x,y\in2\posint$:
\begin{align*}
    f(x)=f(y)&\implies\frac{3x}{2}=\frac{3y}{2}\tag{def. of $f$}\\
    &\implies x=y\tag{algebra}
\end{align*}

And so we have shown that $f(x)=f(y)\implies x=y$, which is the definition of an injective function.

Next we will show $f$ is surjective. Consider an arbitrary $y\in3\posint$. There exists an $x$ such that $f(x)=y$, namely $x=\frac{2y}{3}$. We prove this below:
\begin{align*}
    f(x)&=f\left(\frac{2y}{3}\right)\tag{def. of $x$}\\
    &=\frac{3}{2}\cdot\frac{2y}{3}\tag{def. of $f$}\\
    &=y
\end{align*}

And so we have shown that $(\forall y\in3\posint)\,(\exists x\in2\posint)\,f(x)=y$, which is the definition of a surjective function.

Since a function that is both injective and surjective is a bijection, we have shown that $f$ is bijective and thus, by the definition of cardinality, that:
$$\card{2\posint}=\card{3\posint}$$

\subsection*{Problem 3}
Consider the set of polynomials with integer coefficients $\mathbb Z[x]$. Given two polynomials $f,g\in\mathbb Z[x]$, we define the relation $\sim$ as:
\begin{equation*}
    f\sim g\equiv \exists h\in\mathbb Z[x],\,f-g=h'
\end{equation*}

\noindent\textbf{Part a:} Prove that $f\sim f$.
\bigskip

\noindent\textbf{Solution:} Consider an arbitrary polynomial $f\in\mathbb Z[x]$. We have $f-f=0$ and since there exists an $h\in\mathbb Z[x]$ such that $h'=0$, in particular any integer constant $C$, we have that $f\sim f$.
\bigskip

\noindent\textbf{Part b:} Prove that $f\sim g\implies g\sim f$.
\bigskip

\noindent\textbf{Solution:} If for two functions $f,g\in\mathbb Z$ we have $f\sim g$ this means that:
$$\exists h\in\mathbb Z[x],\,\,f-g=h'$$

Now recall that for any function $a$, the derivative of the negative is equal to the negative of the derivative:
$$(-a)'=-a'$$

Also recall that for any function $a\in\mathbb Z[x]$ we also have $-a\in\mathbb Z[x]$ since the integers (i.e. its coefficients) are closed under negation. Putting these together we have that:
$$g-f=-h'$$

And since $-h'=(-h)'$ and $-h\in\mathbb Z[x]$, since $h\in\mathbb Z[x]$, we have that $g\sim f$.
\bigskip

\noindent\textbf{Part c:} Prove that $f\sim g\wedge g\sim h\implies f\sim h$.
\bigskip

\noindent\textbf{Solution:} If for functions $f,g,h\in\mathbb Z$ we have $f\sim g\wedge g\sim h$ this means that:
\begin{align*}
    \exists a\in\mathbb Z[x],\,\,f-g=a'\\
    \exists b\in\mathbb Z[x],\,\,g-h=b'
\end{align*}

Now recall that for any functions $a,b$, the derivative of their sum is the sum of their derivatives:
$$(a+b)'=a'+b'$$

Also recall that for any functions $a,b\in\mathbb Z[x]$ we also have $a+b\in\mathbb Z[x]$ since the integers (i.e. their coefficients) are closed under addition. Putting these together we have that:
$$f-h=(f-g)+(g-h)=a'+b'$$

And since $a'+b'=(a+b)'$ and $a+b\in\mathbb Z[x]$, since $a,b\in\mathbb Z[x]$, we have that $f\sim h$.
\bigskip

\noindent\textbf{Part d:} For any $f,g\in\mathbb Z[x]$, is $f\sim g$?
\bigskip

\noindent\textbf{Solution:} No. To see this consider $f=2x$ and $g=x$:
$$2x-x=x$$

For $2x\sim x$ it must be the case that their difference $x$ be the derivative of some function $h\in\mathbb Z[x]$. Let us solve for all such possible $h$:
\begin{align*}
    \dv{h}{x}&=x\\
    h_C&=\frac{x^2}{2}+C
\end{align*}

Notice that no solution $h_C$ is contained in $\mathbb Z[x]$ since the squared term of the solutions have a coefficient of $\frac{1}{2}\not\in\mathbb Z$. Thus we have shown by counterexample that:
$$\neg\forall f,g\in\mathbb Z[x],\,\,f\sim g$$

\subsection*{Problem 4}
\noindent\textbf{Problem:} Prove Bernoulli's inequality. That is, prove the following:
\begin{equation*}
    \forall m,n\in\posint,\,\,(1+m)^n\ge1+mn
\end{equation*}

\noindent\textbf{Solution:} Consider an arbitrary $m\in\posint$, we have the following predicate:
$$P(n)\equiv(1+m)^n\ge1+mn$$

We can prove $P(1)$ immediately:
\begin{align*}
    P(1)&\equiv1+m\ge1+m\\
    &\equiv\top
\end{align*}

Now we will prove that $P(n)\implies P(n+1)$:
\begin{align*}
    (1+m)^{n+1}&=(1+m)(1+m)^n\tag{additive prop. of exponent}\\
    &>(1+m)(1+mn)\tag{assume the antecedent, $P(n)$}\\
    &=1+mn+m^2n+m\tag{algebra}\\
    &\ge1+mn+m\tag{$m,n\in\posint\implies m^2n>0$}\\
    (1+m)^{n+1}&\ge1+m(n+1)\tag{$P(n+1)$}
\end{align*}

And so we have shown $P(1)$ and $P(n)\implies P(n+1)$. By the principle of mathematical induction, this implies that $P(n)$ holds for all integers greater than or equal to 1, in other words:
$$\forall m,n\in\posint,\,\,(1+m)^n\ge1+mn$$


\subsection*{Problem 5}
\noindent\textbf{Problem:} Prove that $P(\posint)$ is not countable.
\bigskip

\noindent\textbf{Solution:} Before we prove the desired statement, let us establish Cantor's theorem. First, consider a set $A$ and its power set $\pset A$. The function $g:A\to\pset A$ defined by $x\mapsto\{x\}$ is clearly an injunction from $A\to\pset A$. Since an injunction exits between these two sets we have, by the definition of cardinality:
\begin{equation}
    \card A\le\card{\pset A}\tag{Lemma 1}
\end{equation}
\medskip

Next, consider an arbitrary function $f:A\to\pset A$, and the set $B=\{x\in A\mid x\not\in f(x)\}$. Suppose $f$ is surjective. This means that $\forall y\in \pset A$ there exists an $x\in A$ such that $f(x)=y$, in particular it implies that:

\begin{equation}
    \exists c\in A,\,\,f(c)=B\tag{$f$ is surjective \& $B\in\pset A$}
\end{equation}
\medskip

But now we have a contradiction:
\begin{align*}
    c\in f(c)\iff c\in B\tag{$f(c)=B$ \& def. of set equality}\\
    c\in f(c)\iff c\not\in B\tag{def. of B}
\end{align*}

And so our assumption that $f$ was surjective is false, and thus no map from a set $A$ to its powerset $\pset A$ can be surjective. By the definition of cardinality this means:
\begin{equation}
    \card A\not\ge\card{\pset A}\tag{Lemma 2}
\end{equation}
\medskip

Putting these two lemmas together we have that for an arbitrary set $A$:
\begin{align*}
    \card A\le\card{\pset A}\tag{Lemma 1}\\
    \card A\not\ge\card{\pset A}\tag{Lemma 2}\\
    \hline
    \card A<\card{\pset A}\tag{Cantor's Theorem}
\end{align*}

Or in other words, the cardinality of a set is strictly smaller than that of its powerset. In the case of $\posint$ this implies:
$$\card\posint<\card{\pset\posint}$$

And since, by definition, any cardinality larger than countably infinite (i.e. $\card\posint=\aleph_0$) is uncountably infinite, we are done.

\end{document}
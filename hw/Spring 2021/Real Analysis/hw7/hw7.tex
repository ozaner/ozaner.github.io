\documentclass{article}
\usepackage{amsmath,mathtools}
\usepackage{amssymb}
\usepackage[dvipsnames]{xcolor}
\usepackage{graphicx}
\usepackage{xargs}
\usepackage{enumitem}
\usepackage{systeme}
\usepackage{centernot}
\usepackage{physics}
\usepackage{xfrac}
\usepackage{titling}
\usepackage[margin=1in]{geometry}
\usepackage[skins,theorems]{tcolorbox}
\tcbset{highlight math style={enhanced,
  colframe=blue,colback=white,arc=0pt,boxrule=1pt}}

% calculus commands
\renewcommand{\eval}[3]{\left[#1\right]_{#2}^{#3}}

% linear algebra commands
\renewcommand\vec{\mathbf}
\newenvironment{sysmatrix}[1]
{\left[\begin{array}{@{}#1@{}}}
{\end{array}\right]}
\newcommand{\ro}[1]{%
\xrightarrow{\mathmakebox[\rowidth]{#1}}%
}
\newlength{\rowidth}% row operation width
\AtBeginDocument{\setlength{\rowidth}{3em}}

%set theory commands
\newcommand{\pset}[1]{\mathcal P(#1)}
\newcommand{\card}[1]{\operatorname{card}(#1)}
\newcommand{\R}{\mathbb R}
\newcommand{\Q}{\mathbb Q}
\newcommand{\Z}{\mathbb Z}
\newcommand{\N}{\mathbb N}

%number theory commands
\newcommand{\divides}{\mid}
\newcommand{\ndivides}{\nmid}
\newcommand{\fallfact}[2]{#1^{\underline {#2}}}

%optimization commands
\DeclareMathOperator*{\argmax}{arg\,max}
\DeclareMathOperator*{\argmin}{arg\,min}

\setlength{\droptitle}{-7em}   % This is your set screw

\begin{document}

\title{Intro to Real Analysis\\HW \#7}
\author{Ozaner Hansha}
\date{April 2, 2021}
\maketitle

\subsection*{Problem 1}
\noindent\textbf{Solution:} Consider an arbitrary $x_0\in[a,b]$. Now consider an arbitrary $\epsilon>0$. Since $h(x)\in\{f(x),g(x)\}$, we have 2 cases:
\begin{enumerate}
  \item $h(x_0)=f(x_0)$:
  \begin{align*}
    \exists\delta_f,\quad 0<|x-x_0|<\delta_f&\implies|f(x)-f(x_0)|<\epsilon\tag{$f(x)$ is continuous}\\
    &\implies|h(x)-h(x_0)|<\epsilon\tag{$h(x_0)=f(x_0)$}
  \end{align*}
  \item $h(x_0)=g(x_0)$:
  \begin{align*}
    \exists\delta_f,\quad 0<|x-x_0|<\delta_f&\implies|g(x)-g(x_0)|<\epsilon\tag{$g(x)$ is continuous}\\
    &\implies|h(x)-h(x_0)|<\epsilon\tag{$h(x_0)=g(x_0)$}
  \end{align*}
\end{enumerate}

In both cases, $h(x)$ has been shown to be continuous at $x=x_0$. You'll notice that in the case that $h(x_0)=f(x_0)=g(x_0)$, both cases apply and we can simply pick one.

Since we have shown that $h(x)$ is continuous on any $x_0\in[a,b]$, we have shown that it is continuous over that whole interval.

\subsection*{Problem 2}
\noindent\textbf{Solution:} First let us prove that $\sqrt[3]{4}$ is irrational by contradiction. Let us assume that it is indeed rational, and thus there are coprime integers $a,b\in\Z^+$ such that $\frac{a}{b}=\sqrt[3]{4}$. This implies:
\begin{align*}
  \frac{a}{b}=4^{\sfrac{1}{3}}&\implies a=4^{\sfrac{1}{3}}b\\
  &\implies a^3=4b^3\\
  &\implies 4|a^3\tag{$a$ is an integer}\\
  &\implies 2|a^3\tag{$2|4$}\\
  &\implies 2|a\tag{2 is prime}\\
  &\implies a=2k\tag{$a$ is even}
\end{align*}

Plugging this in to our original expression, this implies that:
\begin{align*}
  4b^3=8k^3&\implies b^3=4k^3\tag{$a=2k$}\\
  &\implies 4|b^3\\
  &\implies 2|b^3\tag{$2|4$}\\
  &\implies 2|b\tag{2 is prime}\\
  &\implies b=2k\tag{$b$ is even}
\end{align*}

Yet this is a contradiction as now $a$ and $b$ are clearly not coprime, since they are both even.

\subsection*{Problem 3}
\noindent\textbf{Solution:} Yes. Let us first establish the following lemma:
\begin{align*}
  e^x&\ge x+1\\
  e^{x-1}&\ge x\\
  x-1&\ge\ln x\tag{lemma 1}
\end{align*}

Now on to the proof. Define $f(0)=0$ Now consider an arbitrary $\epsilon>0$, and $\delta=\epsilon+1$. We then have for $x_0=0$:
\begin{align*}
  |f(x)|&=|x\ln x|\\
  &=x\ln x\tag{$x>0$}\\
  &\le x(x-1)\tag{lemma 1}\\
  &<x-1\\
  &<\delta-1\tag{assume $0<|x|<\delta$}\\
  &<\epsilon\tag{$\delta=\epsilon+1$}
\end{align*}

And since we have only considered $x>0$ as $x\ln x$ isn't defined elsewhere, we have:
$$\lim_{x\to0^+}f(0)=\lim_{x\to0^+}x\ln x=0$$

And we are done.

\subsection*{Problem 4}
\noindent\textbf{Solution:} First let us show that a continuous function $f:[a,b]\to\R$ must also be uniformly continuous. We do this by contradiction. Note that if $G$ was not uniformly continuous, then there exists an $\epsilon>0$ such that for each $\delta>0$, there exists $x,c$ such that $|x-c|<\delta$ and $|f(x)-f(c)|\ge \epsilon$. By the archimedean property this implies that there is some $n\in\N$ such that this holds for some $\delta=\frac{1}{n}$.

This gives us two sequences:
\begin{gather*}
  (x_n)\subseteq[a,b]\\
  (c_n)\subseteq[a,b]
\end{gather*}

such that $|x_n-c_n|<\frac{1}{n}$ and $|f(x_n)-f(c_n)|\ge \epsilon$. But the Bolzano-Weierstrass theorem tells us that $(c_n)$ must itself have a convergent subsequence $(c_{n_k})$. Say this subsequence converges to $c$. Since $|x_{n_k}-c_{n_k}|<\frac{1}{n_k}$, we must have that the subsequence $(x_{n_k})$ converges to $c$ as well. But since $f$ is continuous we have:
\begin{gather*}
  \lim(f(x_{n_k})-f(c))=0\\
  \lim(f(c_{n_k})-f(c))=0
\end{gather*}

This tell us:
\begin{align*}
  |f(c_{n_k})-f(x_{n_k})|&=|f(c_{n_k})-f(c)+f(c)-f(x_{n_k})|\\
  &\le |f(c_{n_k})-f(c)|+|f(c)-f(x_{n_k})|\tag{triangle inequality}
\end{align*}

And so we have $f(c_{n_k})-f(x_{n_k})=0$. But this contradicts our assumption that $|f(c_n)-f(x_n)|\ge \epsilon$.
\bigskip

With this established, consider a restriction of our function $f_0:[0,h]\to\R$. Since $f$ is continuous everywhere, it must be that $f_0$ is continuous over $[0,h]$ and thus is bounded over said interval. This means that for any $\epsilon>0$ there is some $\delta>0$ that satisfies the continuity definition of any $x,x_0\in[0,h]$.

Note that the same holds for $f_1:[h,2h]\to\R$ and in general $f_k:[kh,(k+1)h]\to\R$. Not only that but since these restrictions are identical but just shifted (since $f$ is periodic), for any $\epsilon>0$ the same $\delta>0$ can be used for each of these restrictions. And since we have:
$$f(x)=\begin{cases}
  f_0(x),&x\in[0,h]\\
  &\vdots\\
  f_k(x),&x\in[kh,(k+1)h]\\
  &\vdots
\end{cases}$$

Then we have that our choice of $\delta$ given $\epsilon$ is independent of $x,x_0$ over the whole domain of $\R$ for the complete function $f$. In other words, $f$ is uniformly continuous.

\subsection*{Problem 5}
\noindent\textbf{Solution:} Because $f,g$ are uniformly continuous we have:
\begin{gather}
  \forall\epsilon>0,\exists\delta_1,\forall x,x_0\in (a,b),\quad |x-x_0|<\delta_1\implies |f(x)-f(x_0)|<\epsilon<\frac{\epsilon}{2}\\
  \forall\epsilon>0,\exists\delta_2,\forall x,x_0\in (a,b),\quad |x-x_0|<\delta_1\implies |g(x)-g(x_0)|<\epsilon<\frac{\epsilon}{2}\\
\end{gather}

And also note that by the triangle inequality:
$$|(f+g)(x)-(f+g)(x_0)|\le |f(x)-f(x_0)|+|g(x)-g(x_0)|$$

And so letting $\delta_0=\min\{\delta_1,\delta_2\}$ we have:
$$\forall\epsilon>0,\exists\delta_0,\forall x,x_0\in (a,b),\quad |x-x_0|,\delta_0\implies |(f+g)(x)-(f+g)(x_0)|\le |f(x)-f(x_0)|+|g(x)-g(x_0)|<\frac{\epsilon}{2}+\frac{\epsilon}{2}=\epsilon$$

And so uniform continuity is preserved by addition.
\bigskip

For multiplication, this is not true for an open interval $(a,b)$.
\end{document}
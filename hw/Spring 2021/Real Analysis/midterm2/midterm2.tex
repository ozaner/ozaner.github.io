\documentclass{article}
\usepackage{amsmath,mathtools}
\usepackage{amssymb}
\usepackage[dvipsnames]{xcolor}
\usepackage{graphicx}
\usepackage{xargs}
\usepackage{enumitem}
\usepackage{systeme}
\usepackage{centernot}
\usepackage{physics}
\usepackage{xfrac}
\usepackage{titling}
\usepackage[margin=1in]{geometry}
\usepackage[skins,theorems]{tcolorbox}
\tcbset{highlight math style={enhanced,
  colframe=blue,colback=white,arc=0pt,boxrule=1pt}}

% calculus commands
\renewcommand{\eval}[3]{\left[#1\right]_{#2}^{#3}}

% linear algebra commands
\renewcommand\vec{\mathbf}
\newenvironment{sysmatrix}[1]
{\left[\begin{array}{@{}#1@{}}}
{\end{array}\right]}
\newcommand{\ro}[1]{%
\xrightarrow{\mathmakebox[\rowidth]{#1}}%
}
\newlength{\rowidth}% row operation width
\AtBeginDocument{\setlength{\rowidth}{3em}}

%set theory commands
\newcommand{\pset}[1]{\mathcal P(#1)}
\newcommand{\card}[1]{\operatorname{card}(#1)}
\newcommand{\R}{\mathbb R}
\newcommand{\Q}{\mathbb Q}
\newcommand{\Z}{\mathbb Z}
\newcommand{\N}{\mathbb N}

%number theory commands
\newcommand{\divides}{\mid}
\newcommand{\ndivides}{\nmid}
\newcommand{\fallfact}[2]{#1^{\underline {#2}}}

%optimization commands
\DeclareMathOperator*{\argmax}{arg\,max}
\DeclareMathOperator*{\argmin}{arg\,min}

\setlength{\droptitle}{-7em}   % This is your set screw

\begin{document}

\title{Intro to Real Analysis\\Midterm 2}
\author{Ozaner Hansha}
\date{March 31, 2021}
\maketitle

\subsection*{Problem 1}
\noindent\textbf{Part a:} Consider an arbitrary $\epsilon>0$. And a $\delta=\frac{1}{\epsilon}>0$. We then have $\forall x\in\R^+$:
\begin{align*}
  x>\delta&\implies\frac{1}{x}<\frac{1}{\delta}\tag{$1/x$ is strictly decreasing over $\R^+$}\\
  &\implies\left|\frac{1}{x}\right|<\frac{1}{\delta}\tag{$x>0\implies 1/x>0$}\\
  &\implies|\sin(x)|\left|\frac{1}{x}\right|<\frac{1}{\delta}\tag{$|\sin(x)|\le 1$}\\
  &\implies\left|\frac{\sin(x)}{x}\right|<\frac{1}{\delta}\\
  &\implies\left|\frac{\sin(x)}{x}\right|<\epsilon\tag{def. of $\delta$}
\end{align*}

And this is precisely the definition of:
$$\forall x\in\R^+,\quad\lim_{x\to\infty}\frac{\sin(x)}{x}=0$$
\bigskip

\noindent\textbf{Part b:} Consider the sequence $A=\{1/n\}_{n=1}^\infty$. You'll note that:
$$\forall n\in\Z^+,\quad 0<\frac{1}{n}$$

And so $A$ is a subsequence of our interval $(0,\infty)$, since all $a_n$ are contained within it. Now all that is left is to show that $A$ converges to 0. Consider an arbitrary $\epsilon>0$, and let be an integer $N$ such that $\frac{1}{N}<\epsilon$ (this is guaranteed to us by the archemdian property). We then have:
\begin{align*}
  n\ge N&\implies\frac{1}{n}\le\frac{1}{N}\tag{$1/x$ is strictly decreasing over $\R^+$}\\
  &\implies\frac{1}{n}<\epsilon\tag{def. of $N$}\\
  &\implies\left|\frac{1}{n}\right|<\epsilon\tag{$n>0\implies 1/n>0$}\\
  &\implies\left|\frac{1}{n}-0\right|<\epsilon
\end{align*}

And so we are done. We have shown that a subsequence of $(0,\infty)$ converges to 0, and thus it is an accumulation point of said interval.
\bigskip

\noindent\textbf{Part c:} Note the following:
\begin{align*}
  \lim_{x\to0}\frac{\sin(2x)}{3x(x-3)}&=\lim_{x\to0}\frac{2\sin x\cos x}{3x(x-3)}\tag{double angle formula}\\
  &=\lim_{x\to0}\frac{2}{3}\cdot\frac{\sin x}{x}\cdot\frac{\cos x}{x-3}\\
  &=\frac{2}{3}\lim_{x\to0}\frac{\sin x}{x}\lim_{x\to0}\frac{\cos x}{x-3}\tag{product of limits is limit of products}\\
  &=\frac{2}{3}\lim_{x\to0}\frac{\cos x}{x-3}\tag{limit given}\\
  &=\frac{2}{3}\cdot\frac{\cos 0}{0-3}\tag{$\cos$ is contious at 0}\\
  &=\frac{2}{3}\cdot-\frac{1}{3}=-\frac{2}{9}
\end{align*}
\bigskip

\subsection*{Problem 2}
\noindent\textbf{Problem:} First note that both pieces of this function are continuous on their own. And so, the only points in which $f(x)$ can be continuous are where they coincide:
\begin{align*}
  x^2+1&=3-x^2\\
  2x^2&=2\\
  x^2&=1\\
  x&=\pm1
\end{align*}

And so we have that:
$$\lim{x\to\pm1}=f(\pm1)$$

Every other point, i.e. the $x$ in which the two functions \textit{don't} coincide, are discontinuous. This is because there is no interval of non-zero size that does not contain a rational number. And since we are considering the points $x$ in which the functions do not coincide, $f(x)$ cannot be continuous on such an $x$ as the two functions approach different values. An example of this is at $x=5$.
\bigskip

\subsection*{Problem 3}
\noindent\textbf{Problem:} Consider $x,y\in[a,b]$ such that $x<y$. Note that:
\begin{align*}
  x<y&\implies[a,x]\subseteq[a,y]\\
  &\implies\sup\{f(t)\mid t\in[a,x]\}\le\sup\{f(t)\mid t\in[a,y]\}\\
  &\implies g(x)\le g(y)\tag{def. of $g$}
\end{align*}

With the second implication holding because the supremum of a subset can be no larger than the supremum of its superset.

Now consider and arbitrary $x_0\in[a,b]$. Since $f(x)$ is continuous, we have that for any $\epsilon>0$, there exists a $\delta>0$ such that $\forall x,x_0\in[a,b]$:
\begin{align*}
  0<|x-x_0|<\delta&\implies |f(x)-f(x_0)|<\epsilon\\
  &\implies|g(x)-g(x_0)|<|f(x)-f(x_0)|<\epsilon\\
  &\implies|g(x)-g(x_0)|<\epsilon
\end{align*}

Which is precisely the definition of $g(x)$ being continuous over $[a,b]$.

% Now note that for any $x$ we have:
% \begin{align*}
%   \{x\}\subseteq[a,x]&\implies\sup\{f(x)\}\le\sup\{f(t)\mid t\in[a,x]\}\tag{same reasoning as above}\\
%   &\implies f(x)\le g(x)\tag{def. of $g(x)$, \& $\sup\{a\}=a$}
% \end{align*}


\bigskip

\subsection*{Problem 4}
\noindent\textbf{Part a:} Let us expand both terms in the numerator via the binomial theorem:
\begin{align*}
  \frac{(1+mx)^n-(1+nx)^m}{x^2}&=\frac{\left(1+nmx+\binom{n}{2}m^2x^2+\cdots\right)-\left(1+mnx+\binom{m}{2}n^2x^2+\cdots\right)}{x^2}\\
  &=\frac{\frac{nm(n-m)}{2}x^2+c_1x^3+c_2x^4+\cdots}{x^2}\\
  &=\frac{nm(n-m)}{2}+c_1x+c_2x^2+\cdots\\
\end{align*}

Where $c_k$ are some constants found by calculating out the binomial theorem. Now note that this is a polynomial, meaning it is continuous everywhere. As such we have:
\begin{align*}
  \lim_{x\to0}\frac{(1+mx)^n-(1+nx)^m}{x^2}&=\lim_{x\to0}\frac{nm(n-m)}{2}+c_1x+c_2x^2+\cdots\tag{see above}\\
  &=\frac{nm(n-m)}{2}+c_1\cdot0+c_2\cdot0^2+\cdots\tag{continous}\\
  &=\frac{nm(n-m)}{2}
\end{align*}
\bigskip

\noindent\textbf{Part b:} Note the following:
\begin{align*}
  \lim_{x\to1}\frac{x^m-1}{x^n-1}&=\lim_{x\to1}\frac{(x-1)(x^{m-1}+x^{m-2}+\cdots+x+1)}{(x-1)(x^{n-1}+x^{n-2}+\cdots+x+1)}\\
  &=\lim_{x\to1}\frac{x^{m-1}+x^{m-2}+\cdots+x+1}{x^{n-1}+x^{n-2}+\cdots+x+1}\\
  &=\frac{1^{m-1}+1^{m-2}+\cdots+1+1}{1^{n-1}+1^{n-2}+\cdots+1+1}\tag{$\substack{\text{rational functions are continous}\\\text{where they are defined}}$}\\
  &=\frac{m}{n}
\end{align*}
\bigskip

\noindent\textbf{Part c:} Note, as used in the last problem, the following identity:
$$t^N-1=(t-1)\sum^{N-1}_k=0t^k$$

By setting $x=t^{nm}$, and noticing that $x\to1\implies t^nm\to1\implies t\to1$, we have:
\begin{align*}
  \lim_{x\to1}\frac{\sqrt[m]{x}-1}{\sqrt[n]{x}-1}&=\lim_{t\to1}\frac{\sqrt[m]{t^{nm}}-1}{\sqrt[n]{t^{nm}}-1}\tag{def. of $t^nm$}\\
  &=\lim_{t\to1}\frac{t^n-1}{t^m-1}\\
  &=\frac{n}{m}\tag{problem b}
\end{align*}
\bigskip

\end{document}
\documentclass{article}
\usepackage{amsmath,mathtools}
\usepackage{amssymb}
\usepackage[dvipsnames]{xcolor}
\usepackage{graphicx}
\usepackage{xargs}
\usepackage{enumitem}
\usepackage{systeme}
\usepackage{centernot}
\usepackage{physics}
\usepackage{xfrac}
\usepackage{titling}
\usepackage[margin=1in]{geometry}
\usepackage[skins,theorems]{tcolorbox}
\tcbset{highlight math style={enhanced,
  colframe=blue,colback=white,arc=0pt,boxrule=1pt}}

% calculus commands
\renewcommand{\eval}[3]{\left[#1\right]_{#2}^{#3}}

% linear algebra commands
\renewcommand\vec{\mathbf}
\newenvironment{sysmatrix}[1]
{\left[\begin{array}{@{}#1@{}}}
{\end{array}\right]}
\newcommand{\ro}[1]{%
\xrightarrow{\mathmakebox[\rowidth]{#1}}%
}
\newlength{\rowidth}% row operation width
\AtBeginDocument{\setlength{\rowidth}{3em}}

%set theory commands
\newcommand{\pset}[1]{\mathcal P(#1)}
\newcommand{\card}[1]{\operatorname{card}(#1)}
\newcommand{\R}{\mathbb R}
\newcommand{\Q}{\mathbb Q}
\newcommand{\Z}{\mathbb Z}
\newcommand{\N}{\mathbb N}

%number theory commands
\newcommand{\divides}{\mid}
\newcommand{\ndivides}{\nmid}

%optimization commands
\DeclareMathOperator*{\argmax}{arg\,max}
\DeclareMathOperator*{\argmin}{arg\,min}

\setlength{\droptitle}{-7em}   % This is your set screw

\begin{document}

\title{Intro to Real Analysis\\HW \#2}
\author{Ozaner Hansha}
\date{February 6, 2021}
\maketitle

\subsection*{Problem 1}
\noindent\textbf{Problem:} Use Euclid's lemma to show that there are no rational numbers $x$ such that $x^3=2021$.
\bigskip

\noindent\textbf{Solution:} Before we prove this, we must first prove 3 lemmas.
\begin{itemize}
  \item \textbf{Lemma 1}: For $a_1,a_2,b_1,b_2\in\Z$ we have:
  \begin{align*}
    a_1\divides b_1\wedge a_2\divides b_2&\implies(\exists k_1,k_2\in Z),\,\,\frac{b_1}{a_1}=k_1\wedge\frac{b_2}{a_2}=k_2\tag{def. of divisibility}\\
    &\implies(\exists k_1,k_2\in Z),\,\,\frac{b_1b_2}{a_1a_2}=k_1k_2\\
    &\implies a_1a_2\divides b_1b_2\tag{$\Z$ closed under multiplication}
  \end{align*}
  % \item \textbf{Lemma 1}: For all prime numbers $p$ and integers $m$:
  % \begin{align*}
  %   p\divides m^2&\implies p\divides m\vee p\divides m\tag{Euclid's lemma}\\
  %   &\implies p\divides m\\
  %   &\implies p\divides m^2\tag{lemma 1, $1\divides m$}
  % \end{align*}
  % And so by a chain of implications we have: $p\divides m\iff p\divides m^2$.
  \item \textbf{Lemma 2}: For all prime numbers $p$ and integers $m$: 
  \begin{align*}
    p\divides m^3&\implies p\divides m^2\vee p\divides m\tag{Euclid's lemma}\\
    &\implies (p\divides m\vee p\divides m)\vee p\divides m\tag{Euclid's lemma}\\
    &\implies p\divides m\\
    &\implies p\divides m^2\tag{lemma 1, $1\divides m$}\\
    &\implies p\divides m^3\tag{lemma 1, $1\divides m$}
  \end{align*}
  And so by a chain of implications we have: $p\divides m\iff p\divides m^2\iff p\divides m^3$.
  % \item \textbf{Lemma 3}:
  % For all $n\in\Z^+$:
  % \begin{align*}
  %   a\divides b&\iff \exists c\in\Z,\,\frac{b}{a}=c\tag{def. of divisible}\\
  %   &\iff \exists c\in\Z,\,\frac{b^n}{a^n}=c^n\tag{$c^n\in\Z^+$, $\Z$ closed under repeated multiplication.}\\
  %   &\implies a^n\divides b^n
  % \end{align*}
  \item \textbf{Lemma 3}:
  For all $k\in\Z$:
  \begin{align*}
    a\divides b&\iff \exists c\in\Z,\,\frac{b}{a}=c\tag{def. of divisible}\\
    &\iff \exists c\in\Z,\,\frac{kb}{ka}=c\tag{common factor}\\
    &\implies ka\divides kb
  \end{align*}
\end{itemize}

We can now finally prove the desired statement:
\begin{align*}
  2021&=x^3\\
  &=\left(\frac{m}{n}\right)^3\text{ where $m,n$ are relatively prime}\tag{assume $x\in\Q$}\\
  &=\frac{m^3}{n^3}\\
  2021n^3&=m^3\\
  47\cdot43n^2&=m^3\tag{prime factorization of 2021}\\
  &\implies47\divides2021n^3\tag{47 is a factor of 2021}\\
  &\implies47\divides m^3\tag{$2021n^3=m^3$}\\
  &\implies\tcbhighmath[right=2mm,left=2mm,boxsep=-1mm,boxrule=0.4pt,colframe=blue,colback=blue!10!white]{47\divides m}\tag{lemma 2}\\
  &\implies47^2\divides m^2\tag{lemma 1, with $47\divides m$}\\
  &\implies47^2\divides m^3\tag{lemma 1, with $1\divides m$}\\
  &\implies47^2\divides 47\cdot43n^3\tag{$47\cdot43n^3=m^3$}\\
  &\implies47\divides 43n^3\tag{lemma 3, $k=47$}\\
  &\implies47\divides n^3\tag{Euclid's lemma, $47\ndivides43$}\\
  &\implies\tcbhighmath[right=2mm,left=2mm,boxsep=-1mm,boxrule=0.4pt,colframe=blue,colback=blue!10!white]{47\divides n}\tag{lemma 2}
\end{align*}

Taking notice of the lines highlighted in blue, we can see that in assuming that $x\in Q$ and thus equals $\sfrac{m}{n}$ with $n$ and $m$ relatively prime, we have found a contradiction. Namely that 47 divides both $n$ and $m$ despite them being relatively prime as previously mentioned. This means our assumption that $x$ was rational is wrong and thus we have proven by contradiction that there is no rational $x$ such that $x^3=2021$.

\subsection*{Problem 2}
\noindent\textbf{Problem:} Prove the following:
\begin{equation*}
  (\forall y\in\R,\,\exists x\in\R)\,\,x^3=y
\end{equation*}


\noindent\textbf{Solution:}
\medskip

\textbf{CASE 1:} Consider the case where $y>0$. Now, consider the following set:
\begin{equation*}
  S=\{s\in\R\mid s^3<y\}
\end{equation*}

This set is clearly bounded by $y$, and is nonempty (e.g. $-1<0<y$ thus $-1\in S$). These two facts, by the L.U.B property, imply that $S$ has a supremum, call it $x$:
\begin{equation*}
  \sup S = x
\end{equation*}

We will now prove, by contradiction, that $x^3=y$.
\begin{itemize}
  \item if $x^3<y$ then, by the density of the rationals, there must exist some $\epsilon\in(0,1)$ such that:
  $$x^3<(x+\epsilon)^3<y$$

  However, this implies that $\sup S=x$ is less than another element in $S$, namely $x+\epsilon$. This is a contradiction of the LUB property and thus our assumption was wrong and $x^3\not<y$.
  \item if $x^3>y$ then, by the density of the rationals, there must exist some $\epsilon\in(0,1)$ such that:
  $$x^3>(x+\epsilon)^3>y$$

  However, this implies that $\sup S=x$ is less than another element in $S$, namely $x+\epsilon$. This is a contradiction of the LUB property and thus our assumption was wrong and $x^3\not>y$.
\end{itemize}

Putting these two together we have, by the law of trichotomy, that $x^3=y$. And so every positive real $y$ has a cube root $x$ given by the supremum of the set $S$.
\medskip

\textbf{CASE 2:} Now consider the case where $y<0$. We know from case 1 that there exists a real number $x$ such that $x^3=-y$. Now consider the following:
\begin{equation*}
  (-x)^3=-x^3=-(-y)=y
\end{equation*}

And so, by case 1, every negative real $y$ has a cube root given by the $-x$ above.
\medskip

\textbf{CASE 3:} The only other case to consider is $y=0$. This is simple as $0^3=0=y$. And so 0 is its own cube root.
\medskip

\textbf{Proof:} Since all real numbers $y$ must either be 0, greater than 0, or less than 0, and since we proved that $y$ has a cube root in all those cases, we can now be sure that all real numbers have a cube root.

\subsection*{Problem 3}
\noindent\textbf{Problem:} Find a bijection from $(0,1)$ to $[-2021,2021]$.
\bigskip

\noindent\textbf{Solution:} First let us define a sequence $(y_n)_{n\in\N}$:
\begin{equation*}
  y_n=.1^{n+1}
\end{equation*}

Now let us define a bijective function $f_1:(0,1)\to[0,1]$:
\begin{equation*}
  f_1(x)=\begin{cases}
    0, &x=y_0\\
    1, &x=y_1\\
    y_{n+2},&x=y_n,\,n>1\\
    x,&\text{otherwise}
  \end{cases}
\end{equation*}

Finally we can define our bijection $f_2:(0,1)\to[-2021,2021]$:
\begin{equation*}
  f_2(x)=2021(2f_1(x)-1)
\end{equation*}

\subsection*{Problem 4}
\noindent\textbf{Problem:} Find two bounded sets $A$ and $B$ such that $A\cap B=\emptyset$ and $\sup A=\sup B$.
\bigskip

\noindent\textbf{Solution:} Consider the following sets:
\begin{align*}
  A&=\{0,1\}\\
  B&=\{1-.1^n\mid n\in\Z^+\}
\end{align*}

First note that $A\cap B=\emptyset$ as there is no $n\in\Z^+$ such that $1-.1^n$ equals 0 or 1. For 0 this is obvious. For 1, while $1-.1^0=1$, we still have that $0\not\in\Z+$ and so $1\not\in B$.

Now note that the supremum of $A$ is 1 as it has only two elements and $0<1$. For $B$ note that since it is bounded and has no single largest element, we can find its supremum with the following limit:
\begin{align*}
  \sup B&=\lim_{n\to\infty}1-.1^n\\
  &=1-\lim_{n\to\infty}.1^n\\
  &=1-0=1
\end{align*}

And so we have that $\sup A = \sup B = 1$, despite $A\cap B=\emptyset$.

\subsection*{Problem 5}
Consider a bounded set $A\subseteq R$.
\bigskip

\noindent\textbf{Problem a:} For $B=\{x+1\mid x\in A\}$, show that $\sup B=(\sup A)+1$.
\bigskip

\noindent\textbf{Solution:} Consider the following:
\begin{align*}
  \sup A &= \sup A\\
  (\forall r_1\in\R,\,\forall a\in A)&\,\,\sup A\ge a\wedge\neg(\sup A>r_1\ge a)\tag{def. of supremum}\\
  (\forall r_1\in\R,\,\forall a\in A)&\,\,\sup A+1\ge a+1\wedge\neg(\sup A+1>r_1+1\ge a+1)\tag{$x+1$ is an increasing function}\\
  (\forall r_2\in\R,\,\forall a\in A)&\,\,\sup A+1\ge a+1\wedge\neg(\sup A+1>r_2\ge a+1)\tag{$(\forall r_1\in\R,\,\exists r_2\in\R)\,\,r_2=r_1+1$}\\
  (\forall r_2\in\R,\,\forall b\in B)&\,\,\sup A+1\ge b\wedge\neg(\sup A+1>r_2\ge b)\tag{$a\in A\iff a+1\in B$ (i.e. def. of $B$)}\\
  \sup B &= \sup A + 1\tag{def. of supremum}
\end{align*}

And so we are done.
\bigskip

\noindent\textbf{Problem b:} For $C=\{x^3+1\mid x\in A\}$, show that $\sup C=(\sup A)^3+1$.
\bigskip

\noindent\textbf{Solution:}
\begin{align*}
  \sup A &= \sup A\\
  (\forall r_1\in\R,\,\forall a\in A)&\,\,\sup A\ge a\wedge\neg(\sup A>r_1\ge a)\tag{def. of supremum}\\
  (\forall r_1\in\R,\,\forall a\in A)&\,\,(\sup A)^3+1\ge a^3+1\wedge\neg((\sup A)^3+1>r_1^3+1\ge a^3+1)\tag{$x^3+1$ is an increasing function}\\
  (\forall r_2\in\R,\,\forall a\in A)&\,\,(\sup A)^3+1\ge a^3+1\wedge\neg((\sup A)^3+1>r_2\ge a^3+1)\tag{$(\forall r_1\in\R,\,\exists r_2\in\R)\,\,r_2=r_1^3+1$ (see problem 2)}\\
  (\forall r_2\in\R,\,\forall a\in A)&\,\,(\sup A)^3+1\ge b\wedge\neg((\sup A)^3+1>r_2\ge b)\tag{$a\in A\iff a^3+1\in B$ (i.e. def. of $B$)}\\
  \sup C &=(\sup A)^3 + 1\tag{def. of supremum}
\end{align*}

And so we are done.
\bigskip

\noindent\textbf{Problem c:} For $D=\{x^2+1\mid x\in A\}$, is it the case that $\sup D=(\sup A)^2+1$?
\bigskip

\noindent\textbf{Solution:} No, consider the following counterexample:
\begin{align*}
  A&=\{-2,1\}\\
  D&=\{x^2+1\mid x\in A\}\\
  &=\{5,2\}
\end{align*}

We have:
\begin{align*}
  \sup A&=1\\
  (\sup A)^2+1&=2\\
  &\not=5\\
  &=\sup D
\end{align*}

And so we have that $\sup D\not=(\sup A)^2+1$.

\end{document}
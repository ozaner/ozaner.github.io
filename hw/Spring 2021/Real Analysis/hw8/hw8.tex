\documentclass{article}
\usepackage{amsmath,mathtools}
\usepackage{amssymb}
\usepackage[dvipsnames]{xcolor}
\usepackage{graphicx}
\usepackage{xargs}
\usepackage{enumitem}
\usepackage{systeme}
\usepackage{centernot}
\usepackage{physics}
\usepackage{xfrac}
\usepackage{titling}
\usepackage[margin=1in]{geometry}
\usepackage[skins,theorems]{tcolorbox}
\tcbset{highlight math style={enhanced,
  colframe=blue,colback=white,arc=0pt,boxrule=1pt}}

% calculus commands
\renewcommand{\eval}[3]{\left[#1\right]_{#2}^{#3}}

% linear algebra commands
\renewcommand\vec{\mathbf}
\newenvironment{sysmatrix}[1]
{\left[\begin{array}{@{}#1@{}}}
{\end{array}\right]}
\newcommand{\ro}[1]{%
\xrightarrow{\mathmakebox[\rowidth]{#1}}%
}
\newlength{\rowidth}% row operation width
\AtBeginDocument{\setlength{\rowidth}{3em}}

%set theory commands
\newcommand{\pset}[1]{\mathcal P(#1)}
\newcommand{\card}[1]{\operatorname{card}(#1)}
\newcommand{\R}{\mathbb R}
\newcommand{\Q}{\mathbb Q}
\newcommand{\Z}{\mathbb Z}
\newcommand{\N}{\mathbb N}

%number theory commands
\newcommand{\divides}{\mid}
\newcommand{\ndivides}{\nmid}
\newcommand{\fallfact}[2]{#1^{\underline {#2}}}

%optimization commands
\DeclareMathOperator*{\argmax}{arg\,max}
\DeclareMathOperator*{\argmin}{arg\,min}

\setlength{\droptitle}{-7em}   % This is your set screw

\begin{document}

\title{Intro to Real Analysis\\HW \#8}
\author{Ozaner Hansha}
\date{April 16, 2021}
\maketitle

\subsection*{Problem 1}
\noindent\textbf{Solution:} Suppose, for contradiction, that $f$ is \textit{not} uniformly continuous on $A$. Then there would exist some $\epsilon_0>0$ and sequences whose elements $x_k,y_k\in A$ such that:
$$\lim_{k\to\infty}|x_k-y_k|=0\wedge (\forall k\in\N)\,\,|f(x_k)-f(x_y)|\ge\epsilon_0$$

Now note that since $A$ is is closed and bounded, there is a convergent subsequence $x_{k_i}$ of $x_k$ such that:
$$\lim_{i\to\infty} x_{k_i}=x\in A$$

Moreover, since $x_k-y_k\to0$ as $k\to\infty$, it follows that:
$$\lim_{i\to\infty}y_{k_i}=\lim_{i\to\infty}(x_{k_i}-(x_{k_i}-y_{k_i}))=\lim_{i\to\infty}x_{k_i}-\lim_{i\to\infty}x_{k_i}-y_{k_i}=x$$

so $y_{k_i}$ also converges to $x$. And since $f$ is continuous on $A$, we have:
$$\lim_{i\to\infty}|f(x_{k_i})-f(y_{k_i})|=|\lim_{i\to\infty}f(x_{k_i})\lim_{i\to\infty}-f(y_{k_i})|=|f(x)-f(y)|=0$$

But this contradictions what we stated previously, namely:
$$|f(x_k)-f(x_y)|\ge\epsilon_0$$

Thus, $f$ must be uniformly continuous.

\subsection*{Problem 2}
\noindent\textbf{Solution:} if $f(1/2)\not=0$ then, since $[0,1]$ is compact, there is some neighborhood of points $N$ around $1/2$ such that each $x\in N$ satisfies $f(x)\not=0$. Thus $1/2$ is an accumulation point of $D$.

\subsection*{Problem 3}
\noindent\textbf{Solution:} If $f(0)=0$ or $f(1)=1$ then we are done. Otherwise, define $g(x)=f(x)-x$. Certainly $g$ is continuous as it is the difference of two continuous functions. Now note the following:
\begin{align*}
  0&<f(0)&\tag{$f(0)=0$ case already considered}\\
  &<f(0)-0\\
  &<g(0)\tag{def. of $g$}\\
  1&>f(1)\tag{$f(1)=1$ case already considered}\\
  0&>f(1)-1\\
  &>g(1)\tag{def. of $g$}
\end{align*}

And so we have that $g(1)<0<g(0)$, and so by the intermediate value theorem, there must be some $x\in[0,1]$ such that $g(x)=0$ which is equivalent to:
\begin{align*}
  g(x)&=0\\
  f(x)-x&=0\tag{def. of $g$}\\
  f(x)&=x
\end{align*}

And so there is some $x\in[0,1]$ such that $f(x)=x$.

\subsection*{Problem 4}
\noindent\textbf{Solution:} Recall that a compact subset $E$ of $\R$ is one that is both bounded, and closed. Since $E$ is bounded, the supremum $\sup(E)=a$ must exist. This means there is a sequence $x_n\in E$ such that $x_n\to a$, since the supremum of a set lies either in it, or on its boundary. And since $E$ is closed, that $a\in E$, meaning $a=\sup(E)\in E$.

\subsection*{Problem 5}
\noindent\textbf{Solution:} A Cauchy sequence has a limit, and so $x_k\to x$. Next note that $x$ is an accumulation point of $E$ since each $x_k\in E$. Since $E$ is closed, that accumulation point $x\in E$.
\end{document}
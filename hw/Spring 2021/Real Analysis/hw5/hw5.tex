\documentclass{article}
\usepackage{amsmath,mathtools}
\usepackage{amssymb}
\usepackage[dvipsnames]{xcolor}
\usepackage{graphicx}
\usepackage{xargs}
\usepackage{enumitem}
\usepackage{systeme}
\usepackage{centernot}
\usepackage{physics}
\usepackage{xfrac}
\usepackage{titling}
\usepackage[margin=1in]{geometry}
\usepackage[skins,theorems]{tcolorbox}
\tcbset{highlight math style={enhanced,
  colframe=blue,colback=white,arc=0pt,boxrule=1pt}}

% calculus commands
\renewcommand{\eval}[3]{\left[#1\right]_{#2}^{#3}}

% linear algebra commands
\renewcommand\vec{\mathbf}
\newenvironment{sysmatrix}[1]
{\left[\begin{array}{@{}#1@{}}}
{\end{array}\right]}
\newcommand{\ro}[1]{%
\xrightarrow{\mathmakebox[\rowidth]{#1}}%
}
\newlength{\rowidth}% row operation width
\AtBeginDocument{\setlength{\rowidth}{3em}}

%set theory commands
\newcommand{\pset}[1]{\mathcal P(#1)}
\newcommand{\card}[1]{\operatorname{card}(#1)}
\newcommand{\R}{\mathbb R}
\newcommand{\Q}{\mathbb Q}
\newcommand{\Z}{\mathbb Z}
\newcommand{\N}{\mathbb N}

%number theory commands
\newcommand{\divides}{\mid}
\newcommand{\ndivides}{\nmid}
\newcommand{\fallfact}[2]{#1^{\underline {#2}}}

%optimization commands
\DeclareMathOperator*{\argmax}{arg\,max}
\DeclareMathOperator*{\argmin}{arg\,min}

\setlength{\droptitle}{-7em}   % This is your set screw

\begin{document}

\title{Intro to Real Analysis\\HW \#5}
\author{Ozaner Hansha}
\date{March 5, 2021}
\maketitle

\subsection*{Problem 1}
\noindent\textbf{Problem:} Define $f:(0,1):\to\R$ by $x\mapsto\sqrt{x}\sin\left(\frac{1}{x}\right)$. Use $\epsilon-\delta$ language to find $\lim_{x\to0}f(x)$.
\bigskip

\noindent\textbf{Solution:} Consider a fixed $\epsilon>0$, let $\delta=\epsilon^2>0$. Then, for all $x$ such that $0<|x|<\delta$ we have:
\begin{align*}
  \delta&>|x|\tag{hypothesis}\\
  \epsilon^2&>|x|\tag{def. of $\delta$}\\
  \epsilon&>\sqrt{|x|}\\
  &>|\sqrt{x}|\tag{$\forall x\in(0,1),\,x>0$}\\
  &>|\sqrt{x}|\left|\sin\left(\frac{1}{x}\right)\right|\tag{$\forall c\in\R,\,0\le|\sin c|\le 1$}\\
  &>\left|\sqrt{x}\sin\left(\frac{1}{x}\right)\right|\\
  &>|f(x)|\tag{def. of $f(x)$}\\
  &>|f(x)-0|
\end{align*}

In other words, we have shown:
\begin{equation*}
  (\forall \epsilon>0)\underbrace{(\exists\delta>0)}_{\text{namely }\epsilon^2},\,0<|x-0|<\delta\implies|f(x)-0|<\epsilon
\end{equation*}

Which is prescisly the definition of:
\begin{equation*}
  \lim_{x\to0}f(x)=0
\end{equation*}

\subsection*{Problem 2}
\noindent\textbf{Problem:} Consider a function $f:D\to\R$. Suppose that $\lim_{x\to x_o}f(x)=c$, use $\epsilon-\delta$ language to show that: 
$$\lim_{x\to x_0}|f(x)|=|c|$$

\noindent\textbf{Solution:} First let us establish the reverse triangle inequality. Consider any two reals $x,y$:
\begin{align*}
  |x+y-x|&\le|x|+|y-x|\tag{triangle inequality}\\
  |y|&\le|x|+|y-x|\\
  |y|-|x|&\le|y-x|\\
  |y|-|x|\le|y-x|&\wedge|x|-|y|\le|x-y|\tag{$x$ and $y$ are indistinguishable}\\
  |x|-|y|\ge-|x-y|&\wedge|x|-|y|\le|x-y|\\
  \left||x|-|y|\right|&\le|x-y|\tag{reverse triangle inequality}
\end{align*}

With this in mind, note that by assuming $\lim_{x\to x_o}f(x)=c$ we have, $(\forall\epsilon>0)(\exists\delta>0)(\forall x\in D)$:
\begin{align*}
  0<|x-x_0|<\delta&\implies |f(x)-c|<\epsilon\tag{def. of limit}\\
  &\implies ||f(x)|-|c||\le|f(x)-c|<\epsilon\tag{reverse triangle inequality}\\
  &\implies ||f(x)|-|c||<\epsilon\tag{transitivity}
\end{align*}

Note that this is precisely the definition of:
$$\lim_{x\to x_0}|f(x)|=|c|$$

\subsection*{Problem 3}
\noindent\textbf{Problem:} Define $f:(0,1)\to\R$ by $x\mapsto(1+x)^{\sfrac{1}{x}}$. Find $\lim_{x\to0}f(x)$.
\bigskip

\noindent\textbf{Solution:} First recall Bernoulli's inequality for general $r$ and $y$.
\begin{align*}
  (\forall r\ge 1)(\forall x\ge -1),\,\,(1+x)^r&\ge 1+rx\\
  (\forall r\in[0,1])(\forall x\le -1),\,\,(1+x)^r&\ge 1+rx
\end{align*}

First, note the following:
\begin{align*}
  (1+x)^{\sfrac{1}{x}}\ge 1+\frac{x}{x}=2\tag{$r=\sfrac{1}{x}>1$ \& $x=x\ge -1$, Bernoulli's inequality}\\
  \left(1+\frac{1}{x}\right)^x\le 1+\frac{x}{x}=2\tag{$r=x\in(0,1)$ \& $x=\sfrac{1}{x}\ge -1$, Bernoulli's inequality}
\end{align*}

Leading us to the inequality:
\begin{equation*}
  \left(1+\frac{1}{x}\right)^x\le2\le(1+x)^{\sfrac{1}{x}}
\end{equation*}

And now consider the following:
\begin{align*}
  e&=\lim_{x\to\infty}\left(1+\frac{1}{x}\right)^x\tag{def. of $e$}\\
  &=\lim_{\sfrac{1}{u}\to\infty}\left(1+u\right)^{\sfrac{1}{u}}\tag{change of variables $\substack{u=\sfrac{1}{x}\\x=\sfrac{1}{x}}$}\\
  &=\lim_{\left|\sfrac{1}{u}\right|\to\infty}\left(1+u\right)^{\sfrac{1}{u}}\\
  &=\lim_{u\to0}\left(1+u\right)^{\sfrac{1}{u}}\tag{$\lim_{u\to0}\left|\frac{1}{u}\right|=\infty$}
\end{align*}

\subsection*{Problem 4}
Let $f,g:\R\to\R$ be two functions such that:
\begin{gather*}
  \lim_{x\to0}f(x)=c\\
  \lim_{x\to c}g(x)=d
\end{gather*}
\bigskip

\noindent\textbf{Part a:} Suppose there exists $\delta>0$ such that for any $x\in(-\delta,0)\cup(0,\delta)$ we have $f(x)\not=c$. Prove the following:
\begin{equation*}
  \lim_{x\to 0}g\circ f(x)=d
\end{equation*}
\bigskip

\noindent\textbf{Solution:} Let us call the supposition above condition a. Let us now rewrite this condition, $(\exists\delta>0)$:
\begin{align*}
  x\in(-\delta,0)\cup(0,\delta)&\implies f(x)\not=c\tag{condition a}\\
  0<|x|<\delta&\implies f(x)\not=c\tag{def. of absolute value}\\
  &\implies f(x)-c\not=0\\
  &\implies |f(x)-c|\not=0\\
  &\implies 0<|f(x)-c|\tag{absolute value is nonnegative}
\end{align*}

Now let us prove the statement, $(\forall\epsilon_1>0)(\exists\delta_1>0)(\exists\delta>0)(\forall\epsilon_2>0)(\exists\delta_2>0)$:
\begin{align*}
  0<|x|<\delta_1&\implies|f(x)-c|<\epsilon_1\tag{def. of $\lim_{x\to0}f(x)=c$}\\
  0<|x|<\delta&\implies0<|f(x)-c|<\epsilon_1\tag{condition a, let $\delta_1=\delta$}\\
  &\implies |g(f(x))-d|<\epsilon_2\tag{def. of $\lim_{x\to c}g(x)=d$, let $x=f(x)$, let $\delta_2=\epsilon_1$}
\end{align*}

Simplifying, we have:
\begin{equation*}
  (\forall\epsilon_2>0)(\exists\delta>0),\,\,0<|x|<\delta\implies|f(x)-d|<\epsilon_2
\end{equation*}

Which is precisely the definition of:
\begin{equation*}
  \lim_{x\to 0}g\circ f(x)=d
\end{equation*}
\bigskip

\noindent\textbf{Part b:} Without condition a, find an example where:
\begin{equation*}
  \lim_{x\to 0}g\circ f(x)\not=d
\end{equation*}
\bigskip

\noindent\textbf{Solution:} Let us define our functions and constants:
\begin{align*}
  f(x)&=0\\
  g(x)&=\begin{cases}
    1,&x=0\\
    0,&\text{otherwise}
  \end{cases}\\
  c&=0\\
  d&=0
\end{align*}

Now let us verify that these definitions satisfy the conditions of the problem:
\begin{gather*}
  \lim_{x\to0}f(x)=\lim_{x\to0}0=0=c\\
  \neg(\exists\delta,\,\,0<|x|<\delta\implies f(x)\not=c)\tag{$\forall x, f(x)=0=c$}\\
  \lim_{x\to c}g(x)=\lim_{x\to 0}\left(\begin{cases}
    1,&x=0\\
    0,&\text{otherwise}\\
  \end{cases}\right)=0=d
\end{gather*}

Now note the following:
\begin{align*}
  \lim_{x\to0}g(f(x))=\lim_{x\to0}g(0)=\lim_{x\to0}1=1\not=0=d
\end{align*}

And so we have produced our counterexample.

\end{document}
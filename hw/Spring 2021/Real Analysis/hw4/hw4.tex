\documentclass{article}
\usepackage{amsmath,mathtools}
\usepackage{amssymb}
\usepackage[dvipsnames]{xcolor}
\usepackage{graphicx}
\usepackage{xargs}
\usepackage{enumitem}
\usepackage{systeme}
\usepackage{centernot}
\usepackage{physics}
\usepackage{xfrac}
\usepackage{titling}
\usepackage[margin=1in]{geometry}
\usepackage[skins,theorems]{tcolorbox}
\tcbset{highlight math style={enhanced,
  colframe=blue,colback=white,arc=0pt,boxrule=1pt}}

% calculus commands
\renewcommand{\eval}[3]{\left[#1\right]_{#2}^{#3}}

% linear algebra commands
\renewcommand\vec{\mathbf}
\newenvironment{sysmatrix}[1]
{\left[\begin{array}{@{}#1@{}}}
{\end{array}\right]}
\newcommand{\ro}[1]{%
\xrightarrow{\mathmakebox[\rowidth]{#1}}%
}
\newlength{\rowidth}% row operation width
\AtBeginDocument{\setlength{\rowidth}{3em}}

%set theory commands
\newcommand{\pset}[1]{\mathcal P(#1)}
\newcommand{\card}[1]{\operatorname{card}(#1)}
\newcommand{\R}{\mathbb R}
\newcommand{\Q}{\mathbb Q}
\newcommand{\Z}{\mathbb Z}
\newcommand{\N}{\mathbb N}

%number theory commands
\newcommand{\divides}{\mid}
\newcommand{\ndivides}{\nmid}
\newcommand{\fallfact}[2]{#1^{\underline {#2}}}

%optimization commands
\DeclareMathOperator*{\argmax}{arg\,max}
\DeclareMathOperator*{\argmin}{arg\,min}

\setlength{\droptitle}{-7em}   % This is your set screw

\begin{document}

\title{Intro to Real Analysis\\HW \#4}
\author{Ozaner Hansha}
\date{February 20, 2021}
\maketitle

\subsection*{Problem 1}
Consider $(a_n)_{n=1}^\infty$ where $a_n\ge 0$ and:
$$\lim_{n\to\infty} a_n=A$$

\noindent\textbf{Part a:} Show that $(b_n)_{n=1}^\infty$ where $b_n=\sqrt{a_n}$ is also a convergent sequence and that:
$$\lim_{n\to\infty} b_n=\sqrt{A}$$

\noindent\textbf{Solution:} Before we prove this, let use first show why $A\not<0$:

Let us suppose that indeed $A<0$. We have $(\forall \epsilon>0)(\exists N\in\N)(\exists n>N)$:
\begin{align*}
  |a_n-A|&<\epsilon\tag{def. of convergence}\\
  |a_n-A|&<-A\tag{$-A>0$}\\
  A&<a_n-A<-A\\
  2A&<a_n<0
\end{align*}
Which is a contradiction since we know $a_n$ is nonnegative. Thus, by the trichotomy of the reals, we have two cases to consider:
\begin{itemize}
  \item If $A=0$ then we have $(\forall \epsilon>0)(\exists N\in\N)(\exists n>N)$:
  \begin{align*}
    |a_n-0|&<\epsilon^2\tag{def. of convergence, $\epsilon^2>0$}\\
    |a_n|&<\epsilon^2\\
    \sqrt{|a_n|}&<\epsilon\tag{both sides of inequality are positive}\\
    {|\sqrt{a_n}|}&<\epsilon\tag{$a_n>0$}\\
    {|\sqrt{a_n}-0|}&<\epsilon\\
    {|b_n-0|}&<\epsilon\tag{def. of $b_n$}\\
    {|b_n-\sqrt{0}|}&<\epsilon\\
    {|b_n-\sqrt{A}|}&<\epsilon\tag{$A=0$ by assumption}
  \end{align*} 

  So in other words we have $\lim_{n\to\infty}b_n=\sqrt{A}$.

  \item  If $A>0$ then we have the following $(\forall \epsilon>0)(\exists N\in\N)(\exists n>N)$:
  \begin{align*}
    |b_n-\sqrt{A}|&=|\sqrt{a_n}-\sqrt{A}|\tag{def. of $b_n$}\\
    &=\frac{|a_n-A|}{\sqrt{a_n}+\sqrt{A}}\tag{multiply by conjugate}\\
    &<\frac{|a_n-A|}{\sqrt{A}}\tag{$\sqrt{a_n}>0$}\\
    &<\frac{\epsilon\sqrt{A}}{\sqrt{A}}\tag{def. of convergence, $\epsilon\sqrt{A}>0$}\\
    &=\epsilon
  \end{align*}

  So in other words we have $\lim_{n\to\infty}b_n=\sqrt{A}$.
\end{itemize}
\bigskip

\noindent\textbf{Part b:} Prove the following:
$$(\forall c>0)\,\,\lim_{n\to\infty}\sqrt[n]{c}=1$$

\noindent\textbf{Solution:} We have two cases:
\begin{itemize}
  \item $c\ge 1$. Let $c=1+b$. We then have:
  \begin{align*}
    \left(1+\frac{b}{n}\right)^n&=\sum_{k=0}^n\binom{n}{k}\left(\frac{b}{n}\right)^k\tag{binomial theorem}\\
    &>\binom{n}{0}+\binom{n}{1}\frac{b}{n}\tag{first 2 terms}\\
    &=1+b
  \end{align*}

  And so we have:
  \begin{align*}
    1&\le c=1+b<\left(1+\frac{b}{n}\right)^n\tag{see above}\\
    1&\le \sqrt[n]{c}<1+\frac{b}{n}\tag{$n$th root, all sides positive}\\
    \lim_{n\to\infty}1&\le\lim_{n\to\infty}\sqrt[n]{c}<\lim_{n\to\infty}1+\frac{b}{n}\tag{squeeze theorem}\\
    1&\le\lim_{n\to\infty}\sqrt[n]{c}<1\\
    &\implies\lim_{n\to\infty}\sqrt[n]{c}=1
  \end{align*}

  \item $c<1$. Consider an arbitrary real $\epsilon>0$. Choose a positive integer $N>\frac{1}{c\epsilon}$. We then have for $n\ge N$:
  \begin{align*}
    \left(\sqrt[n]{c}+\epsilon\right)^n&=\sum_{k=0}^n\binom{n}{k}\sqrt[n]{c}^{n-k}\epsilon^k\tag{binomial theorem}\\
    &>c+\sqrt[n]{c}^{n-1}n\epsilon\tag{first two terms}\\
    &>c+cn\epsilon\\
    &>c+\frac{c}{c\epsilon}\epsilon\\
    &=c+1\\
    &>c
  \end{align*}

  Transforming this result further, we have:
  \begin{align*}
    1&<\left(\sqrt[n]{c}+\epsilon\right)^n\tag{see above}\\
    1&<\sqrt[n]{c}+\epsilon\\
    1-\sqrt[n]{c}&<\epsilon\\
    -(\sqrt[n]{c}-1)&<\epsilon\\
    |\sqrt[n]{c}-1|&<\epsilon\tag{$c<1\implies\sqrt[n]{c}<1\implies\sqrt[n]{c}-1<0$}
  \end{align*}

  And so we have shown that for any $\epsilon>0$ there is a choice of $N$ such that for all $n>N$ we have $|\sqrt[n]{c}-1|<\epsilon$. In other words:
  \begin{equation*}
    \lim_{n\to\infty}\sqrt[n]{c}=1
  \end{equation*}
\end{itemize}

\subsection*{Problem 2}
\noindent\textbf{Problem:} Let $a,b>0$, show that:
$$\lim_{n\to\infty}\sqrt[n]{a^n+b^n}=\max\{a,b\}$$

\noindent\textbf{Solution:} W.l.o.g we can assume $a\le b$. Now consider the following:
\begin{align*}
  b&=\lim_{n\to\infty}b\\
  &=\lim_{n\to\infty}\sqrt[n]{b^n}\\
  &\le\lim_{n\to\infty}\sqrt[n]{a^n+b^n}\tag{$a^n+b^n\ge b^n$}\\ 
  &\le\lim_{n\to\infty}\sqrt[n]{2b^n}\tag{$a<b$}\\
  &=\lim_{n\to\infty}\sqrt[n]{2}\lim_{n\to\infty}\sqrt[n]{b^n}\\
  &=\lim_{n\to\infty}\sqrt[n]{b^n}\tag{Problem 1, Part b}\\
  &=b
\end{align*}

In other words we have:
\begin{align*}
  &b\le \lim_{n\to\infty}\sqrt[n]{a^n+b^n}\le b\tag{see above}\\
  \implies&\lim_{n\to\infty}\sqrt[n]{a^n+b^n}=b\tag{squeeze theorem}\\
  \implies&\lim_{n\to\infty}\sqrt[n]{a^n+b^n}=\max\{a, b\}\tag{$a\le b$}
\end{align*}

Of course, the same argument holds when $a\ge b$ with $a$ and $b$ switching roles.

\subsection*{Problem 3}
Consider a sequence $(a_n)_{n=1}^\infty$ whose limit is $A$.
\bigskip

% \noindent\textbf{Part a:} Prove the following:
\noindent\textbf{Problem:} Prove the following:
\begin{equation*}
  \lim_{n\to\infty}\frac{a_1+\cdots+a_n}{n}=A
\end{equation*}
\medskip

\noindent\textbf{Solution:} Since $a_n\to L$ we must have that $(\forall \epsilon>0)(\exists N\in\N)(\forall n\ge N)$:
\begin{align*}
  |a_n-A|<\frac{\epsilon}{2}\tag{lemma 1}
\end{align*}

And so $\exists M>N$ such that $\forall n>M$:
\begin{align*}
  \frac{|a_1-A|+\cdots+|a_N-A|}{n}<\frac{\epsilon}{2}\tag{lemma 2}
\end{align*}

Then $(\forall n>M)$ we have:
\begin{align*}
  \left|\frac{a_1+\cdots+a_n}{n}-A\right|&=\left|\frac{a_1+\cdots a_n-nA}{n}\right|\\
  &=\left|\frac{(a_1-A)+\cdots+(a_n-A)}{n}\right|\\
  &=\left|\frac{(a_1-A)+\cdots+(a_N-A)+(a_{N+1}-A)+\cdots+(a_n-A)}{n}\right|\tag{$n>M>N$}\\
  &\le\frac{|a_1-A|+\cdots+|a_N-A|}{n}+\frac{|a_{N+1}-A|+\cdots+|a_n-A|}{n}\tag{triangle inequality}\\
  &<\frac{\epsilon}{2}+\frac{(n-N)\epsilon}{2n}\tag{lemma 1 \& 2}\\
  &<\frac{\epsilon}{2}+\frac{\epsilon}{2}=\epsilon\tag{$n>N\implies 0<\frac{n-N}{n}<1$}
\end{align*}

And so, by the definition of convergence, we have:
\begin{equation*}
  \lim_{n\to\infty}\frac{a_1+\cdots+a_n}{n}=A
\end{equation*}
\bigskip

% \noindent\textbf{Part b:} Prove the following:
% \begin{equation*}
%   \lim_{n\to\infty}\frac{a_1+2a_2\cdots+na_n}{n^2}=\frac{A}{2}
% \end{equation*}
% \bigskip

% \noindent\textbf{Solution:} 

\subsection*{Problem 4}
\noindent\textbf{Part a:} Consider a sequence $(a_n)_{n=1}^\infty$ where $n=(1+a_n)^2$. Show that for $n>1$:
$$0<a-n<\sqrt{\frac{2}{n-1}}$$

\noindent\textbf{Solution:} First let us solve for $a_n$:
\begin{align*}
  (1+a_n)^n&=n\\
  1+a_n&=\sqrt[n]{n}\\
  a_n&=\sqrt[n]{n}-1
\end{align*}

Recall from problem 1, part b that $c>1\implies \sqrt[n]{c}>1$ for any $n\in\N$. And so we have:
\begin{align*}
  a_n&=\sqrt[n]{n}-1\tag{def. of $a_n$}\\
  &>1-1=0
\end{align*}

Now we have to prove the other side of the inequality. Consider the following:
\begin{align*}
  n&=\left(\sqrt[n]{n}\right)^n\\
  &=(1+(\sqrt[n]{n}-1))^n\\
  &=\sum_{k=0}^n\binom{n}{k}(\sqrt[n]{k})^k\tag{binomial theorem}\\
  &\ge\binom{n}{2}(\sqrt[n]{n}-1)^2\tag{second term only, $n>1$}\\
  &=\frac{n(n-1)}{2}(\sqrt[n]{n}-1)^2
\end{align*}

In other words we have:
\begin{align*}
  \frac{n(n-1)}{2}(\sqrt[n]{n}-1)^2&\le n\tag{see above}\\
  (\sqrt[n]{n}-1)^2&\le\frac{2}{n-1}\\
  \sqrt[n]{n}-1&\le\sqrt{\frac{2}{n-1}}\\
  a_n&\le\sqrt{\frac{2}{n-1}}\tag{def. of $a_n$}
\end{align*}

And so, putting our two inequalities together, we have proved the desired statement:
\begin{equation*}
  (\forall n>1)\,\,0<a_n\le\sqrt{\frac{2}{n-1}}
\end{equation*}

\noindent\textbf{Part b:} Show that:
$$\lim_{n\to\infty}\sqrt[n]{n}=1$$

\noindent\textbf{Solution:} Consider an arbitrary real $\epsilon>0$. Choose a positive integer $N>1+\frac{2}{\epsilon^2}$. We then have:
\begin{align*}
  N&>1+\frac{2}{\epsilon^2}\tag{$N$ exists by archemdiean property}\\
  N-1&>\frac{2}{\epsilon^2}\\
  \frac{1}{N-1}&<\frac{\epsilon^2}{2}\\
  \frac{2}{N-1}&<\epsilon^2\\
  \sqrt{\frac{2}{N-1}}&<\epsilon
\end{align*}

And so we have for any $n\ge N$:
\begin{align*}
  \sqrt[n]{n}-1&\le\frac{2}{n-1}\tag{part a}\\
  |\sqrt[n]{n}-1|&\le\frac{2}{n-1}\tag{$n>N>1\implies\sqrt[n]{n}-1>0$}\\
  &\le\frac{2}{N-1}\\
  &<\epsilon\tag{see above}
\end{align*}

And so we have shown that $(\forall\epsilon>0)(\exists N>\N)(\forall n>N)$ we have $|\sqrt[n]{n}-1|<\epsilon$. This is the definition of:
\begin{equation*}
  \lim_{n\to\infty}\sqrt[n]{n}=1
\end{equation*}

% \subsection*{Problem 5}
% \noindent\textbf{Problem:} 
% \bigskip

% \noindent\textbf{Solution:} 

\end{document}
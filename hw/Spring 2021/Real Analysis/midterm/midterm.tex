\documentclass{article}
\usepackage{amsmath,mathtools}
\usepackage{amssymb}
\usepackage[dvipsnames]{xcolor}
\usepackage{graphicx}
\usepackage{xargs}
\usepackage{enumitem}
\usepackage{systeme}
\usepackage{centernot}
\usepackage{physics}
\usepackage{xfrac}
\usepackage{titling}
\usepackage[margin=1in]{geometry}
\usepackage[skins,theorems]{tcolorbox}
\tcbset{highlight math style={enhanced,
  colframe=blue,colback=white,arc=0pt,boxrule=1pt}}

% calculus commands
\renewcommand{\eval}[3]{\left[#1\right]_{#2}^{#3}}

% linear algebra commands
\renewcommand\vec{\mathbf}
\newenvironment{sysmatrix}[1]
{\left[\begin{array}{@{}#1@{}}}
{\end{array}\right]}
\newcommand{\ro}[1]{%
\xrightarrow{\mathmakebox[\rowidth]{#1}}%
}
\newlength{\rowidth}% row operation width
\AtBeginDocument{\setlength{\rowidth}{3em}}

%set theory commands
\newcommand{\pset}[1]{\mathcal P(#1)}
\newcommand{\card}[1]{\operatorname{card}(#1)}
\newcommand{\R}{\mathbb R}
\newcommand{\Q}{\mathbb Q}
\newcommand{\Z}{\mathbb Z}
\newcommand{\N}{\mathbb N}

%number theory commands
\newcommand{\divides}{\mid}
\newcommand{\ndivides}{\nmid}
\newcommand{\fallfact}[2]{#1^{\underline {#2}}}

%optimization commands
\DeclareMathOperator*{\argmax}{arg\,max}
\DeclareMathOperator*{\argmin}{arg\,min}

\setlength{\droptitle}{-7em}   % This is your set screw

\begin{document}

\title{Intro to Real Analysis\\Midterm}
\author{Ozaner Hansha}
\date{February 22, 2021}
\maketitle

\subsection*{Problem 1}
\noindent\textbf{Problem:} Consider the sequences $(a_n)_{n=1}^\infty$ and $(b_n)_{n=1}^\infty$ given by:
\begin{align*}
  a_n&=\sum_{k=1}^n4k\\
  b_n&=\sum_{k=1}^n2k+1
\end{align*}

Show that $\left(\frac{a_n}{b_n}\right)_{n=1}^\infty$ is convergent and give its limit.
\bigskip

\noindent\textbf{Solution:} Note that, if the desired limit existed, it would be an infinite series:
\begin{align*}
  \lim_{n\to\infty}\frac{a_n}{b_n}&=\lim_{n\to\infty}\frac{\sum_{k=1}^n4k}{\sum_{k=1}^n 2k+1}\\
  &=\lim_{n\to\infty}\sum_{k=1}^n \frac{4k}{2k+1}\\
  &=\sum_{k=1}^\infty \frac{4k}{2k+1}\tag{def. of infinite series}
\end{align*}

However, you'll note that this series is divergent. To see this, recall that a necessary, but not sufficient, condition for a series to be convergent is for the limit of its summand to be 0. Yet this is not the case:
\begin{align*}
  \lim_{k\to\infty}\frac{4k}{2k+1}&=\lim_{k\to\infty}\frac{4}{2+\sfrac{1}{k}}\\
  &=\lim_{k\to\infty}\frac{4}{2+0}\\
  &=2\not=0
\end{align*}

And so, we \text{cannot} give the limit of this series as it does not exist.

\subsection*{Problem 2}
\noindent\textbf{Problem:} Find a bijection from $I$ to the following set to $[0,1]$, where $I$ is given by:
\begin{equation*}
  I=(1,2)\cup(2,3)\cup\cdots\cup(2020,2021)
\end{equation*}
\bigskip

\noindent\textbf{Solution:} We will first construct a bijective function $g_{a,b,c,d}:[a,b)\to(c,d)$ for any $a<b\wedge c<d$.
\medskip

Consider the following sequence $(y_n)_{n=1}^\infty$ given by:
\begin{equation*}
  y_n=a+.1^{n}(b-a)
\end{equation*}

This is an injective infinite sequence of numbers in $[a,b)$. Now consider the following bijection $\qquad \qquad$ $h_{a,b}:(a,b)\to[a,b)$:
\begin{equation*}
  h_{a,b}(x)=\begin{cases}
    a, &x=y_1\\
    y_{n-1},&x=y_n,\,\,n>1\\
    x,&\text{otherwise}
  \end{cases}
\end{equation*}

We can now transform this bijection to have our desired codomain, resulting in $g_{a,b,c,d}:(a,b)\to[c,d)$:
\begin{equation*}
  g_{a,b,c,d}(x)=\frac{d}{b-a+c}(h_{a,b}(x)-a+c)\tag{call this lemma 1}
\end{equation*}
\medskip

Now let us note one last fact:
\begin{equation*}
  [0,1]=\left[0,\frac{1}{2021}\right)\cup\left[\frac{1}{2021},\frac{2}{2020}\right)\cup\cdots\cup\left[\frac{2020}{2021},1\right)\tag{call this lemma 2}\cup\{1\}
\end{equation*}

And so by lemma 2, a bijection $f_1:I\to[0,1)$ can be given as the following piecewise function:
$$f_1(x)=\begin{cases}
  g_{1,2,0,\frac{1}{2021}}(x),&x\in(1,2)\\
  g_{2,3,\frac{1}{2021},\frac{2}{2021}}(x),&x\in(2,3)\\
  \vdots\\
  g_{i,i+1\frac{i-1}{2021},\frac{i}{2021}}(x),&x\in(i,i+1)\\
  \vdots\\
  g_{2020,2021,\frac{2020}{2021},1}(x),&x\in(2020,2021)
\end{cases}$$

And of course, that each case of the partition of $I\setminus\{1\}$ is bijective is given by lemma 1. Now all that's left is to deal with the leftover 1 not yet in the codomain. Consider another sequence:
$(z_n)_{n=1}^\infty$ given by:
\begin{equation*}
  z_n=1-.2^{n}
\end{equation*}

With this we can finally give our desired bijection $f_2:I\to[0,1]$:
\begin{equation*}
  f_2(x)=\begin{cases}
    1, &f_1(x)=z_1\\
    z_{n-1},&f_1(x)=z_n,\,n>1\\
    f_1(x),&\text{otherwise}
  \end{cases}
\end{equation*}

\subsection*{Problem 3}
\noindent\textbf{Problem:} Compute the following limit:
\begin{equation*}
  \lim_{n\to\infty}\left(1+\frac{1}{3n+1}\right)^{2n}
\end{equation*}
\bigskip

\noindent\textbf{Solution:} Consider the following:
\begin{align*}
  \lim_{n\to\infty}\left(1+\frac{1}{3n+1}\right)^{2n}&=\lim_{m\to\infty}\left(1+\frac{1}{m}\right)^{\frac{2m-2}{3}}\tag{$\substack{m=3n+1\\n=\frac{2m-2}{3}}$}\\
  &=\lim_{m\to\infty}\left(1+\frac{1}{m}\right)^{\frac{2m}{3}}\left(1+\frac{1}{m}\right)^{-\frac{2}{3}}\\
  &=\lim_{m\to\infty}\left(1+\frac{1}{m}\right)^{\frac{2m}{3}}\lim_{m\to\infty}\left(1+\frac{1}{m}\right)^{-\frac{2}{3}}\tag{limit of products is product of limits}\\
  &=\left(\lim_{m\to\infty}\left(1+\frac{1}{m}\right)^{m}\right)^{\frac{2}{3}}\left(\lim_{m\to\infty}\left(1+\frac{1}{m}\right)\right)^{-\frac{2}{3}}\tag{$\substack{a_m\ge0\implies\\\text{limit of root is root of limit}}$}\\
  &=\left(\lim_{m\to\infty}\left(1+\frac{1}{m}\right)^{m}\right)^{\frac{2}{3}}\left(1+\lim_{m\to\infty}\frac{1}{m}\right)^{-\frac{2}{3}}\tag{limit of sum is sum of limit}\\
  &=\left(\lim_{m\to\infty}\left(1+\frac{1}{m}\right)^{m}\right)^{\frac{2}{3}}\left(1+0\right)^{-\frac{2}{3}}\\
  &=e^{\frac{2}{3}}\tag{def. of $e$}
\end{align*}

\subsection*{Problem 4}
Consider the sequences $(a_n)_{n=1}^\infty$ and $(b_n)_{n=1}^\infty$ given by:
\begin{align*}
  a_n&=\left(1+\frac{1}{n}\right)^n\\
  b_n&=1+\sum_{k=1}^n\frac{1}{k!}=\sum_{k=0}^n\frac{1}{k!}\tag{$0!=1$}
\end{align*}
\bigskip

\noindent\textbf{Part a:} Show that $a_n\le b_n$ for all $n\ge 1$.
\bigskip

\noindent\textbf{Solution:} We wish to prove $(\forall n\in N)\,P(n)$ where:
\begin{equation*}
  P(n)\equiv a_n\le b_n
\end{equation*}

First we will show $P(1)$:
\begin{align*}
  P(1)&\iff\left(1+\frac{1}{1}\right)^1\le 1+\sum^1_{k=1}\frac{1}{k!}\tag{def. of $P(1)$}\\
  &\iff(1+1)^1\le 1+\frac{1}{1!}\\
  &\iff2\le 2\\
  &\iff T
\end{align*}

% Nowe we take a quick detour and establish two lemmas:
% \begin{itemize}
%   \item Lemma 1:
%   \begin{align*}
%     n^n=\overbrace{n\cdot n\cdots n}^{n\text{ factors}}\ge\overbrace{n\cdot (n-1)\cdots 1}^{n\text{ factors}}=n!
%   \end{align*}
%   \item Lemma 2:
%   \begin{align*}
%     n^n=\overbrace{n\cdot n\cdots n}^{n\text{ factors}}\ge\overbrace{n\cdot (n-1)\cdots 1}^{n\text{ factors}}=n!
%   \end{align*}
% \end{itemize}

Now we will show $P(n)\implies P(n+1)$
\begin{align*}
  (n+1)^{n+1}&\ge (n+1)!\\
  \frac{1}{(n+1)^{n+1}}&\le\frac{1}{(n+1)!}\\
  \frac{1}{(n+1)^{n+1}}+a_n&\le\frac{1}{(n+1)!}+b_n\tag{assume $P(n)$}\\
  \frac{1}{(n+1)^{n+1}}+\left(1+\frac{1}{n}\right)^n&\le\frac{1}{(n+1)!}+\sum_{k=0}^n\frac{1}{k!}\tag{def. of $a_n$ \& $b_n$}\\
  \frac{1}{(n+1)^{n+1}}+\left(1+\frac{1}{n+1}\right)^n&\le\frac{1}{(n+1)!}+\sum_{k=0}^n\frac{1}{k!}\tag{$\frac{1}{n+1}<\frac{1}{n}$}\\
  \frac{1}{(n+1)^{n+1}}+\sum^n_{k=0}\binom{n}{k}\frac{1}{(n+1)^k}&\le\frac{1}{(n+1)!}+\sum_{k=0}^n\frac{1}{k!}\tag{binomial theorem}\\
  \sum^{n+1}_{k=0}\binom{n}{k}\frac{1}{(n+1)^k}&\le\sum_{k=0}^{n+1}\frac{1}{k!}\\
  \left(1+\frac{1}{n+1}\right)^{n+1}&\le\sum_{k=0}^n\frac{1}{k!}\tag{binomial theorem}\\
  a_{n+1}&\le b_{n+1}\tag{def. of $a_{n+1}$ \& $b_{n+1}$}
\end{align*}

And so we have shown that both $P(1)$ and $P(n)\implies P(n+1)$. Thus, by the PMI, we have that:
\begin{equation*}
  (\forall n\ge1)\,\,\underbrace{a_n\le b_n}_{P(n)}
\end{equation*}
\bigskip

\noindent\textbf{Part b:} Find the limit of $(b_n)_{n=1}^\infty$.
\bigskip

\noindent\textbf{Solution:}  First let us establish the following identity:
\begin{align*}
  \sum^\infty_{k=0}\binom{n}{k}\frac{1}{n^k}&=\binom{n}{0}\frac{1}{n^0}+\binom{n}{1}\frac{1}{n^1}+\binom{n}{2}\frac{1}{n^2}+\binom{n}{3}\frac{1}{n^3}\cdots\\
  &=1+1+\frac{1}{2!}\left(\frac{n-1}{n}\right)+\frac{1}{3!}\left(\frac{n-1}{n}\right)\left(\frac{n-2}{n}\right)+\cdots\\
  &=1+1+\frac{1}{2!}\left(1-\frac{1}{n}\right)+\frac{1}{3!}\left(1-\frac{1}{n}\right)\left(1-\frac{2}{n}\right)+\cdots\\
  &=\sum^\infty_{k=0}\frac{1}{k!}\prod^{k-1}_{j=1}\left(1-\frac{j}{n}\right)
\end{align*}

Now consider the following:
\begin{align*}
  e&=\lim_{n\to\infty}\left(1+\frac{1}{n}\right)^n\tag{def. of $e$}\\
  &=\lim_{n\to\infty}\sum^n_{k=0}\binom{n}{k}\frac{1}{n^k}\tag{binomial theorem}\\
  &=\lim_{n\to\infty}\sum^\infty_{k=0}\binom{n}{k}\frac{1}{n^k}\\
  &=\lim_{n\to\infty}\sum^\infty_{k=0}\frac{1}{k!}\prod^{k-1}_{j=1}\left(1-\frac{j}{n}\right)\tag{identity from above}\\
  &=\sum^\infty_{k=0}\frac{1}{k!}\prod^{k-1}_{j=1}\lim_{n\to\infty}\left(1-\frac{j}{n}\right)\tag{limit of sum/product is sum/product of limit}\\
  &=\sum^\infty_{k=0}\frac{1}{k!}\prod^{k-1}_{j=1}\left(1-0\right)\\
  &=\sum^\infty_{k=0}\frac{1}{k!}\\
  &=\lim_{n\to\infty}\sum^n_{k=0}\frac{1}{k!}\tag{def. of infinite series}\\
  &=\lim_{n\to\infty}b_n\tag{def. of $b_n$}
\end{align*}

And so we are done. We have shown that the desired limit is equivalent to $e$.

% Consider the family of sequences $(c_n^{(m)})_{n=0}^\infty$ parmateried by $m\in\N$:
% \begin{align*}
%   c^{(m)}_n&=\sum^m_{k=0}\binom{n}{k}\frac{1}{n^k}\\
%   &=\binom{n}{0}\frac{1}{n^0}+\binom{n}{1}\frac{1}{n^1}+\binom{n}{2}\frac{1}{n^2}+\binom{n}{3}\frac{1}{n^3}\cdots+\binom{n}{m}\frac{1}{n^m}\\
%   &=1+1+\frac{1}{2!}\left(\frac{n-1}{n}\right)+\frac{1}{3!}\left(\frac{n-1}{n}\right)\left(\frac{n-2}{n}\right)+\cdots+\frac{1}{m!}\left(\frac{n-1}{n}\right)\left(\frac{n-2}{n}\right)\cdots\left(\frac{n-m+1}{n}\right)\\
%   &=1+1+\frac{1}{2!}\left(1-\frac{1}{n}\right)+\frac{1}{3!}\left(1-\frac{1}{n}\right)\left(1-\frac{2}{n}\right)+\cdots+\frac{1}{m!}\left(1-\frac{1}{n}\right)\left(1-\frac{2}{n}\right)\cdots\left(1-\frac{m-1}{n}\right)\\
%   &=\sum^m_{k=0}\frac{1}{k!}\prod^{k-1}_{j=1}\left(1-\frac{j}{n}\right)
% \end{align*}

% Note that $b_n=c^{(n)}_n$ and that when $m\ge n$:
% \begin{equation*}
%   b_n\le c^{(m)}_n
% \end{equation*}  

% Now consider the following:
% \begin{align*}
%   b_n&\le c^{(m)}_n\\
%   \lim_{m\to\infty}b_n&\le\lim_{m\to\infty}c^{(m)}_n\\
%   e&=\lim_{n\to\infty}\left(1+\frac{1}{n}\right)^n\tag{def. of $e$}\\
%   &=\lim_{n\to\infty}\sum^n_{k=0}\binom{n}{k}\frac{1}{n^k}\tag{binomial theorem}\\
%   &=\lim_{n\to\infty}\sum^\infty_{k=0}\binom{n}{k}\frac{1}{n^k}\\
%   &=\lim_{n\to\infty}\sum^\infty_{k=0}\frac{1}{k!}\prod^{k-1}_{j=1}\left(1-\frac{j}{n}\right)\tag{identity from above}\\
%   &=\sum^\infty_{k=0}\frac{1}{k!}\prod^{k-1}_{j=1}\lim_{n\to\infty}\left(1-\frac{j}{n}\right)\tag{limit of sum/product is sum/product of limit}\\
%   &=\sum^\infty_{k=0}\frac{1}{k!}\prod^{k-1}_{j=1}\left(1-0\right)\\
%   &=\sum^\infty_{k=0}\frac{1}{k!}\\
%   &=\lim_{n\to\infty}\sum^n_{k=0}\frac{1}{k!}\tag{def. of infinite series}\\
%   &=\lim_{n\to\infty}b_n\tag{def. of $b_n$}
% \end{align*}

% And so we are done. We have shown that the desired limit is equivalent to $e$.

\end{document}
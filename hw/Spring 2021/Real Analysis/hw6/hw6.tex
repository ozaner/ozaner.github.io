\documentclass{article}
\usepackage{amsmath,mathtools}
\usepackage{amssymb}
\usepackage[dvipsnames]{xcolor}
\usepackage{graphicx}
\usepackage{xargs}
\usepackage{enumitem}
\usepackage{systeme}
\usepackage{centernot}
\usepackage{physics}
\usepackage{xfrac}
\usepackage{titling}
\usepackage[margin=1in]{geometry}
\usepackage[skins,theorems]{tcolorbox}
\tcbset{highlight math style={enhanced,
  colframe=blue,colback=white,arc=0pt,boxrule=1pt}}

% calculus commands
\renewcommand{\eval}[3]{\left[#1\right]_{#2}^{#3}}

% linear algebra commands
\renewcommand\vec{\mathbf}
\newenvironment{sysmatrix}[1]
{\left[\begin{array}{@{}#1@{}}}
{\end{array}\right]}
\newcommand{\ro}[1]{%
\xrightarrow{\mathmakebox[\rowidth]{#1}}%
}
\newlength{\rowidth}% row operation width
\AtBeginDocument{\setlength{\rowidth}{3em}}

%set theory commands
\newcommand{\pset}[1]{\mathcal P(#1)}
\newcommand{\card}[1]{\operatorname{card}(#1)}
\newcommand{\R}{\mathbb R}
\newcommand{\Q}{\mathbb Q}
\newcommand{\Z}{\mathbb Z}
\newcommand{\N}{\mathbb N}

%number theory commands
\newcommand{\divides}{\mid}
\newcommand{\ndivides}{\nmid}
\newcommand{\fallfact}[2]{#1^{\underline {#2}}}

%optimization commands
\DeclareMathOperator*{\argmax}{arg\,max}
\DeclareMathOperator*{\argmin}{arg\,min}

\setlength{\droptitle}{-7em}   % This is your set screw

\begin{document}

\title{Intro to Real Analysis\\HW \#6}
\author{Ozaner Hansha}
\date{March 22, 2021}
\maketitle

\subsection*{Problem 1}
Let $f:\R\to\R$ be a function that has a limit at 0, and that satisfies $f(x+y)=f(x)f(y)$ for all $x,y\in\R$.
\bigskip

\noindent\textbf{Part a:} Show that $f(x)$ has a limit at every point $x\in\R$.
\bigskip

\noindent\textbf{Solution:} Note the following for an arbitrary $y\in\R$:
\begin{align*}
  L&=\lim_{x\to0}f(x)\tag{limit exists at 0}\\
  f(y)L&=f(y)\lim_{x\to0}f(x)\\
  &=\lim_{x\to0}f(x)f(y)\tag{product respects limit}\\
  &=\lim_{x\to0}f(x+y)\tag{def. of $f$}\\
  &=\lim_{x\to y}f(x)\tag{shift limit}
\end{align*}

And so we have shown that $f(x)$ has a limit for all points $y\in\R$, namely the product $f(y)L$ where $L$ is the limit of $f(x)$ at 0.
\bigskip

\noindent\textbf{Part b:} Show the following:
$$\lim_{x\to0}f(x)=1\vee\lim_{x\to0}f(x)=0$$

\noindent\textbf{Solution:} Note the following:
\begin{align*}
  L&=\lim_{x\to0}f(x)\tag{limit exists at 0}\\
  &=\lim_{x\to0}f(2x)\tag{$x\to0\implies 2x\to0$}\\
  &=\lim_{x\to0}f(x)f(x)\tag{def. of $f$}\\
  &=\lim_{x\to0}f(x)\lim_{x\to0}f(x)\tag{product of limits is limit of products}\\
  &=L^2\tag{limit exists at 0}
\end{align*}

We have no established that, whatever $L$ is, it is equal to its own square $L^2$. Recall that exactly two real numbers satisfy this property: $0$ and $1$. Thus we have that:
$$\lim_{x\to0}f(x)=L\in\{0,1\}$$

\subsection*{Problem 2}
\noindent\textbf{Problem:} Consider the following function for positive integer $n$:
$$f:(0,\infty)\to\R,\,\,f(x)=x^{\sfrac{1}{n}}$$

Prove that this function is continuous over its domain.
\bigskip

\noindent\textbf{Solution:} First recall the reverse triangle inequality for $0<p<1$:
$$||a|^p-|b|^p|\le|a-b|^p$$

Now consider an arbitrary $x_0\in(0,\infty)$, an arbitrary $\epsilon>0$, and let $\delta=\epsilon^n>0$. We have:
\begin{align*}
  0<|x-x_0|<\delta&\implies0<|x-x_0|<\epsilon^n\tag{$\delta=\epsilon^n$}\\
  &\implies |x-x_0|^{\frac{1}{n}}<\epsilon\tag{$n$th root is increasing}\\
  &\implies||x|^{\frac{1}{n}}-|x_0|^{\frac{1}{n}}|<\epsilon\tag{reverse triangle inequality}\\
  &\implies|x^{\frac{1}{n}}-x_0^{\frac{1}{n}}|<\epsilon\tag{domain is positive}
\end{align*}

This is precisely the definition of a continuous function over the domain $(0,\infty)$, and so we are done.

\subsection*{Problem 3}
\noindent\textbf{Part a:} Let $(b_n)_{n=1}^\infty$ be a sequence of rational numbers converging to $b$. Show that it is a Cauchy sequence.
\bigskip

\noindent\textbf{Solution:} Note that whenever we have:
$$\left|\frac{p_n}{q_n}-\frac{p_m}{q_m}\right|<\frac{1}{N}$$

Then we must have:
\begin{align*}
  \left|a^{\frac{p_n}{q_n}}-a^{\frac{p_m}{q_m}}\right|&=\frac{p_m}{q_m}\left|a^{\frac{p_n}{q_n}-\frac{p_m}{q_m}}-1\right|\tag{all positive}\\
  &<\frac{p_m}{q_m}\cdot\max\{a^{\frac{1}{N}-1},1-a^{-\frac{1}{N}}\}
\end{align*}

Yet recall that:
\begin{gather*}
  \lim_{n\to\infty}\sqrt[n]{a}=1\\
  \lim_{n\to\infty}\sqrt[n]{\frac{1}{a}}=1
\end{gather*}

And so we can always find an $N$ large enough that $\frac{p_m}{q_m}\cdot\max\{a^{\frac{1}{N}-1},1-a^{-\frac{1}{N}}\}$ is as close to an arbitrary $\epsilon>0$.
\bigskip

\noindent\textbf{Part b:} Let $(b_n)_{n=1}^\infty$ and $(b'_n)_{n=1}^\infty$ be two sequences of rational numbers both converging to $b$. Show that $(a^{b_n})_{n=1}^\infty$ and $(a^{b'_n})_{n=1}^\infty$ have the same limit.
\bigskip

\noindent\textbf{Solution:} Note that whenever we have:
$$\left|\frac{p_n}{q_n}-\frac{p'_n}{q'_n}\right|<\frac{1}{N}$$

Then we must have:
\begin{align*}
  \left|a^{\frac{p_n}{q_n}}-a^{\frac{p'_n}{q'_n}}\right|&=\frac{p'_n}{q'_n}\left|a^{\frac{p_n}{q_n}-\frac{p'_n}{q'_n}}-1\right|\tag{all positive}\\
  &<\frac{p'_n}{q'_n}\cdot\max\{a^{\frac{1}{N}-1},1-a^{-\frac{1}{N}}\}
\end{align*}

Yet recall that:
\begin{gather*}
  \lim_{n\to\infty}\sqrt[n]{a}=1\\
  \lim_{n\to\infty}\sqrt[n]{\frac{1}{a}}=1
\end{gather*}

And so we can always find an $N$ large enough that $\frac{p'_n}{q'_n}\cdot\max\{a^{\frac{1}{N}-1},1-a^{-\frac{1}{N}}\}$ is as close to an arbitrary $\epsilon>0$.
\bigskip

\subsection*{Problem 4}
For some real $b>0$, consider the function:
$$f:(0,\infty)\to\R,\quad f(x)=x^b$$

\noindent\textbf{Part a:} Show that $f(x)$ is an increasing function.
\bigskip

\noindent\textbf{Solution:} We already know that $x^b$ is an increasing function for $b\in\Q^+$. Call this lemma 1. Now consider a cauchy sequence $(b_n)_{n=1}^\infty$ that converges to $b\in\R$ where each $b_n\in\Q^+$. Note the following:
\begin{align*}
  x>y&\implies (\forall n\in\Z^+),\,\,x^{b_n}>y^{b_n}\tag{lemma 1}\\
  &\implies \lim_{n\to\infty}x^{b_n}\ge\lim_{n\to\infty}y^{b_n}\tag{limits exist \& respect inequalities}\\
  &\implies x^b\ge y^b\tag{problem 3}
\end{align*}

And so we have shown that for any real $b>0$, the function $x^b$ is increasing.
\bigskip

\noindent\textbf{Part b:} Show that $f(x)$ is continuous over $\R^+$. 
\bigskip

\noindent\textbf{Solution:} First let us note the reverse triangle inequality for $p>1$:
$$2^{p-1}||a|^p-|b|^p|\le|a-b|^p$$

Consider the case where $0<b<1$. Now consider an arbitrary $x_0\in(0,\infty)$, an arbitrary $\epsilon>0$, and let $\delta=\epsilon^{\sfrac{1}{b}}>0$. We have:
\begin{align*}
  0<|x-x_0|<\delta&\implies0<|x-x_0|<\epsilon^{\sfrac{1}{b}}\tag{$\delta=\epsilon^{\sfrac{1}{b}}$}\\
  &\implies |x-x_0|^b<\epsilon\tag{$x^b$ is increasing, part a}\\
  &\implies||x|^b-|x_0|^b|<\epsilon\tag{reverse triangle inequality $0<p<1$}\\
  &\implies|x^b-x_0^b|<\epsilon\tag{domain is positive}
\end{align*}

Finally, consider the case where $b>1$. Again we consider an arbitrary $x_0\in(0,\infty)$, an arbitrary $\epsilon>0$, and we now let $\delta=(2^{b-1}\epsilon)^{\frac{1}{b}}>0$. We have:
\begin{align*}
  0<|x-x_0|<\delta&\implies0<|x-x_0|<(2^{b-1}\epsilon)^{\frac{1}{b}}\tag{$\delta=(2^{b-1}\epsilon)^{\frac{1}{b}}$}\\
  &\implies |x-x_0|^b<2^{b-1}\epsilon\tag{$x^b$ is increasing, part a}\\
  &\implies2^{b-1}||x|^b-|x_0|^b|<2^{b-1}\epsilon\tag{reverse triangle inequality $p>1$}\\
  &\implies|x^b-x_0^b|<\epsilon\tag{domain is positive}
\end{align*}

Since these two cases are exhaustive for $b>0$, we have from the definition of continuity that $f(x)$ is continuous on its domain.

\subsection*{Problem 5}
Consider the following function:
$$f:\R\to\R,\quad f(x)=e^x$$

\noindent\textbf{Part a:} Show that $f(x)$ is an increasing function.
\bigskip

\noindent\textbf{Solution:} Note:
\begin{align*}
  y>x&\implies y-x>0\\
  &\implies e^{y-x}-1>0\\
  &\implies e^x(e^{y-x}-1)>0\\
  &\implies e^y-e^x>0\\
  &\implies e^y>e^x
\end{align*}
\bigskip

\noindent\textbf{Part b:} Show that $\forall x,y\in\R$ we have:
$$f(x+y)=f(x)f(y)$$

\noindent\textbf{Solution:} Note:
\begin{align*}
  e^xe^y&=\lim_{n\to\infty}\left(1+\frac{x}{n}\right)^n\lim_{n\to\infty}\left(1+\frac{y}{n}\right)^n\\
  &=\lim_{n\to\infty}\left(1+\frac{x+y}{n}+\frac{xy}{n^2}\right)^n\\
  &=\lim_{n\to\infty}\left(1+\frac{x+y}{n}\right)^n\\
  &=e^{x+y}
\end{align*}
\bigskip

\noindent\textbf{Part c:} Show that $f(x)$ is continuous over $\R$.
\bigskip

\noindent\textbf{Solution:} Note:
\begin{align*}
  &1+x\le e^x\le\frac{1}{1-x}\\
  \implies&\lim_{x\to0}1+x\le\lim_{x\to0}e^x\le\lim_{x\to0}\frac{1}{1-x}\\
  \implies&1\le\lim_{x\to0}e^x\le1\\
  \implies&\lim_{x\to0}e^x=1\\
\end{align*}

Which is to say $f(x)$ has a limit at 0, namely 1. Recall from problem 1 that this, plus part b, imply that $f(x)$ is continuous.

\subsection*{Problem 6}
Consider the logarithm function:
$$f:(0,\infty)\to\R,\quad f(x)=\ln x$$

\noindent\textbf{Part a:} For any positive number $x$, show that there is a unique $y$ such that $e^y=x$.
\bigskip

\noindent\textbf{Solution:} Recall that we have already shown that $e^x$ is both continuous and strictly increasing. The strictly increasing implies that it is injective, while being continuous over $\R$ implies it is surjective over $\R$. An injective, surjective function has a bijection by the Cantor-Bernstein theorem. A bijective function must have an inverse (i.e. $\ln x$) such that $f^{-1}(f(x))=x$ In other words, for any $x$, there is always a unique $y$ such that $e^y=x$.
\bigskip

\noindent\textbf{Part b:} Prove that $\ln$ is increasing.

\noindent\textbf{Solution:} Note that $\ln x$ is the inverse of $e^x$. Also note that $e^x$ is strictly increasing. As a result $\ln x$ must be strictly monotone. Whether it is increasing or decreasing can simply be tested:
\begin{align*}
  e^0=1&\implies\ln 1=0\\
  e^1=e&\implies\ln e=1
\end{align*}

Since $2<e<3$, i.e. $1<e$, the $\ln$ function must be increasing.
\bigskip
\bigskip

\noindent\textbf{Part c:} Prove that $\ln$ is continuous.
\bigskip

\noindent\textbf{Solution:} Consider an arbitrary $\epsilon>0$ and let $\delta=x_0(e^\epsilon-1)$:
\begin{align*}
  |x-x_0|&<\delta\\
  &<x_0(e^\epsilon-1)\tag{def. of $\delta$}\\
  x-x_0&<x_0(e^\epsilon-1)\tag{positive domain}\\
  x&<x_0(e^\epsilon-1)+x_0\\
  x&<x_0(e^\epsilon-1+1)\\
  x&<x_0e^\epsilon\\
  \ln x&<\ln x_0e^\epsilon\tag{$\ln$ is increasing, part b}\\
  \ln x&<\ln x_0\ln e^\epsilon\tag{$e^{x+y}=e^xe^y$}\\
  \ln x-\ln x_0&<\ln e^\epsilon\\
  \ln x-\ln x_0&<\epsilon\\
  |\ln x-\ln x_0|&<\epsilon\tag{positive domain}
\end{align*}

And with that we are done.

\subsection*{Problem 7}

\noindent\textbf{Part a:} Compute the following limits for positive $x$:
\begin{gather*}
  \lim_{x\to0} ((1+x)^{\sfrac{1}{3}}-x^{\sfrac{1}{3}})\\
  \lim_{x\to\infty} ((1+x)^{\sfrac{1}{3}}-x^{\sfrac{1}{3}})
\end{gather*}

\noindent\textbf{Solution:} Since $\frac{1}{3}$ is a positive rational less than 1, it follows from part b below that:
\begin{gather*}
  \lim_{x\to0} ((1+x)^{\sfrac{1}{3}}-x^{\sfrac{1}{3}})=1\\
  \lim_{x\to\infty} ((1+x)^{\sfrac{1}{3}}-x^{\sfrac{1}{3}})=0
\end{gather*}
\bigskip

\noindent\textbf{Part b:} Compute the following limits for a positive rational $\frac{p}{q}<1$, and positive $x$:
\begin{gather*}
  \lim_{x\to0} ((1+x)^{\sfrac{p}{q}}-x^{\sfrac{p}{q}})\\
  \lim_{x\to\infty} ((1+x)^{\sfrac{p}{q}}-x^{\sfrac{p}{q}})
\end{gather*}

\noindent\textbf{Solution:} First we compute the first limit:
\begin{align*}
  \lim_{x\to0}((1+x)^{\sfrac{p}{q}}-x^{\sfrac{p}{q}})&=\lim_{x\to0}(1+x)^{\sfrac{p}{q}}-\lim_{x\to0}x^{\sfrac{p}{q}}\tag{$x^c$ is continuous, problem 4}\\
  &=(\lim_{x\to0}(1+x))^{\sfrac{p}{q}}-(\lim_{x\to0}x)^{\sfrac{p}{q}}\tag{limit of power is power of limit}\\
  &=1^{\sfrac{p}{q}}-0^{\sfrac{p}{q}}\\
  &=1
\end{align*}

Now we compute the second limit via the squeeze theorem. First we find an upper bound:
\begin{align*}
  (1+x)^{\sfrac{p}{q}}-x^{\sfrac{p}{q}}&=||1+x|^{\frac{p}{q}}-|x|^{\frac{p}{q}}|\tag{$x\in\R^+$}\\
  &\le|1+x-x|^{\sfrac{p}{q}}\tag{reverse triangle inequality $0<p<1$}\\
  &=1
\end{align*}

Next we find a lower bound:
\begin{align*}
  (1+x)^{\sfrac{p}{q}}-x^{\sfrac{p}{q}}&\ge\frac{1}{(1+x)^{\sfrac{p}{q}}-x^{\sfrac{p}{q}}}\\
  &=\frac{1}{||1+x|^{\frac{p}{q}}-|x|^{\frac{p}{q}}|}\tag{$x\in\R^+$}\\
  &\ge\frac{1}{|1+x-x|^{\sfrac{p}{q}}}\tag{reverse triangle inequality $0<p<1$}\\
  &=1
\end{align*}

And so, applying the squeeze theorem we have:
\begin{align*}
  &1\le(1+x)^{\sfrac{p}{q}}-x^{\sfrac{p}{q}}\le 1\\
  \implies&\lim_{x\to\infty}1\le\lim_{x\to\infty}((1+x)^{\sfrac{p}{q}}-x^{\sfrac{p}{q}})\le \lim_{x\to\infty} 1\tag{squeeze theorem}\\
  \implies&1\le\lim_{x\to\infty}((1+x)^{\sfrac{p}{q}}-x^{\sfrac{p}{q}})\le1\\
  \implies&\lim_{x\to\infty}((1+x)^{\sfrac{p}{q}}-x^{\sfrac{p}{q}})=1\\
\end{align*}
\end{document}
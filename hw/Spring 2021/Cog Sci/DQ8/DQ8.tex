\documentclass{article}
\usepackage[dvipsnames]{xcolor}
\usepackage{amsmath}
\usepackage{graphicx}
\usepackage{enumitem}
\usepackage{centernot}
\usepackage{setspace}
\usepackage[margin=0.95in]{geometry}
\usepackage{titling}

\setlength{\droptitle}{-7em}   % This is your set screw

\begin{document}

\title{Seminar in Cognitive Science\\ DQ \#8}
\author{Ozaner Hansha}
\date{April 1, 2021}
\maketitle

\subsection*{Question 1}
\noindent\textbf{Prompt:} In order to communicate via speech, is it necessary that the first stage of processing be categorization? Why or why not? (Give evidence for your answer from the papers.)
\bigskip

\noindent\textbf{Response:} Halt \& Lotto mention that the evidence for speech categorization as a necessary stage in speech processing is not strong. They mention examples of speech disorders like Broca's aphasia which demonstrate that impairment in speech recognition can be independent of impairment of syllable/phoneme perception.

It is conceivable that speech recognition makes heavy use of lower level auditory data rather than solely more abstracted concepts like phonemes.

\subsection*{Question 2}
\noindent\textbf{Prompt:} Describe the relation between top-down and bottom-up processes in speech perception. Why is top-down processing important to speech perception?
\bigskip

\noindent\textbf{Response:} Bottom-up speech processing is how we might generally first conceive of it. At the lowest level is the auditory signals we receive, which are then processed into more abstract concepts like phonmes, words, and eventually complete thoughts. This serves as the basis of studying phoneme perception.

The top-down approach instead considers the effect our own minds and their expectations have on what we perceive (in this case what we hear). It's no secret that people can be primed to hear certain ambiguous words or sounds on way or the other. Clearly, in these cases, some top-down processing is at play. For a field as complex as speech processing, an understanding of the top-down processes used in language perception can help capture the nuance lost in just exploring the bottom-up abstractions.

\subsection*{Question 3}
\noindent\textbf{Prompt:} Historically, what was the assumed role of the primary sensory cortices? What do we believe today about these cortices? Describe one of the sources of evidence described in Weinberger (2015) that led to this change.
\bigskip

\noindent\textbf{Response:} Largely in part due to Campbell (1905) the primary sensory cortices, such as the primary auditory cortex (A1) discussed in Weinberger 2015, have been ascribed the purpose of ``pure sensory analysis.'' Such a role would have presumably precluded it from acting in learning, memory, and behavioral processes. However much evidence has been shown to the contrary. One example given in the paper is by Galambos (1956) where a cat was classically conditioned to associate a puff of air in its eyes with a clicking noise. It was found that, once conditioned on the click, the cat's A1 region lit up in response to it. This would imply that this region plays some role in behavioral learning and not just ``pure sensory analysis.''

\subsection*{Question 4}
\noindent\textbf{Prompt:} What is representational plasticity and how does it differ from physiological plasticity? Explain one consequence or outcome that differs between the two different forms of plasticity.
\bigskip

\noindent\textbf{Response:} Physiological plasticity refers to some change in the nervous system, here a sensory cortex, in response to some stimulus or an environment. Representational plasticity (RP) refers to a `systematic change' in how some sensory stimuli is encoded. This is to say that RP is more powerful in learning, as it can bias how stimuli/environments in general are processed not just particular ones like with physiological plasticity.

\subsection*{Question 5}
\noindent\textbf{Prompt:} Explain the ``specificity problem'' in plasticity and memory and how this problem was solved. What new experimental measurements/procedures were used to deal with the specificity problem?
\bigskip

\noindent\textbf{Response:} The specificity problem refers to how it is easy to test if a particular audio signal had affected the auditory cortex, it was unclear in which way it did so. For example, would audio signals of similar frequency but different in some other dimension arouse the same reaction?

To solve this, researchers obtained auditory receptive fields both before and after subjects were trained with some sound. This allows them to compare the receptive fields and see how sound perception might have been altered by the training.
\end{document}
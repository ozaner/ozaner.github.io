\documentclass{article}
\usepackage[dvipsnames]{xcolor}
\usepackage{amsmath}
\usepackage{graphicx}
\usepackage{enumitem}
\usepackage{centernot}
\usepackage{setspace}
\usepackage[margin=0.95in]{geometry}
\usepackage{titling}

\setlength{\droptitle}{-7em}   % This is your set screw

\begin{document}

\title{Seminar in Cognitive Science\\ DQ \#10}
\author{Ozaner Hansha}
\date{April 15, 2021}
\maketitle

\subsection*{Question 1}
\noindent\textbf{Prompt:} Distinguish between the study of stereotype content and stereotype processes. What are the pros and cons of each of these approaches?
\bigskip

\noindent\textbf{Response:} Studying stereotype content involves gauging the particular sentiment different groups may have about others. These sentiments are usually expressed in terms of `warmth' (i.e. how friendly/unfriendly this group is) and `competence' (i.e. how capable members of this group are to act upon this (un)friendlyness). For example, elderly people are often viewed as friendly or nice, thus giving them high warmth. But they are also viewed as incompetent due to their old age, giving them low competency.

The study of stereotype processes is more interested in the underlying principles that give rise to different stereotypes of different groups. The hope is that understanding these psycho-social processes will give greater insight into the specific stereotypes we see today. A more top-down approach than stereotype content.

The content approach gets at the heart of the matter, what stereotypes exist, but they miss the bigger picture of why and how stereotypes come about. As social dynamics change, stereotypes come and go, but the underlying dynamics of their appearance (i.e. human psychology) doesn't.

\subsection*{Question 2}
\noindent\textbf{Prompt:} What is the social structure hypothesis? Explain the two proposals the hypothesis makes about perceived competence and warmth of outgroups.
\bigskip

\noindent\textbf{Response:} The social structure hypothesis posits that out-groups are perceived as more competent if they hold high social status/power and less competent if they occupy a low social status/have little power. Further, the perceived correspondence between the groups competence and its social status/power reinforces people's beliefs about the social system. For example, rich people have high social status and power (afforded by wealth), this leads people to view them as more competent and thus more deserving of said wealth and status.

\subsection*{Question 3}
\noindent\textbf{Prompt:} What are glial cells, and how are they different from neurons? Name the three kinds of glial cells and their functions.
\bigskip

\noindent\textbf{Response:} Glial cells are known to act as `support cells' that maintain neurons in optimal condition, however more recently they have been postulated to do much more.

The three major types of glial cells, and some of their known functions, are:
\begin{itemize}
    \item Astrocytes: Regulate transmission between neurons, altering physical space between neurons, delivering nutrients to neurons, and controlling blood flow.
    \item Oligodendrocytes: Increase the speed of transmission through nerve axons.
    \item Microglia: Prune synapses by monitoring transmission, rewriting neural connections. 
\end{itemize} 

\subsection*{Question 4}
\noindent\textbf{Prompt:} What are astrocytes and why do neuroscientists think they might be important in learning?
\bigskip

\noindent\textbf{Response:} As mentioned before astrocytes are a type of lial cell that has a variety of functions, in particular modifying how neurons communicate with one another. This glial cell type in particular seems to have a direct role in learning. The signaling of astrocytes have been shown to be related to plasticity, and modifying transmissions by either exciting or inhibiting. 

\subsection*{Question 5}
\noindent\textbf{Prompt:} What are tripartite synapses? Describe the three components of these structures and their function.
\bigskip

\noindent\textbf{Response:} Tripartite synapses refer to the structure comprised of the pre-synaptic neuron, post synaptic neuron, and astrocyte process, that comprise the functioning of the chemical synpase (i.e. the transmission of neurotransmitters from the axon to the dendrite.)

The pre-synaptic membrane refers to the membrane of the axon of a neuron facing the dendrite of another. The post-synaptic membrane refers to the membrane of the dendrite that faces the pre-synaptic membrane.
\end{document}
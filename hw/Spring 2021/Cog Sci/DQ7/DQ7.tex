\documentclass{article}
\usepackage[dvipsnames]{xcolor}
\usepackage{amsmath}
\usepackage{graphicx}
\usepackage{enumitem}
\usepackage{centernot}
\usepackage{setspace}
\usepackage[margin=0.95in]{geometry}
\usepackage{titling}

\setlength{\droptitle}{-7em}   % This is your set screw

\begin{document}

\title{Seminar in Cognitive Science\\ DQ \#7}
\author{Ozaner Hansha}
\date{March 25, 2021}
\maketitle

\subsection*{Question 1}
\noindent\textbf{Prompt:} What are dual-process theories of reasoning? How do the two processes posited by these models map on to the two types of games studied in the reading?
\bigskip

\noindent\textbf{Response:} Dual-process theories explain how our thoughts arise as a result of two different processes: an automatic, intuitive process that is fast and emotional, and a controlled, conscious one that is slow a deliberative. These are neatly analogous to the two different types of game the paper explores, with intuitive reasoning needed to solve pure coordination games (by imagining what other minds are thinking) and deliberative reasoning needed to solve dominance-solvable ones (by applying an algorithm onto/simulating outcomes of different strategies).

\subsection*{Question 2}
\noindent\textbf{Prompt:} What are focal points? What role do they play in coordination games?
\bigskip

\noindent\textbf{Response:} Making use of focal points is a common strategy that arises in pure coordination games. They are ideas that are distinguished or common (distinguished in their commonality.. I suppose) in the collective headspace and thus serve as a useful goto. Salient is the term the paper uses. For example, in a game where two players must guess the same number, yet cannot communicate, the number `1' serves as a focal point since it is special, or salient, as a number. Similar things can be said about red (if the game was about colors) or the current year (if it was about years).

\subsection*{Question 3}
\noindent\textbf{Prompt:} What are phonemes? Describe the challenge of phoneme perception, providing an example from the readings. Why is it a surprising challenge?
\bigskip

\noindent\textbf{Response:} Phonemes are the minimal unit of sound that distinguishes two words in a spoken language. For example, ``help'' and ``held'' are two different words, audibly distinguished by their last phoneme, /p/ vs. /d/.

Phoneme perception is not as simple as one would first imagine. The exact same sound can be perceived to be a different phoneme depending on the speaker, the listener, and the surronding context of the phoneme. A well studied example of this is how infants of any background seem to be able to discriminate between different phonemes of different languages but, by age 1, their phoneme perception has been `warped' and they can no longer distinguish between phonemes that play no role in their language (e.g. a Japanese 1 year old would no longer be able to distinguish between the /r/ and /l/ phonemes, while an American 1 year old would).

This is surprising as one would expect that these individual units of language are totally encapsulated by how they sound, and that a simple pattern mathching program could identify them via audio alone. This, however, is not the case. 

\subsection*{Question 4}
\noindent\textbf{Prompt:} What does it mean that speech perception is categorization, but not categorical? What are the three hallmarks of categorical perception?
\bigskip

\noindent\textbf{Response:} Speech perception is categorization in that it is a process by which the listener categorizes the audio they hear (along with any context) into distinct categories (e.g. he just said the word `alligator' and not anything similar to it like `allegation').

This is different from the statement that speech perception is categorical. An example of this latter claim is how abruptly the perception of some phonemes change (e.g. /ba/) amongst a group of subjects as it varies across some dimension, say frequency. One would expect some sort of continuum of changing perceptions as the acoustic dimension is perterbed but instead we see a nearly perfect discontinuous border at which subjects perceive it as one or the other.

This abruptness in classification, and discontinuity of perception vs pertbation are both hallmarks of categorical perception. The third hallmark is that performance in categorization, that is generally being able to identify different phonmes/words/speech, predicts performance categorical discrimination.

\subsection*{Question 5}
\noindent\textbf{Prompt:} What are the three theories of speaker normalization addressed in Johnson, Strand \& D'Imperio (1999)? Describe them.
\bigskip

\noindent\textbf{Response:} The three theories to speaker normalization they discuss are given below;
\begin{enumerate}
    \item \textbf{Radical invariance:} This theory holds that speaker normalization is a purely auditory effect. While the specifics depend on the sub-theory, they all hold that there is some mapping of acoustic space (the objective space of sounds a listener can hear) to one in which all the different, say vowel sounds, all map to the same portion. They also hold, of course, that it is precisely this representation that humans use in listening to others.
    \item \textbf{Vocal tract normalization:} This theory holds that the listener essentially models the length of the speaker's vocal tract by use acoustic cues. Once this is known, different sounds can be reverse engineered with this length, however the brain may perform this.
    \item \textbf{Talker normalization:} This theory is similar to the previous except that normalization takes place over more than just the vocal tract length. Familiarity, socio-cultural expectations, and gender of the talker, to name some, all impact how we perceive and discern their speech.
\end{enumerate}
\end{document}
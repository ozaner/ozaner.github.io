\documentclass{article}
\usepackage[dvipsnames]{xcolor}
\usepackage{amsmath}
\usepackage{graphicx}
\usepackage{enumitem}
\usepackage{centernot}
\usepackage{setspace}
\usepackage[margin=0.95in]{geometry}
\usepackage{titling}

\setlength{\droptitle}{-7em}   % This is your set screw

\begin{document}

\title{Seminar in Cognitive Science\\ DQ \#1}
\author{Ozaner Hansha}
\date{January 28, 2021}
\maketitle

\subsection*{Question 1}
\noindent\textbf{Prompt:} What, broadly, is the claim of the cognitive enrichment hypothesis? Describe Hertzog, et al.’s version of the hypothesis. Contrast their thesis with an alternative picture of the relation between individual action and cognitive improvements.
\bigskip

\noindent\textbf{Response:} The cognitive enrichment hypothesis is the claim that behaviors that promote cognitive activity (e.g. social engagement, exercise, learning new skills) can produce meaningful and positive effect in the the cognitive functioning of adults in old age. Such a claim may seem uncontroversial or maybe even obvious, but we can highlight how it should not be taken for granted by considering the reverse situation. Indeed, could it not be the case that older adults that already happen to have higher cognitive functioning are those that happen to take part in behaviors that promote of cognitive activity? To be sure one way or the other experiments must be done.

\subsection*{Question 2}
\noindent\textbf{Prompt:} Describe in detail at least one methodology for evaluating hypotheses about cognitive enrichment and include a particular example of a study discussed in the readings. Discuss at least two advantages or disadvantages to the approach.
\bigskip

\noindent\textbf{Response:} One way to test the cognitive enrichment hypothesis is to gather two groups of elderly people, one as the experimental group and one as a control, and have the experimental group take part in activities and behaviors that require considerable cognitive activity. Both groups' would have their cognitive function measured (whether it by imaging the brain, using a slew of metrics/tests, or both) both before and after the activities. A increase in cognitive ability, as measured by the tests, in the experimental group and a lack of increase in the control group would seem to demonstrate the hypothesis.

Such a study has already taken place, dubbed the ACTIVE trial, it included 2832 partipants from ages 65 to 94. The results showed that each session of activities (in this case group training in memory, reasoning, processing, etc.) had the cognitive abilities of the particpants increase.

That said, such an approach fails to isolate certain confounding factors. For example, was it just more social engagement that increased the cognitive functioning of the particpants? The approach also does not pinpoint a particular type or level of cognitive engagement as more effective than any other. For example, is too much engagement harmful?

\subsection*{Question 3}
\noindent\textbf{Prompt:} What kind of activities did Park et al. find did not significantly improve cognitive function? How did their study show this? Make sure your answer discusses the difference between activities that involve productive engagement and those that involve receptive engagement.
\bigskip

\noindent\textbf{Response:} One of the main goals of the Synapse Project, conducted by Park et al., was to test the difference (if any) between different types of cognitive engagement and their effects on old people's resulting cognitive functioning. This first is \bf productive engagement \normalfont which involves the acquisition of new skills and ideas, making use of working memory, long-term memory, and executive processes like planning and control. The second is \bf receptive engagement\normalfont, which involves less novelty than the former kind and is instead based on engaging with the individuals existing ideas and schemas, making use of their automatic memory and their feelings of familiarity with the topic.

The results of the experiment support with the group's initial hypothesis: productive engagement improves cognitive functioning more so than receptive engagement since it engages more cognitive processes. The experiment also accounts for social engagement via its social group (which only had social engagement) which did not perform as well as the productive engagement group.
\end{document}
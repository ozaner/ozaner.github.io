\documentclass{article}
\usepackage[dvipsnames]{xcolor}
\usepackage{amsmath}
\usepackage{graphicx}
\usepackage{enumitem}
\usepackage{centernot}
\usepackage{setspace}
\usepackage[margin=0.95in]{geometry}
\usepackage{titling}

\setlength{\droptitle}{-7em}   % This is your set screw

\begin{document}

\title{Seminar in Cognitive Science\\ DQ \#2}
\author{Ozaner Hansha}
\date{February 4, 2021}
\maketitle

\subsection*{Question 1}
\noindent\textbf{Prompt:} Explain why studying individual differences is productive? How did the Park et al. study quantify individual differences?
\bigskip

\noindent\textbf{Response:} Humans widely vary in the level of their abilities and skills, some being better at some tasks than others. These differences are ultimately, maybe with exception to physical abilities, a result of differences between our brains, whether that be encoded in the particular wiring of the brain or by its general structure. Thus, by honing in on the differences between people's abilities, we can indirectly study the brain and how these differences relate to it. 

\subsection*{Question 2}
\noindent\textbf{Prompt:} Explain the challenge of selection bias in older adult populations.
\bigskip

\noindent\textbf{Response:} Simply trying to study older populations can easily lead to selection bias. For instance, older adults that are less physically able may not able to go out to a group training exercise, like in the SYNAPSE project. This skews the pool of potential subjects away form those who are less able, which may be related to how cognitively able they are.

And even amongst able bodied older adults, whether or not they are cognitively active might relate to whether they are proactive enough to consent to partipating in a particular study. This skews the pool of potential subjects towards more proactive older adults.

\subsection*{Question 3}
\noindent\textbf{Prompt:} What is the phenomenon of motivated cognition? Describe it and provide an example. What are ``the three principles"? What does it reveal about the relationship between top-down processes and low-level cognition?
\bigskip

\noindent\textbf{Response:} Motivated cognition is the phenomenon by which we form our evaluations and perceptions about the world around us in a manner that is biased towards what we would wish to be the case.

For example, people viewing their presidential hopeful as having a lead over the other, maybe by paying more attention to more favorable polls than others.

There are three principles that characterize motivated cognition:
\begin{itemize}
	\item It's pervasive. There are so many reasons, or motives, for us to make unrealistic judgments. For example, we may wish to believe we are above-average in some or all dimensions. This may lead us to believe so, maybe by passing off inadequacies as flukes and overexaggerating achievements. 
	\item It's goal-directed. There is a correspondence between what judgments we skew in our favor and what goals these skewed judgement would support.
	\item It's impactful. These skewed judgments have real world consequences for how we behave. For example, being more optimistic about oneself may lead to more confidence and thus more successful relationships. However an overly optimistic view of oneself can also lead to undue risk taking or being perceived as arrogant.
\end{itemize}

Motivated cognition is an example of top-down processing. Top-down processing is when our already formed model of the world informs our further perception of it, thus skewing it towards what we expect rather than how it actually is. This phenomenon seems to pervade even low-level cognition like that of sight, in so far as it pertains to the skewed judgments we form.

\subsection*{Question 4}
\noindent\textbf{Prompt:} Why is it important that stimuli presented to participants are ambiguous in the categorization tasks discussed in the readings?
\bigskip

\noindent\textbf{Response:} If either of the different ways to perceive the stimuli were clearer than the other, i.e. the stimuli was less ambiguous, then the results would be skewed towards that particular categorization of the stimuli. To best demonstrate motivated perception, the stimuli should be ambiguous so as to put the motivated perceptual process center stage and have the results depend on it more than anything else. 

\subsection*{Question 5}
\noindent\textbf{Prompt:} Describe the two possible psychological mechanisms suggested to underlie ``wishful seeing". In each case, describe evidence (i.e. an experimental finding) that suggests that mechanism's involvement.
\bigskip

\noindent\textbf{Response:} The first mechanism the paper describes is \textit{perceptual sets}. These are mental states that the subject may have already been in or associations that have been primed before the stimuli has even come into view. These associations can then subtly lead one to skew what they are viewing to something related to that perceptual set.

An experimental example of this is that thirsty people, presumably because they are thinking about water, perceived stimuli as being more transparent in an unrelated ambiguous transparency test.

The second mechanism the paper mentions is \textit{attention}. Paying more attention to a particular part of the environment may alter our perception of it or make us hyperaware of what it contains.

An experimental example of this is that particpants viewing words on a computer screen were able to better recognize those words when a picture of a delicious dessert was close to them, presumably because they were paying attention to areas with the dessert more so than the areas without them.
\end{document}
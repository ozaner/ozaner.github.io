\documentclass{article}
\usepackage[dvipsnames]{xcolor}
\usepackage{amsmath}
\usepackage{graphicx}
\usepackage{enumitem}
\usepackage{centernot}
\usepackage{setspace}
\usepackage[margin=0.95in]{geometry}
\usepackage{titling}

\setlength{\droptitle}{-7em}   % This is your set screw

\begin{document}

\title{Seminar in Cognitive Science\\ DQ \#3}
\author{Ozaner Hansha}
\date{February 11, 2021}
\maketitle

\subsection*{Question 1}
\noindent\textbf{Prompt:} What were the purposes of the two pilot studies discussed in the Cole et al. paper? 
\bigskip

\noindent\textbf{Response:} The first pilot study had two goals. The first was to check whether the particpants even noticed that the faces were digitally altered for this purpose. The next was to see if particpants chose faces more that lined up with the definition of 'attractiveness' the researchers had used.

In terms of the first goal, while the participants did seem to notice something off between the faces they never mentioned that one was any more attractive than another. Regarding the second goal, the partipcants on average chose faces that were digitally altered to be more attractive, confirming the researchers attractiveness method.

The goal of the second pilot study was to see if the task of choosing the correct face was ambiguous enough for motivated biases to emerge from participants. If the task was too easy, e.g. the altered faces look much different to the real one, then there would be no room for bias in the participant's choice.

The partipcants did indeed find the task challenging, i.e. ambiguous, and choose the more attractive faces far more than the less attractive ones. Confirming that this is a reasonable way to allow for bias to emerge in participants.

\subsection*{Question 2}
\noindent\textbf{Prompt:} Describe the visual matching task used in the Cole et al. studies, including how the researchers measured participant responses in Study 1 vs. Study 2.
\bigskip

\noindent\textbf{Response:} The visual matching task used in the study had the particpants pick a face out of 11 (1 being the original face, 5 being altered to be less attractive and 5 being altered to be more attractive) that matched the given reference face. They had an unlimited amount of time to perform this task.

In study 1 particpants were given a choice of altered faces that had a 10\% gradation on their attractiveness, and were told that the target was either single or dating.

This confirmed the hypothesis that those participants in relationships viewed single targets as less attractive than usual. To account for other possible hypothesis (e.g. maybe the fact that they are single means they are less attractive) study 2 changed the parameters slightly.

Now the attractiveness gradations were at 7\% and the particpants were always told that the target was single. The participants were also told that the target was either interested in dating or not interested, serving as a threat to partipants in a relationship while avoiding the status of dating or not. The results were, again, in favor of the hypothesis of perceptual downgrading.

\subsection*{Question 3}
\noindent\textbf{Prompt:} What are some of the goals in designing presentations of information for ill patients? How do these goals conflict with each other? 
\bigskip

\noindent\textbf{Response:} When designing such a presentation, the information would ideally be easily digestible (especially to a patient stressed about said illness) as well as detailed and accurate in it's presentation of the possible outcomes (so that the patient can make an informed decision).

Making a presentation easily digestible conflicts with it being accurate. While a detailed presentation may help in allowing the patient make an informed decision about how to proceed, the patient may be confused about what information is relevant to them if there is too much of it.

\subsection*{Question 4}
\noindent\textbf{Prompt:} What were the two themes that emerged in the analysis in the Kim et al. paper? Describe an example of a problem that can arise when each of these themes is handled poorly.
\bigskip

\noindent\textbf{Response:} The two themes that arose were ``balancing utility and fear'' and ``establishing authenticity and credibility''.

An example of the first theme being poorly balanced might be having patients newly diagnosed with some illness attend an information session that overloads them with new, and possibly irrelevant, information regarding their illness. While it would seem that telling them about it would be helpful, doing it in a manner that is less stressfull and more personalized would alleviate doubts fear and confusion the pateint may have, which may do more harm than the good of knowing that information.

An example of the second theme being poorly handled is in the calculation of survival rates for a particular treatment. Patients may ask how a particular statistic is calculated, i.e. try to establish the \textit{authenticity} of the data, before trusting it. While a doctor might be able to provide them the algorithm for how it is calculated, it is most likely going to be quite complex and a pateint under the stress of having some illness may not be equipped to understand it. Indeed, they may become even more stressed trying ot make sense of it. In this case we are faced with a problem in establishing authenticity due to the technical barrier of understanding relevant data about the illness.

\subsection*{Question 5}
\noindent\textbf{Prompt:} Describe the principles of user-centered iterative design process (UCD). How was UCD used in design of the materials used in the Kim et al. paper?
\bigskip

\noindent\textbf{Response:} In UCD, the UI is designed with the end user and their goals explicitly in mind. Iterative changes are then made in response to the end users' feedback and evaluation of the UI in a cycle.

In the paper, the UI team gathered requirements from both the patients and clinicians, paying special attention to the needs of the patients. Using this information, prototypes of the UI were made and evaluation sessions were had. One key takeaway was that the pie charts should be simple to understand and ambiguous in what decision should be taken next. 

Based on these sessions, some interactive prototypes were designed which included a pie chart, a vertical bar chart, and a horizontal bar chart. After conducting interview sessions with patients, it was decided that the vertical bar chart was the favorite amongst them.
\end{document}
\documentclass{article}
\usepackage[dvipsnames]{xcolor}
\usepackage{amsmath}
\usepackage{graphicx}
\usepackage{enumitem}
\usepackage{centernot}
\usepackage{setspace}
\usepackage[margin=0.95in]{geometry}
\usepackage{titling}

\setlength{\droptitle}{-7em}   % This is your set screw

\begin{document}

\title{Seminar in Cognitive Science\\ DQ \#4}
\author{Ozaner Hansha}
\date{February 18, 2021}
\maketitle

\subsection*{Question 1}
\noindent\textbf{Prompt:} What were the main findings of the Kim et al. paper? Be sure to discuss the results on different types of visual displays, numeric values, and credibility. 
\bigskip

\noindent\textbf{Response:} The key findings of the paper were three-fold:
\begin{enumerate}
    \item Patients and clinicians have different perspectives on what information is important to convey to the patients. Clinicians seem to focus more on providing information about the patients' condition while the patients seem to find emotional support just as, if no more, important than information about their objective condition.
    \item That the patient's clinical data should be presented in a simple manner for them to make sense of it and be able to make an informed decision.
    \item That visualizing clinical data about a patient is a sensitive process. Giving them too much or having it be conveyed in a way that can be perceived as negative, regardless of its validity, may cause the patient's mental condition to worsen. By being ambiguous about the data, patients can better view it in an optimistic way.
\end{enumerate}

After the team's preliminary requirement gathering for the graphical presentations' design, the differing views between clinicans and patients made finding 1 apparent. After holding informational sessions for patients, their confusion at more complicated sources of information, making them feel even more stressed about their situation, was what finding 2 was borne of. Finally, finding 3 came from asking patients directly for feedback from the different types of charts the team presented them, finding they wanted to be consoled about their condition more so than be accurately informed of it.

\subsection*{Question 2}
\noindent\textbf{Prompt:} What problem did the researchers encounter in assessing patients' feelings toward graphical presentations of data (as opposed to the numerical value of the data)? How did they address this problem? 
\bigskip

\noindent\textbf{Response:} Researchers found when asking patients about whether graphical or numerical representations of the data were better, their response was heavily correlated with how positive the data shown was. This leads to a situation where the responses are more influenced by the data shown on the prototype graphics rather than the graphical method itself.

To remedy this, the researchers showed the patients the graphical formats several times each with different data so that they would focus less on the data being shown and more on the the format itself. This strategy proved successful.

\subsection*{Question 3}
\noindent\textbf{Prompt:} What is the `theory-theory' view of the emergence of Theory of Mind? Distinguish it from the view that Theory of Mind is a distinct, innate mechanism. Briefly describe a piece of evidence that favors one of these views over the other.
\bigskip

\noindent\textbf{Response:} The theory-theory of mind posits that children, by implicitly coming up with hypotheses about the world around them and finding out whether they work in predicting it or not, come to learn to have a theory of mind after much trial and error. This is opposed to the view that the theory of mind is an innate mechanism in humans that manifests at a certain age.

Evidence against the mechanism view is that, despite a theory of mind emerging in two-year-olds, three-year-olds still could not solve the Sally and Anne task. The theory-theory explains this phenomena as just a lack of experience/data to come to the conclusion that people can't know what they haven't been told/seen.

\subsection*{Question 4}
\noindent\textbf{Prompt:} Describe the ``Sally Anne'' false-belief task. How do normal developing children and autistic children behave in false-belief tasks?
\bigskip

\noindent\textbf{Response:} The Sally Anne task has a test subject, usually a young child or infant, watch a pair of dolls, the eponymous Sally and Anne, in a room. Sally places her ball in a basket and departs. Then, unbeknownst to Sally, Anne comes in the room and moves her ball somewhere else and leaves before Sally returns. Sally then returns and before she looks for he ball, the child watching is asked to predict where Sally will look for it. After this, Sally checks the basket for her ball before finding it in its new hiding place.

A child with a fully developed theory of mind should predict that Sally, not knowing that the ball was moved, would check the basket first. However, children without a developed theory of mind would predict that she'd check the new hiding place. Because these children saw where the ball was, they assume Sally would go look there just as they would, ignorant of the fact that Sally is a separate person with separate information and beliefs.

It was found that normally developing four-year-olds succeeded in this task, while four-year-olds with autism did not. This suggests that autistic children may have a less developed theory of mind comapred to neurotypical children.

\subsection*{Question 5}
\noindent\textbf{Prompt:} Describe the ``photographs'' task and the results with preschoolers. Explain the design and results of Leslie and Thaiss (1992) comparing performance of autistic children and normally developing 4-year-olds. What do these results suggest about the cognitive mechanisms responsible for Theory of Mind?
\bigskip

\noindent\textbf{Response:} In the photographs task, a child is told to take a picture of, say, an object on a chair. The picture is then placed face down on a table, without the child having been able to see it. The object on the chair is then replaced and the child is asked what object is on the chair in the picture. As we can see, this experiment is analogous to the Sally Anne task with the crucial difference of Sally's immaterial beliefs being replaced by the more concrete photograph.

The results showed that, while neurotypical children performed better on the Sally Anne task than autistic children, autistic children fared far better than neurotypical children on the photograph task.

This experiment and it's findings give reason to rule out the idea that autistic children are not capable of the Sally Anne task because of some general impairment to their cognitive abilities (e.g. memory, executive action, etc.) and that their failure to succeed in the task is indeed related to their theory of mind, or lack thereof.
\end{document}
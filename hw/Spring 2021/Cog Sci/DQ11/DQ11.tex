\documentclass{article}
\usepackage[dvipsnames]{xcolor}
\usepackage{amsmath}
\usepackage{graphicx}
\usepackage{enumitem}
\usepackage{centernot}
\usepackage{setspace}
\usepackage[margin=0.95in]{geometry}
\usepackage{titling}

\setlength{\droptitle}{-7em}   % This is your set screw

\begin{document}

\title{Seminar in Cognitive Science\\ DQ \#11}
\author{Ozaner Hansha}
\date{April 20, 2021}
\maketitle

\subsection*{Question 1}
\noindent\textbf{Prompt:} Discuss the benefits of studying global changes compared to local (neuron-to-neuron) changes.
\bigskip

\noindent\textbf{Response:} Studying global changes in neuronal structure will lead to a better understanding of activity that takes place on this scale, like the reorganization of the brain indicative of brain plasticity, the organization of memory across the brain, and the processing of visual stimuli that cannot be parsed locally. Studying \textit{local} neuronal activity, on the other hand, leads to an understanding of how they work in conjunction and in smaller networks. 

\subsection*{Question 2}
\noindent\textbf{Prompt:} Describe the way that astrocytes and neurons interact. Be sure to mention the role that Calcium plays.
\bigskip

\noindent\textbf{Response:} Astrocytes can affect large scale neuronal behavior by managing Calcium activity in the brain. The paper describes this action as wave like in nature, with spikes, or waves, in intracellular Calcium cascading across the brain. One use of this is the synchronization of neurons for some purpose.  

\subsection*{Question 3}
\noindent\textbf{Prompt:} What is the function of domain-level expectations in everyday cognition about natural objects, animals, and people? How do religious concepts interact with these domain-level expectations? 
\bigskip

\noindent\textbf{Response:} Domain-level expectations allow people to grapple with the properties and aspects of everyday objects even if they have never seen or heard of them before. While I may have never seen or heard of a tangerine before, upon hearing it is a fruit, I can expect it to be relatively small, juicy and potentially sweet or `fruity` tasting (fruit category). Further, upon hearing it is `sort of like an orange' I can expect it to taste of citrus, be orange in color, etc (orange category). This on the fly thinking is vital for the everyday operation of humans.

\subsection*{Question 4}
\noindent\textbf{Prompt:} Explain the template theorized to be common to successful religious concepts. How do concepts differ from templates? 
\bigskip

\noindent\textbf{Response:} The model the paper gives for a successful religious template consists of:
\begin{enumerate}
    \item A pointer to a particular domain concept
    \item An explicit representation of a violation of intuitive expectations, either:
    \begin{enumerate}
        \item A breach of relevant expectations for the category, or
        \item A transfer of expectations associated with another category
    \end{enumerate}
    \item A link to (non-violated) default expectations for the category.
\end{enumerate} 

The example the paper gives for the ghosts/spirits template is: (1) the person category (ghosts are like people), (2) they have weird physical properties, like phasing through walls, invisible, etc. and (3) they still have minds/personalities, like normal people. Note that concepts are different from templates, with concepts adding two additional points to their model:
\begin{enumerate}
    \item[4.] A slot for additional encyclopedic information
    \item[5.] A lexical label.
\end{enumerate}

For example, our western concept of `ghost' (somethign that falls into, but is more specific than, the generic template of ghost/spirit) would have the following additional info: (4) ghosts often haunt where they died, and (5) the word ``ghost.''

\subsection*{Question 5}
\noindent\textbf{Prompt:} How do violations of intuitive expectations from the domain-level help religious concepts? Describe at least one other factor that is probably necessary for these concepts to be culturally successful.
\bigskip

\noindent\textbf{Response:} Violations (i.e. point 2 of the model) of the intuitive expectations of the concept (i.e. point 1), aid in making the concept more memorable. A regular person follows our normal expectations of the person category (tautologically), but a ghost and its violation of those expectations (namely phasing through walls and being undead) is far more memorable and thus is remembered and communicated more often, leading to its spread as a concept.

Another oft cited property of supernatural beliefs is their power in producing causal explanation of complex or misunderstood phenomena. The belief in a rain god may be more useful to a human/society not advanced enough to know of the water cycle than the existence of a rock God. Rocks aren't nearly as important or mysterious, as weather patterns. The lack of knowledge of said patterns may be frightening, especially since agriculture is contingent on them. As such an explanation of it (in the form of a rain god pleased with the recent actions of humans say) would go a long way to assuage those who believe in it.

\end{document}